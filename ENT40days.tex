% Elementary Number Theory in 40 Days with Prof Allen
% lecture notes taken during the term Spring 2017
% by Yün Han 

\documentclass{amsbook}
\usepackage{mathtools}
\mathtoolsset{showonlyrefs} % only label equations that are referenced; didn't work with cleveref however
\usepackage{amsfonts}
\usepackage{amssymb,latexsym}
% pstricks for high quality graphics 
\usepackage[pdf]{pstricks} % load pstricks with pdf output option
% pdflatex with pstricks without dvi->ps->pdf 
\usepackage[off]{auto-pst-pdf} % off option, Herbert's answer  https://tex.stackexchange.com/a/8415/106930
\usepackage{pst-plot,pst-grad} % pst-plot will load xcolor package automatically
% multi line of long url
% https://tex.stackexchange.com/questions/3033/forcing-linebreaks-in-url
\PassOptionsToPackage{hyphens}{url} % this has to be BEFORE bibtex or amsrefs

\usepackage[margin=3cm]{geometry} % custom margin 
%\usepackage[usenames]{xcolor} % already loaded by pst-plot
\usepackage{enumitem}
\usepackage{hyperref}

% hyperref setup
\hypersetup{pdfstartview={XYZ null null 1.00}, pdfpagemode=UseNone, colorlinks, breaklinks, linkcolor=blue, urlcolor=blue, anchorcolor=blue, citecolor=blue}
%\usepackage[nameinlink,capitalize]{cleveref} % cleveref the only exception loaded AFTER hyperref
%\usepackage{autonum} % label equations that are referenced; loading in order hyperref - cleveref - autonum; also cannot use align* anymore 
\usepackage[makeroom]{cancel} % cancel terms; terms vanish 

% bibliography
\usepackage[alphabetic]{amsrefs} % cite NAME YEAR

% add exercises without loading other packages, instead modify LaTeX macros
% \labelenumi - how label looks like, with/out parentheses etc. (first level item)
% \theenumi - how number looks like, alpha, roman, arabic etc. (first level item)
% exercise numbering : chapter - exercise number
\renewcommand*{\labelenumi}{\thechapter.\arabic{enumi}.}
\renewcommand*{\theenumi}{\labelenumi}
% skip section number in chapter.subsection.subsubsection -> chapter.subsubsection
% \renewcommand*{\subsubsection}{\thechapter.\arabic{section}}

% default group of theorem style
\theoremstyle{plain}
\newtheorem{theorem}{Theorem}[chapter] % reset numbering after chapter; could use \newtheorem{theorem}{Theorem}[section]
%%%%%%%%%%%%%%%%%%%%%%%%%%%%%%%%%%%%%%%%%%%%%%%%
% restate theorem etc. this chunk has to be loaded after theorem
% https://tex.stackexchange.com/questions/298866/restate-theorem-numbering-issue
\usepackage{amsthm,thmtools,thm-restate}

%%%%%%%%%%%%%%%%%%%%%%%%%%%%%%%%%%%%%%%%%%%%%%%% 
\newtheorem{algorithm}[theorem]{Algorithm} % all defn, prop, thms share numbering
\newtheorem{axiom}[theorem]{Axiom}
\newtheorem{case}[theorem]{Case}
\newtheorem{condition}[theorem]{Condition}
\newtheorem{conjecture}[theorem]{Conjecture}
\newtheorem{corollary}[theorem]{Corollary}
\newtheorem{criterion}[theorem]{Criterion}
\newtheorem{lemma}[theorem]{Lemma}
\newtheorem{proposition}[theorem]{Proposition}
\newtheorem{solution}[theorem]{Solution}
% different group of styles for defn, ex, prob, etc. 
\theoremstyle{definition}
\newtheorem{definition}[theorem]{Definition}
\newtheorem{example}[theorem]{Example}
\newtheorem{exercise}[theorem]{Exercise}
\newtheorem{problem}[theorem]{Problem}
% another different group of styles for rmk, notation, claim, summary, conclusion, etc.
\theoremstyle{remark}
\newtheorem{remark}[theorem]{Remark}
\newtheorem{notation}[theorem]{Notation}
\newtheorem{claim}[theorem]{Claim}
\newtheorem{summary}[theorem]{Summary}
\newtheorem{conclusion}[theorem]{Conclusion}
\newtheorem{acknowledgement}[theorem]{Acknowledgement}

% conflict with numbering scheme used in line 23; comment all out
% \numberwithin{theorem}{chapter}
% \numberwithin{proposition}{chapter}
% \numberwithin{corollary}{chapter}
% \numberwithin{conjecture}{chapter}
% \numberwithin{notation}{chapter}
% \numberwithin{definition}{chapter}
% \numberwithin{example}{chapter}
% \numberwithin{problem}{chapter}
% \numberwithin{claim}{chapter}
% \numberwithin{lemma}{chapter}
% \numberwithin{remark}{chapter}

\numberwithin{equation}{chapter}
\numberwithin{figure}{chapter}

% change Chapter to Lecture
\renewcommand*{\chaptername}{Lecture}

% flushright "(by SOME FACT)" in proof. See https://tex.stackexchange.com/questions/83509/hfill-in-math-mode
\makeatletter
\newcommand*{\btfact}[1]{\ifmeasuring@#1\else\omit\hfill$\displaystyle#1$\fi\ignorespaces}

% does not imply
\newcommand*{\notimply}{\mathrel{\rlap{\hskip .5em/}}\Longrightarrow}

% big \mid
\newcommand*{\bigmid}{\mathrel{\bigg|}}
% various fields
\newcommand{\Q}{\mathbb{Q}}
\newcommand{\Z}{\mathbb{Z}}
\newcommand{\R}{\mathbb{R}}
\newcommand{\C}{\mathbb{C}}
\newcommand{\F}{\mathbb{F}}
% a set in script font
\newcommand{\sS}{\mathcal{S}}
\newcommand{\sA}{\mathcal{A}}
\newcommand{\sB}{\mathcal{B}}
\newcommand{\sC}{\mathcal{C}}
% order of a mod n
\newcommand*{\ord}{\text{ord}}

\begin{document}
\frontmatter
\title[ENT40Days]{Elementary Number Theory in 40 Days \\
with Prof. Allen}
\author[Y\"un Han]{Y\"un Han}
\address{} %
% \address[Y\"un Han]{Department of ECE, University of Illinois \\
% Urbana, IL 61801}
\curraddr{}
% \curraddr[Y\"un Han]{Control Systems Lab, Department of ECE \\
% University of Illinois, Urbana, IL 61801}
\email{}
% \email[Y\"un Han]{han82@illinois.edu}
\urladdr{}
% \urladdr{http://www.chipnotized.org}
\thanks{Thanks to Prof. Patrick Allen for his unsurpassed inspirational teaching}
\author{} % leave empty
\subjclass{}
%\subjclass[2000]{Primary 11A05, 11A07, 11A15, 11A25; Secondary 11A41, 11A51}
\dedicatory{}
% \dedicatory{Dedicated to the Copyleft Movement}

% \begin{abstract}
% This \AmS-\LaTeX~ version of lecture notes is based on lectures of Math 453 at University of Illinois taught by Prof. Patrick Allen during the term Spring 2017. 
% \end{abstract}

\maketitle
\tableofcontents

\chapter*{Preface}

This book is based on the notes I took while sitting in Prof. Patrick Allen's lectures of Math 453 in Spring 2017. Each lecture was around 50 minutes and the term started on 17 January and ended on 3 May. 

The format of the course included lectures on Monday, Wednesday and Friday every week; one office hours each Monday; one homework session each Wednesday; two midterms (on 24 February and 31 March each) and one final (on 5 May). Weekly homework and reflection was assigned every Friday.

On top of the lecture notes, I also added reading notes that were not in the original lectures. Most included exercise problems are from \emph{100 Problems in Elementary Number Theory} by Ko Chao~\cite{Ko1980}.

This book can be used as a companion to \emph{Elementary Number Theory} by James Strayer~\cite{Strayer2001} or a self-study guide.


\vspace{10mm}
\noindent \emph{Champaign, Illinois, 2017} \hfill {\sc Y\"un Han}

\mainmatter

\part{January to February, 2017}\label{part1} % part one
\chapter[Lecture One]{Day One \hfill {\footnotesize \rm --- 18.01.2017}}

\footnote{Text of Math 453: \emph{Elementary Number Theory} by J. Strayer~\cite{Strayer2001}.}Recall the set of integers

\begin{align}
\mathbb{Z} = \{ \cdots, -2, -1, 0, 1, 2, \cdots \}.
\end{align}
We ask the following questions:
\begin{itemize}
\item Does 2 divide 6 in $\mathbb{Z}$? (Yes.)
\item Does 6 divide 2 in $\mathbb{Z}$? (No.)
\item Does 0 divide 0 in $\mathbb{Z}$? (Yes.)
\end{itemize}

\begin{definition}
Let $a, b$ be integers. We say that $b$ \textbf{divides} $a$, written as $b \mid a$ if there is an integer $c$ such that $a = bc$. And $b$ is called a \textbf{divisor} of $a$; otherwise $b \nmid a$.
\end{definition}

For the above preceding questions, 
\begin{itemize}
\item $2 \mid 6$ since $6 = 3 \times 2$. Both 3 and 2 are integers.
\item $6 \nmid 2$ since  there is no $c \in \mathbb{Z}$ such that $2 = 6c$.
\item $0 \mid 0$ since for $\forall~ c \in \mathbb{Z}$, we have $0 = 0c$.
\end{itemize}

\begin{proposition}\label{prop:div_trans}
Let $a, b, c \in \mathbb{Z}$. If $a \mid b$ and $b \mid c$ then $a \mid c$.
\end{proposition}
\begin{proof}
Since $a \mid b$, we have $\exists~ d \in \mathbb{Z}$ such that $b = ad$. Similarly, $\exists~ e \in \mathbb{Z}$ such that $c = be$. Then $c = (ad)e = a(de)$ and $de$ is an integer. By the definition of divisibility, $a \mid c$.
\end{proof}

\begin{proposition}\label{prop:div_linear_combi}
Let $a, b, c \in \mathbb{Z}$. If $c \mid a$ and $c \mid b$, then for any $m, n \in \mathbb{Z}$, we have $c \mid (am + bn)$.
\end{proposition}
\begin{proof}
  $c \mid a $ implies $\exists~ d \in \mathbb{Z}$ such that $a = cd$. $c \mid b $ implies $\exists~ e \in \mathbb{Z}$ such that $b = ce$. For any $m, n \in \mathbb{Z}$,
  \begin{align}
    am + bn &= (cd)m + (ce)n \\
            &= c(dm) + c(en) \\
            &= c(dm+en),
  \end{align}
  where $dm+en$ is a linear combination of $d, e$ with integer coefficients $m, n$. (i.e., $dm+en \in \mathbb{Z}$.) Therefore $c \mid (am + bn)$.
\end{proof}

\begin{theorem}[Division algorithm]\label{thm:div_alg} % add theorem name in brackets []
Let $a, b \in \mathbb{Z}$, $b > 0$. There are \textbf{unique} integers $q, r \in \mathbb{Z}$, such that $a = qb +r $ with $0 \leqslant r < b$.
\end{theorem}
\begin{proof}
To prove the \underline{existence} of such $q$ and $r$, consider the rational number $\frac{a}b$. (Our assumption $b > 0$ implies $b \neq 0$.) Let $q$ be the largest integer less than $\frac{a}b$, denoted as $q = \lfloor \frac{a}b \rfloor$. 

Let $r = a - qb$. The range of $r$ is $0 \leqslant r < b$. 
\begin{itemize}
\item $r = a - qb \geqslant 0$. Indeed, since $q \leqslant \frac{a}b$ and $b > 0$, we have $qb \leqslant a$, i.e., $ a - qb \geqslant 0$. 
\item $r < b$, otherwise if $r \geqslant b$, $r = a - qb \geqslant b$. Further it implies $a \geqslant (q+1)b$, so $\frac{a}b \geqslant q+1$, which contradicts the fact that $q$ is the largest integer such that $q \leqslant \frac{a}b$. (Now $q + 1$ is an integer such that $q < q+1 \leqslant \frac{a}b$ if $r \geqslant b$.)
\end{itemize}

Hence the proof of existence.

To prove the \underline{uni}q\underline{ueness} of such $q$ and $r$, let $q', r' \in \mathbb{Z}$, such that $a = q'b + r'$ with $0 \leqslant r' <b$. 

We want to show that $q' = q$, and $r' = r$. Since $r' \geqslant 0$, we have $0 \leqslant r' = a - q'b \implies a \geqslant q'b$. So $\frac{a}b \geqslant q'$ by $b > 0$. Since $r' < b$, we have $a = q'b + r < q'b + b \implies a < (q'+1)b$. It further implies $\frac{a}b < q' + 1$. Therefore, we showed $q'$ is also the largest integer bounded by $\frac{a}b$. Hence $q' =\lfloor \frac{a}b \rfloor = q$.

$r' = r$ follows from the fact that $r' = a - q'b = a - qb = r$.
\end{proof}
\subsection*{Exercise}
\begin{enumerate}
% ko80, no. 3
\item Let $m, n \in \Z$ with $m, n > 0$. Show that
  \[
    \frac{1}m + \frac{1}{m+1} + \cdots + \frac{1}{m+n}
  \]
  is not an integer.
% ko80, no. 7(1)
\item Let $\alpha \in \Q$. Let $b$ be the smallest positive integer such that $b \alpha$ is an integer. If there is a $c \in \Z$ such that $c \alpha$ is an integer, then $b \mid c$.
% ko80, no. 10
\item Let $k, n \in \Z$, $n > 1$. Let $m = 2^{n - 1}(2^n - 1)$. Show that any $1 \leqslant k \leqslant m$ can be written as $k = \sum_{d \, \mid \, m} \ell d$, where $\ell = 0 \text{ or } 1$, $d$ is a divisor of $m$.
% ko80, no. 23  
\item Let $a, b \in \Z$ with $a > 0$, $b > 2$. Show $2^b - 1 \nmid a^a + 1$.
\end{enumerate}

\chapter[Lecture Two]{Day Two \hfill {\footnotesize \rm --- 20.01.2017}}

\begin{theorem}[Mathematical induction]\label{thm:math_induct}
Let $m \in \mathbb{Z}$ and $\mathcal{S} \subseteq \mathbb{Z}$ satisfying 
\begin{itemize}
\item $m \in \mathcal{S}$;
\item If $k \geqslant m$ and $k \in \mathcal{S}$ then $k + 1 \in \mathcal{S}$.
\end{itemize} 
Then all integers $n \geqslant m$ are in $\mathcal{S}$.
\end{theorem}
\begin{remark}
Theorem~\ref{thm:math_induct} was stated without proof. The proof uses ``well ordering property''.
\end{remark}
How to use this? Say we have a statement depending on integer $n \geqslant m$ that we want to prove true, then if we show it is 
\begin{itemize}
\item \underline{Base case}: true for $n = m$;
\item \underline{Inductive ste}p: also true for $k + 1$ assuming true for arbitrary $k \geqslant m$.
\end{itemize}
Then the statement is true for all $n \geqslant m$.
\begin{example}[Bernoulli's inequality]
Let $x > -1$ be a non zero real number. Then
\[
(1+x)^n \geqslant 1 + nx 
\]
for all $n \geqslant 2$.
\end{example}
\begin{proof}
  We prove it by induction on $n \geqslant 2$.  

  First take $n = 2$. Then
\begin{align}
\hspace{5cm}(1+x)^2 &= 1 + 2x + x^2 \\
        &\geqslant 1 + 2x. \hspace{3.5cm} & \btfact{\text{(by $x^2 \geqslant0 $)}}
\end{align} 

Assume it is true for $n = k$, $k \geqslant 2$.
\begin{align}
\hspace{4cm}(1+x)^{k+1} &= (1+x)^k (1+x) \\
            &= (1+x)^k + (1+x)^kx \\
            &\geqslant 1+kx + (1+kx)x & \btfact{\text{(by induction hypothesis)}}\\
            &\geqslant 1 + kx + x & \btfact{\text{(by $kx^2 \geqslant 0$)}}\\
            &=1 + (k+1)x.
\end{align}
By mathematical induction in Theorem~\ref{thm:math_induct}, $(1+x)^n \geqslant 1 + nx$ for all $n \geqslant 2$.
\end{proof}
There is also a variant of mathematical induction called \emph{strong induction}: Say we have a statement indexed by integers $n \geqslant m$ that we want to prove true. If we prove
\begin{itemize}
\item \underline{Base case}: the statement is true for $n = m$;
\item \underline{Inductive ste}p: it is also true for $k+1$ assuming it is true for each of $m$, $m+1$, \ldots, $k$ given arbitrary $k \geqslant m$.
\end{itemize}
Then the statement is true for all $n \geqslant m$.

Let us go back to number theory. 
\begin{definition}
Let $p > 1$ be an integer. We say $p$ is \textbf{prime} if its only divisors are 1 and $p$ itself. An integer $n > 1$ is called \textbf{composite} if it is not prime.
\end{definition}

\begin{proposition}\label{prop:int_prod_prime}
Any integer $n > 1$ is a product of primes.
\end{proposition}
\begin{proof}
We prove this by strong induction. When $n = 2$, $n$ is prime, and it is a product of one prime 2 itself. Fix some $k \geqslant 2$ and assume any integer $2 \leqslant n \leqslant k$ is a product of primes. We want to show that $k+1$ is also a product of primes.

If $k+1$ is prime, then we are done; if $k+1$ is not prime, then it is composite. Then $k+1$ can be written as 
\[
k + 1 = ab,
\]
where $1 < a < k+1$ (or $2 \leqslant a \leqslant k$) and $1 < b < k+1$ (or $2 \leqslant b \leqslant k$). By induction hypothesis, both $a$ and $b$ are a product of primes. Then $k+1 = ab$ is a product of primes. Hence the proposition.
\end{proof}

\begin{theorem}[Fundamental theorem of arithmetic]
Any integer $n \geqslant 2$ can be written as $n = p_1 p_2 \cdots p_k$ with $p_1$, $p_2$, \ldots, $p_k$ primes.\footnote{Not necessarily distinct.} This expression is \textbf{unique} up to an ordering.
\end{theorem}
\begin{proof}
The fundamental theorem of arithmetic (FTA) will be proved in Lecture~\ref{chp:fta}.
\end{proof}

\begin{theorem}[Euclid]
There are infinitely many primes.
\end{theorem}
\begin{proof}
Assume the contrary, there are finitely many primes. Let $p_1$, $p_2$, \ldots, $p_k$ be all the primes, $k \in \mathbb{Z}$, $k > 0$. Consider the number $N = p_1p_2 \cdots p_k + 1= \left(\prod_{i=1}^k p_i\right) + 1$. By Proposition~\ref{prop:int_prod_prime}, for $N > 1$, $N$ is a product of primes, i.e., there is a prime $p$ such that $p \mid N$. But $p = p_i$ for some $1 \leqslant i \leqslant k$. Now we have $p_i \mid N$ and $p_i \mid \prod_{i=1}^k p_i$. By Proposition~\ref{prop:div_linear_combi} from last lecture, we have 
\begin{align}
p_i \mid 1 \cdot N + (-1) \cdot \prod_{i=1}^k p_i &= \left(\prod_{i=1}^k p_i\right) + 1 - \prod_{i=1}^k p_i \\
                                                  &= 1.
\end{align}
So $p_i \mid 1$ for some prime $p_i$, a contradiction.
\end{proof}

\begin{proposition}\label{prop:p_le_sqrtn}
Let $n > 1$ be composite. Then there is a prime divisor $p$ of $n$ with $p \leqslant \sqrt{n}$.
\end{proposition}
\begin{proof}
  Since $n$ is composite, we can write $n = ab$ for some $a, b \in \mathbb{Z}$, $1 < a < n$ and $1 < b < n$. Without loss of generality, assume $a \leqslant b$. Then $a \leqslant \sqrt n$. (Otherwise $ab > a \cdot a > \sqrt n \cdot \sqrt n = n$.) 

Since $a > 1$, $a$ has some prime divisor $p$ with $p \leqslant a$ by Proposition~\ref{prop:int_prod_prime}. $p \mid a$ and $a \mid n$ imply $p \mid n$ by Proposition~\ref{prop:div_trans}. $p \leqslant a \leqslant \sqrt n$. Therefore $p$ is the prime divisor of $n$ we are looking for. Hence such prime divisor of $n$ exists.
\end{proof}

\subsection*{Exercises}
\begin{enumerate}
\item Let $n \in \Z$ and $n > 1$. Show that $n^5 + n^4 + 1$ is composite.
\item Find all $n \in \Z$, $n > 0$ such that $3^{2n+1} - 4^{n+1} + 6^n$ is prime. (\emph{Hint}: $3x^2 + xy - 4y^2 = (3x+4y) (x-y)$.)
% ko80, no. 7(2)
\item Let $p$ be a prime. Let $a \in \Z$ with $p \nmid a$. If $b$ is the smallest positive integer such that $\frac{ba}p$ is an integer, then $b = p$.
% ko80, no. 63
\item Let $a_1 = a_2 = a_3 = 1$, $\displaystyle a_{n+1} = \frac{1 + a_n a_{n-1}}{a_{n-2}}$ for $n \geqslant 3$. Show that $a_i$ is an integer for all $i \in \Z$, $i > 0$.
\end{enumerate}

\chapter[Lecture Three]{Day Three \hfill {\footnotesize \rm --- 23.01.2017}}

The immediate consequence of Proposition~\ref{prop:p_le_sqrtn} in the last lecture is that: if an integer $n \geqslant 2$ has no divisors $d$ for all $1 < d \leqslant \sqrt n$, then $n$ is prime. This leads us to the \emph{Sieve of Eratosthenes}, to find all primes bounded by a given integer $n$.

\underline{Al}g\underline{orithm} (Sieve of Eratosthenes): 
\begin{itemize} 
\item Write down all integers $i$, $2 \leqslant i \leqslant n$; 
\item Start at 2: leave 2 but cross out all multiples of 2;
\item Repeat, leaving the next non crossed out integer and cross out all its multiples;
\item When the non crossed out integer is greater than $\sqrt n$, stop the process;
\item The resulting table of non crossed out integers are exactly the primes bounded by $n$.
\end{itemize}

How do primes behave?

Look at the sequence of primes,
\[
2, 3, 5, 7, 11, 13, 17, 19, 23, 29, 31, \ldots  
\]
 note the occurrences of primes that differ by 2, i.e., 
\[
\{3, 5\}, \{5, 7\}, \{11, 13\}, \{17, 19\}, \{29, 31\}, \ldots 
\]
these are called \emph{twin primes}.
\begin{conjecture}[Twin prime conjecture]
There are infinitely many twin primes.
\end{conjecture}
For an integer $n \geqslant 1$, let $p_n$ be the $n$th prime, e.g., $p_1 = 2$, $p_2 = 3$, $p_3 = 5$, $p_4 = 7$. Then the twin prime conjecture can be formulated as 
\[
p_{n+1} - p_n = 2 
\]
for infinitely many $n$.
\begin{theorem}[Zhang\footnote{Zhang's proof was regarded a Cinderella story. He submitted his proof to \emph{Ann. Math.} in 2013 while he had been a lecturer in an unknown college for years without a secure academic appointment. He also had had many odd jobs over the years including working part time in a Subway's sandwich shop and sleeping rough from time to time since he obtained his doctorate in maths more than twenty years before he made his name in the mathematics community. He was thought to begin his college education at the age of 25 and doctoral studies not until he almost turned 30. He has a truly unusual and uneven career path of a mathematician.}, 2013]
There is a constant $C$ such that 
\[
p_{n+1} - p_n \leqslant C
\]
for infinitely many $n$. We can take $C = 70,000,000$. 
\end{theorem}
The above bound can be improved to 246. Philosophically, the difficulty of the problem is primes are the atoms of integers with respect to \emph{multiplication} (i.e., thinking of the fundamental theorem of arithmetic). But the twin prime conjecture involves \emph{additive} information (i.e., the $+2$ part) as well. In general, no relation between the divisors of $n$ and divisors of $n+2$ can be said.

\begin{conjecture}[Goldbach's conjecture, 1742]
Every even integer $n > 2$ can be written as the sum of two primes. 
\end{conjecture}

The best result so for as far as Goldbach's conjecture is concerned is from J. Chen.
\begin{theorem}[Chen, 1973]
There is a constant $m$ such that every even integer $n \geqslant m$ can be written either as a sum of two primes or as a sum of a prime and a number that is the product of two primes.
\end{theorem}

\begin{theorem}[Montgomery--Vaughan, 1975]
  The set of positive integers for which the Goldbach's conjecture fails has density zero, i.e., Goldbach's conjecture holds most of the time.
\end{theorem}
\begin{remark}
Montgomery--Vaughan does not imply the integers which fail the Goldbach's conjecture form a \underline{finite} set.
\end{remark}
\begin{conjecture}[Weak Goldbach's conjecture]
Every odd integer $n \geqslant 7$ is the sum of three primes.
\end{conjecture}
\begin{remark}
Goldbach's conjecture implies the weak Goldbach's conjecture. For odd $n \geqslant 7$, $n - 3$ is even and $n - 3 \geqslant 4 \geqslant 2$. Hence the word ``weak''.
\end{remark}
\begin{theorem}[Vinogradov, 1937]
There is a constant $m$ such that any odd integer $n \geqslant m$ is the sum of three primes.
\end{theorem}
\begin{theorem}[Helfgott, 2013]
The weak Goldbach's conjecture is true.
\end{theorem}
The consequence of the weak Goldbach's conjecture is that: every even integer $n \geqslant 4$ is the sum of at most four primes. 
\begin{remark}
All the theorems listed in this lecture are beyond the scope of this course and their proofs require heavy machinery from Analytic Number Theory (e.g., at least at the level of Math 531).
\end{remark}

We notice that primes seem to get more spaced out on average.
\begin{proposition}
For any integer $n \geqslant 1$, we can find $n$ consecutive composite numbers.
\end{proposition}
\begin{proof}
Consider the string of $n$ consecutive integers,
\[
(n+1)!+2, ~ (n+1)!+3, ~\ldots, ~ (n+1)!+(n+1).
\]
Since $(n+1)! = 1 \cdot 2 \cdot\, \cdots\, \cdot (n+1)$, we have $k \mid (n+1)!$ for $k = 2, 3, \ldots, n+1$. Further $k \mid (n+1)! + k$ by Proposition~\ref{prop:div_linear_combi} for $k = 2, 3, \ldots, n+1$.

Notice $ 1 < k < (n+1)! + k $ by\footnote{Or by $(n+1)!>0$, then $k + (n+1)! > k$.} $k \leqslant n+1 \leqslant (n+1)! < (n+1)! + k$, it implies each of $(n+1)! + k$'s is composite.
\end{proof}

\subsection*{Exercise}
\begin{enumerate}
% ko80, no. 46
\item Let $n \in \Z$, $n \geqslant 2$. Show that there exists an arithmetic progression of $n$ pairwise coprime composite numbers. (\emph{Hint}: Let $p > n$ be an odd prime and let $N$ be an integer such that $N \geqslant p - (n-1)n!$. Consider arithmetic progression $N!+p, ~ N!+p+n!, ~\ldots, ~ N!+p+(n-1)n!$.)
\end{enumerate}

\chapter[Lecture Four]{Day Four \hfill {\footnotesize \rm --- 25.01.2017}}\label{chp:prime_num_theorem_gcd}

A notation from analytic number theory. If $f$ and $g$ are two real valued functions on an interval $(a, \infty) \subseteq \mathbb{R}$, $a \in \mathbb{R}$, we say $f$ and $g$ are \textbf{asymptotic} if 
\[
\lim_{x \to \infty} \frac{f(x)}{g(x)} = 1,
\]
and write $f \sim g$. Heuristically, this means $f = g + \text{error term}$ or in the little-$o$ notation,
\[
f = g + o (g).
\]
Let $\pi (x)$ be the number of primes that are bounded by $x$. If we define a set $\mathcal{P}$
\[
\mathcal{P} := \left\{ 1 \leqslant p \leqslant x : \text{$p$ is prime} \right\},
\]
then
\begin{align}
\pi (x) &= \mathbf{card}(\mathcal{P}) \\
        &= \# \left\{ 1 \leqslant p \leqslant x : \text{$p$ is prime} \right\} \\
        &= | \left\{ 1 \leqslant p \leqslant x : \text{$p$ is prime} \right\} |.
\end{align}
$\mathbf{card}(\rm{set})$, $\#\rm{set}$, and $|\rm{set}|$ all denote the cardinality or the number of elements, or the size of a set. 
\begin{example}
$\pi(10) = 4$, $\pi(25) = 9$.
\end{example}
\begin{theorem}[Prime number theorem] 
Let $\pi(x)$ be the number of primes that are less than or equal to $x$. We have
\[
\pi(x) \sim \frac{x}{\log x}.
\]
\end{theorem}
\begin{remark}
The logarithm $\log x$ in the statement is the natural logarithm. The proof is an application of complex analysis and will not be given in this course. However there is an elementary proof given by A. Selberg in 1949~\cite{Selberg1949} but it is winding and difficult.
\end{remark}

Let us step back to our discussion of fundamental theorem of arithmetic. 
\begin{definition}
Let $a, b \in \mathbb{Z}$, not both zero. The \textbf{greatest common divisor} of $a$ and $b$ is the largest divisor $d$ such that $d \mid a$ and $d \mid b$. We write $d = \gcd(a,b)$. If $\gcd(a,b) = 1$, we say $a$ and $b$ are \textbf{coprime} or relatively prime.
\end{definition}

\begin{example}\label{ex:ex2_day4}
Quick examples.
\begin{itemize}
\item $\gcd(-12, 15) = 3$.
\item Let $a \in \mathbb{Z}$, $a > 0$, then $\gcd(a, 0) = a$.
\item Let $a \in \mathbb{Z}$, then $\gcd(a, 1) = 1$.
\item Let $a \in \mathbb{Z}$ and $p$ prime, then 
\[
\gcd(a, p) = \left\{ \begin{array}{rl}
               1 & \text{if $p \nmid a$,} \\[1mm]
               p & \text{if $p \mid a$.} 
\end{array} \right.
\]
\end{itemize}
\end{example}

\begin{example}\label{ex:takeout_gcd_coprime}
Let $a, b \in \mathbb{Z}$, not both zero and let $d = \gcd(a, b)$. Note that $\frac{a}d$ and $\frac{b}d$ are integers, not both zero either. Show $\frac{a}d$ and $\frac{b}d$ are coprime.
\end{example}
\begin{proof}
We need to show $\gcd (\frac{a}d, \frac{b}d) = 1$. Assume otherwise, we can write 
\begin{align}
\frac{a}d = me, \,
\frac{b}d = ne 
\end{align}
for some $m, n, e \in \mathbb{Z}$ with $e > 1$. Then we have $a = m(de)$ and $b = n(de)$ with $de>0$ a positive divisor of both $a$ and $b$. Further $de > d \cdot 1 > d$, which contradicts the fact that $d$ is the greatest common divisor of $a$ and $b$. Hence we completed the proof. 
\end{proof}

\begin{proposition}\label{prop:gcd_min_lin}
Let $a, b \in \mathbb{Z}$, not both zero, then 
\[
\gcd(a, b) = \min \left\{ ax + by : ax + by >0, \forall~ x, y \in \mathbb{Z} \right\}.
\]
\end{proposition}
\begin{proof}
Let $\mathcal{S} = \left\{ ax + by : ax + by >0, \forall~ x, y \in \mathbb{Z} \right\}$. First we notice that $\mathcal{S}$ is nonempty. Indeed, $a$ and $b$ are not both zero, at least one of $\pm a$, $\pm b$ is positive, i.e., at least one of them belongs to $\mathcal{S}$. By well ordering property of a nonempty set $\mathcal{S}$, it has a minimal element. Call this minimum $d$. Since $d \in \mathcal{S}$, there exist $m, n \in \mathbb{Z}$ such that $d = am + bn$.

Let us show $d \mid a$ and $d \mid b$. By division algorithm (See Theorem~\ref{thm:div_alg}), there exist unique $q, r \in \mathbb{Z}$ with $0 \leqslant r < d$ such that 
\[
a = dq + r.
\] 
Notice 
\begin{align}
r &= a - dq \\
  &= a - (am + bn)q \\
  &= a(1-mq) + b(-nq). 
\end{align}
If $r > 0$, then $r \in \mathcal{S}$. But $r < d$ which contradicts the fact $d$ is the minimum of $\mathcal{S}$. So $r = 0$. Hence $d \mid a$. Similar argument can be said about $d \mid b$. We proved $d$ is a common divisor of $a$ and $b$.

Next we show $d$ is the greatest common divisor. Let $c \in \mathbb{Z}$, $c > 0$ be any common divisor of $a$ and $b$. We need to show $c \leqslant d$. Indeed, $c \mid a$ and $c \mid b$ imply $c \mid am + bn = d$. Both $c$ and $d$ are positive and $c$ is a divisor of $d$, i.e., $d = ce$ for some $e \in \mathbb{Z}$, $e > 0$. Hence $c \leqslant d$.
\end{proof}

\begin{proposition}\label{prop:p_div}
Let $a, b, p \in \mathbb{Z}$ with $p$ prime. If $p \mid ab$, then $p \mid a$ or $p \mid b$.
\end{proposition}
\begin{proof}
If $p \mid a$, we are done. We may assume $p \nmid a$ and we want to show $p \mid b$. By Example~\ref{ex:ex2_day4}, $p \nmid a \implies \gcd(a, p) = 1$. By Proposition~\ref{prop:gcd_min_lin}, there are $m, n \in \mathbb{Z}$ such that
\begin{equation}\label{eqn:eq1_day4}
\begin{split}
1 &= \gcd(a, p) \\
  &= am + pn. 
\end{split}
\end{equation}
Multiply Equation~\eqref{eqn:eq1_day4} by $b$ on both sides, we have $b = (ba)m + p(bn)$. $pbn$ is a multiple of $p$, further by our assumption, $p \mid ab$, so $p \mid b$ as desired. 
\end{proof}

\begin{corollary}\label{cor:p_div}
Let $p$ be prime and let $a_1$, $a_2$, \ldots, $a_k \in \mathbb{Z}$ for some $k \geqslant 1$. If $p \mid \prod_{i = 1}^k a_i$, then $p \mid a_i$ for some $1 \leqslant i \leqslant k$.
\end{corollary}
% in the sketch of proof, change the default QED symbol to black square to indicate the proof is not fully "complete" with every detail.
\begin{proof}[Sketch of Proof]\renewcommand*{\qedsymbol}{\ensuremath{\blacksquare}}
We induct on $k$. The rough idea is to group the product $a_1a_2 \cdots a_{k+1}$ as $(a_1a_2 \cdots a_k)a_{k+1}$, so we can treat this product as if the product is the multiplication of two terms. Then apply Proposition~\ref{prop:p_div} which handles the case of product of exactly two terms. Check out~\cite{Strayer2001}*{\S 1.5, Corollary~1.15} for details.
\end{proof}

\begin{example}
Let $p \in \mathbb{Z}$ with $p > 1$. $p$ satisfies the following property,
\[
\forall~ a, b \in \mathbb{Z} \text{ such that } p \mid ab \implies p \mid a \text{ or } p \mid b. 
\]
Show $p$ is prime.
\end{example}
\begin{proof}\footnote{Left as an exercise, proof was not given in the lecture.}
Assume the contrary, $p$ is composite. Then $p = mn$ for some $1 < m < p$, $1 < n < p$, $m, n \in \mathbb{Z}$. By the assumption, whenever $p \mid ab $ we have either $p \mid a$ or $p \mid b$. Swap $a$ for $b$ if necessary, we may assume $p \mid ab \implies p \mid a$. 

Take $a = m$ and $b = n$ in our case. Then $p \mid m \implies m = pc$ for some integer $c$. But $p = mn = pcn$, it follows that $c = n = 1$ while $n > 1$, a contradiction.
\end{proof}
\begin{remark}
This example gives an alternative definition of a prime.
\end{remark}

\subsection*{Exercise}
\begin{enumerate}
\item Let $a, b \in \Z\backslash\{0, 1\}$. Show there exists $c \in \Z\backslash \{0,1\}$  such that $\gcd (c, a) = \gcd (c, b) = 1$. (\emph{Hint}: Assume $a, b > 0$ and take $c = ab + 1$.)
% ko80, no. 1
\item Let $m, n \in \Z$, $m, n > 0$ with $m$ odd. Show $\gcd (2^m - 1, 2^n + 1) = 1$.
% ko80, no. 6
\item Let $a, b \in \Z$ with $\gcd (a, b) = 1$, $a + b \neq 0$. Let $p$ be an odd prime. Show that
  \[
    \gcd \left( a+b, \frac{a^p+b^p}{a+b} \right) = 1 \text{ or } p.
  \]
% IMO 1988, no. 6
\item Let $a,b \in \Z$, $a, b>0$ such that $ab+1 \mid a^2+b^2$. Show $\frac{a^2+b^2}{ab+1}$ is a perfect square.
% ko80, no. 36
\item Let $n \in \Z$, $n > 0$. Compute $\gcd \left( \binom{2n}{1}, \binom{2n}{3}, \ldots, \binom{2n}{2n-1} \right)$.
\end{enumerate}
\chapter[Lecture Five]{Day Five \hfill {\footnotesize \rm --- 27.01.2017}}\label{chp:fta}

\underline{Fundamental theorem of arithmetic} (FTA)
\begin{theorem}[Fundamental theorem of arithmetic]\label{thm:fta}
Any integer $n \geqslant 2$ can be written as a product of primes 
\[
n = p_1p_2 \cdots p_k,
\]
where $k$ is a positive integer, and this expression is unique up to an ordering.
\end{theorem}
\begin{proof}
  We already saw that $n$ can be written as a product of primes. (See Proposition~\ref{prop:int_prod_prime}.) So it remains to show the uniqueness part (up to an ordering).

  Let $p_1, p_2, \ldots, p_k, q_1, q_2, \ldots, q_m$ be primes such that
\begin{equation}\label{eqn:eq1_day5}
    n = p_1 p_2 \cdots p_k = q_1 q_2 \cdots q_m.
\end{equation}
We want to show that $k = m$ and there is a reordering of $q_1, q_2, \ldots, q_m$ such that $p_i = q_i$ for some $1 \leqslant i \leqslant k = m$. From Equation~\eqref{eqn:eq1_day5}, we see
\[
  p_1 \mid q_1 q_2 \cdots q_m, 
\]
while $p_1$ is prime. By Corollary~\ref{cor:p_div}, there is some $1 \leqslant j \leqslant m$ for which $p_1 \mid q_j$. Both $p_1$ and $q_j$ are prime, therefore $p_1 = q_j$.

Relabeling and reordering if necessary, we can assume $j = 1$, i.e., $p_1 = q_1$. Dividing Equation~\eqref{eqn:eq1_day5} by $p_1 = q_1$, we obtain
\[
  p_2 p_3 \cdots p_k = q_2 q_3 \cdots q_m.
\]
The same argument about $p_1 = q_1$ applies to $p_2 = q_2$, using a reordering (if necessary) of $p_2 p_3 \cdots p_k = q_2 q_3 \cdots q_m$.

We can keep this procedure for $p_3, p_4, \ldots, p_k$ and conclude that there is a relabeling of $q_1, q_2, \ldots, q_k$ such that $p_i = q_i$ for all $1 \leqslant i \leqslant k$.

If $m > k$, we divide Equation~\eqref{eqn:eq1_day5} by $p_1 p_2 \cdots p_k = q_1 q_2 \cdots q_k$ to get
\[
  1 = q_{k+1} q_{k+2} \cdots q_m.
\]
So we must have $m \leqslant k$. Combining $k \geqslant m$, we have $m = k$.
\end{proof}

A useful reformulation of Theorem~\ref{thm:fta}:
\begin{theorem}[Prime factorization]\label{thm:fta_factor}
  Any integer $n \geqslant 2$ can be written as
  \[
    n = \prod_{i = 1}^k p_i^{e_i},
  \]
  where $p_i$'s are distinct primes and $e_i$'s are integers $e_i \geqslant 1$ for $1 \leqslant i \leqslant k$, $k \in \mathbb{Z}$ with $k > 0$. This expression is \textbf{unique} up to an ordering. $n = \prod_{i = 1}^k p_i^{e_i}$ is called \textbf{prime factorization}.
\end{theorem}

\begin{proposition}\label{prop:prop1_day5}
  Let $a \geqslant 2$ be an integer and let
  \[
    a =  p_1^{e_1} p_2^{e_2} \cdots p_k^{e_k}
  \]
  be the prime factorization of $a$ with $p_1$, $p_2$, \dots, $p_k$ distinct primes and $e_i \geqslant 1$ integers for $1 \leqslant i \leqslant k$, $k \in \mathbb{Z}$ with $k > 0$. The positive divisors of $a$ are precisely the elements
  \[
    p_1^{d_1} p_2^{d_2} \cdots p_k^{d_k},
  \]
  where $0 \leqslant d_i \leqslant e_i$ for each $1 \leqslant i \leqslant k$. And these divisors $p_1^{d_1} p_2^{d_2} \cdots p_k^{d_k}$ are all distinct.
\end{proposition}
\begin{proof}
  If $b = p_1^{d_1} p_2^{d_2} \cdots p_k^{d_k}$ with $0 \leqslant d_i \leqslant e_i$ for each $1 \leqslant i \leqslant k$, $k \in \mathbb{Z}$ with $k > 0$, then $e_i - d_i \geqslant 0$ for each $1 \leqslant i \leqslant k$, we know 
\[
c = p_1^{e_1 - d_1} p_2^{e_2 - d_2} \cdots p_2^{e_2 - d_2} \in \mathbb{Z}.
\]
Also $bc = p_1^{e_1} p_2^{e_2} \cdots p_k^{e_k}$ implies $b \mid a$.

Now assume $b$ is a positive divisor of $a$.
\begin{itemize}
\item If $b = 1$, then $b = 1 = p_1^0 p_2^0 \cdots p_k^0$, i.e., $d_1 = d_2 = \cdots = d_k = 0$;
\item If $b = a$, then $b = a = p_1^{e_1} p_2^{e_2} \cdots p_k^{e_k}$, i.e., $d_1 = e_1, d_2 = e_2, \ldots, d_k = e_k$.
\end{itemize}

Now we assume $b \neq 1, a$. Write $a = bc$. Since $b > 1$ and $b < a$, $1 < c < a$. Apply fundamental theorem of arithmetic (Theorem~\ref{thm:fta}) to both $b$ and $c$. There are primes $q_1, q_2, \ldots, q_m$ and $q_{m+1}, q_{m+2}, \ldots, q_n$ where $m, n \in \mathbb{Z}$, $m > n> 0$, such that
\begin{align}
b &= q_1 q_2 \cdots q_m, \\
c &= q_{m+1} q_{m+2} \cdots q_n.
\end{align}
Then
\begin{align}
a &= bc \implies \\
p_1^{e_1} p_2^{e_2} \cdots p_k^{e_k} &= q_1 q_2 \cdots q_m \cdot q_{m+1} q_{m+2} \cdots q_n.
\end{align}
The uniqueness of the prime factorization of $a$ guarantees that each $q_j$, $j \in \mathbb{Z}$, $1 \leqslant j \leqslant n$, equals some $p_i$. Moreover, this $q_j$ appears exactly $e_i$ times in 
\[
q_1 q_2 \cdots q_m \cdot q_{m+1} q_{m+2} \cdots q_n.
\]
In particular, for each $1 \leqslant j \leqslant m$, $q_j = p_i$ for some $1 \leqslant i \leqslant k$ and $p_i$ appears \underline{at most} $e_i$ times in $q_1 q_2 \cdots q_m$. Thus $b = p_1^{d_1} p_2^{d_2} \cdots p_k^{d_k}$ with $0 \leqslant d_i \leqslant e_i$ for $1 \leqslant i \leqslant k$.
\end{proof}
\begin{definition}
Let $a, b$ be positive integers. The \textbf{least common multiple} of $a$ and $b$, written as ${\rm lcm} (a, b)$, is the smallest positive integer $m$ such that $a \mid m$, $b \mid m$.
\end{definition}
\begin{restatable}{proposition}{gcdlcm}\label{prop:gcd_lcm} % name of restatable contains only letters a - z, A - Z like in gcdlcm
Let $a, b \in \mathbb{Z}$ with $a, b \geqslant 1$. Let $p_1, p_2, \ldots, p_k$ be primes such that 
\begin{align}
a &=  p_1^{e_1} p_2^{e_2} \cdots p_k^{e_k}, \\
b &=  p_1^{f_1} p_2^{f_2} \cdots p_k^{f_k}
\end{align}
with $e_i, f_i \in \mathbb{Z}$, $e_i, f_i \geqslant 0$ for $1 \leqslant i \leqslant k$, $k$ a positive integer. Then we have 
\begin{align}
\gcd(a, b) &= p_1^{\min \{e_1, f_1\}} p_2^{\min \{e_2, f_2\}} \cdots p_k^{\min \{e_k, f_k\}}, \\
{\rm lcm}(a, b) &= p_1^{\max \{e_1, f_1\}} p_2^{\max \{e_2, f_2\}} \cdots p_k^{\max \{e_k, f_k\}}.
\end{align}
\end{restatable}
\begin{proof}[Sketch of Proof]\renewcommand*{\qedsymbol}{\ensuremath{\blacksquare}}
Left to class as an exercise. Double check~\cite{Strayer2001}*{\S 1.5} for details.
\end{proof}

\subsection*{Exercise}
\begin{enumerate}
% ko80, no. 17
\item Let $n, k \in \Z$, $n \geqslant k > 0$. Consider $k$ integers
  \[
    1 \leqslant a_1 < a_2 < \cdots < a_k \leqslant n.
  \]
  If any two $a_i, a_j$, $i \neq j$, $1 \leqslant i, j \leqslant k$, $i, j \in \Z$, satisfy $\text{lcm} (a_i, a_j) > n$, then
  \[
    \sum_{i = 1}^k \frac{1}{a_i} < \frac{3}2.
  \]
% ko80, no. 20
\item Let $n, k \in \Z$, $n \geqslant k > 0$. Consider $k$ integers
  \[
    1 \leqslant a_1 < a_2 < \cdots < a_k \leqslant n.
  \]
  If any $a_i$, $1 \leqslant i \leqslant k$, $i \in \Z$, satisfies
  \[
    a_i \nmid \prod_{\substack{1 \, \leqslant j \, \leqslant k \\ i \neq j}} a_j,
  \]
  then $k \leqslant \pi (n)$.
\end{enumerate}

\chapter[Lecture Six]{Day Six \hfill {\footnotesize \rm --- 30.01.2017}}

Recall Proposition~\ref{prop:gcd_lcm} of last lecture,
% how to restate a theorem
% https://tex.stackexchange.com/questions/422/how-do-i-repeat-a-theorem-number
% https://tex.stackexchange.com/questions/51286/recalling-a-theorem

% restatable theorem etc.
\gcdlcm*
\begin{proof}[Sketch of Proof]\renewcommand*{\qedsymbol}{\ensuremath{\blacksquare}}
If $a = 1$ (or similarly consider $b = 1$) then 
\begin{align}
\gcd (a, b) &= 1 = \gcd (1, b), \\
{\rm lcm} (a, b) &= b = {\rm lcm} (1, b).
\end{align}
Checking the right hand sides of $\gcd (a, b)$ and ${\rm lcm} (a, b)$ respectively in the proposition is straightforward. 

Now assume $a, b > 1$, we saw last time (in the proof of Proposition~\ref{prop:prop1_day5}) that 
\[
c \mid a \text{ and } c \mid b \iff c = p_1^{s_1} p_2^{s_2} \cdots p_k^{s_k}
\]
with $s_i \leqslant e_i $ and $s_i \leqslant f_i$ for $1 \leqslant i \leqslant k$, $s_i \in \mathbb{Z}, s_i \geqslant0$. This happens if and only if $s_i \leqslant \min \{e_i, f_i\}$ for $1 \leqslant i \leqslant k$. Thus when $s_i = \min \{e_i, f_i\}$ for $1 \leqslant i \leqslant k$ occurs, we have the greatest common divisor of $a$ and $b$. 

The same idea can be used in the case of proving the least common multiple.
\end{proof}
\begin{corollary}
For $a, b \in \mathbb{Z}$ with $a, b > 0$, we have 
\[
\gcd(a, b) \, {\rm lcm}(a, b) = ab.
\]
\end{corollary}
\begin{proof}
 (\emph{Hint}: Use the fact that for two real numbers $x, y$, we have $\min \{x, y\} + \max \{x, y\} = x + y$.)
Take $p_1, p_2, \ldots, p_k$ distinct primes such that 
\begin{align}
a &=  p_1^{e_1} p_2^{e_2} \cdots p_k^{e_k}, \\
b &=  p_1^{f_1} p_2^{f_2} \cdots p_k^{f_k}
\end{align}
with $e_i, f_i \in \mathbb{Z}$, $e_i, f_i \geqslant 0$ for $1 \leqslant i \leqslant k$, $k$ a positive integer. Note that for two real numbers $x, y$, 
\[
\min \{x, y\} + \max \{x, y\} = x + y.
\]
Proposition~\ref{prop:gcd_lcm} implies 
\begin{align}
\gcd(a, b) \, {\rm lcm}(a, b) &= p_1^{ \min \{ e_1 + f_1\} + \max\{ e_1 + f_1 \} } p_2^{ \min \{ e_2 + f_2\} + \max\{ e_2 + f_2 \} } \cdots \\
& \hspace{7.8mm} p_k^{ \min \{ e_k + f_k\} + \max\{ e_k + f_k \} } \\
                              &= p_1^{e_1 + f_1} p_2^{e_2 + f_2} \cdots p_k^{e_k + f_k} \\
                              &= ab.
\end{align}
We completed the proof.
\end{proof}
Proposition~\ref{prop:gcd_lcm} gives a conceptual description of the $\gcd$, but it is not very useful in computing. (Since integer factorization is hard.) 

One practical way\footnote{Other (faster, as used in \texttt{Magma}) ways include Accelerated GCD (See~\cite{Weber1995}) and Lehmer extended GCD (See~\cite{Knuth1997}, pp. 345 -- 348.)} to compute $\gcd(a, b)$ is the \textbf{Euclidean algorithm}.

\begin{lemma}\label{lem:lem1_day6}
Let $a, b \in \mathbb{Z}$, $b \neq 0$. Write $a = bq + r$ with $q, r \in \mathbb{Z}$. Then 
\[
\gcd (a, b) = \gcd (b, r).
\]
\end{lemma}
\begin{proof}
Left as an exercise.\footnote{Hint: Show the set $\{ \text{common divisors of } a, b \} = \{ \text{common divisors of } b, r \}$.}
\end{proof}

\underline{Euclidean al}g\underline{orithm}: Let $a, b \in \mathbb{Z}$ with $a \geqslant b > 0$.
\begin{itemize}
\item Write $a = q_1 b + r_1$ with $q_1, r_1 \in \mathbb{Z}$ and $0 \leqslant r_1 < b$. If $r_1 = 0$, stop; 
\item If $r_1 \neq 0 $, write $b = q_2 r_1  + r_2$ with $q_2, r_2 \in \mathbb{Z}$ and $0 \leqslant r_2 < r_1$. If $r_2 = 0$, stop;
\item If $r_2 \neq 0 $, write $r_1 = q_3 r_2 + r_3$ with $q_3, r_3 \in \mathbb{Z}$ and $0 \leqslant r_3 < r_2$, \ldots ;
\item Continue writing $r_{n-1} = q_{n+1} r_n + r_{n+1}$ with $q_{n+1}, r_{n+1} \in \mathbb{Z}$ and $0 \leqslant r_{n+1} < r_n$ until $r_{n+1} = 0$.
\end{itemize}

\begin{theorem}\label{thm:euclid_gcd}
For $a, b \in \mathbb{Z}$ with $a \geqslant b > 0$, the Euclidean algorithm terminates in finite time. There exists $n \geqslant 1$ such that $r_{n+1} = 0$. Moreover, if $r_{n+1} = 0$ and $r_n \neq 0$, then $\gcd (a, b) = r_n$.
\end{theorem}
\begin{proof}
Assume otherwise, if Euclidean algorithm does not terminate in finite time, then 
\[
b > r_1 > r_2 > \cdots > r_n > \cdots 
\]
is a strictly decreasing infinite sequence of positive integers. Contradiction. 

Therefore there exists $n \geqslant 1$ such that $r_{n+1} = 0$ and $r_n \neq 0$.

Repeatedly apply Lemma~\ref{lem:lem1_day6}, we have 
\begin{align}
\hspace{4.4cm} \gcd (a, b) &= \gcd (b, r_1)  \\
            &= \gcd (r_1, r_2) \\
            &= \cdots \\
            &= \gcd (r_n, r_{n+1}) \\
            &= \gcd (r_n, 0) \hspace{2.2cm} &  \btfact{(\text{by } r_{n+1} = 0)}\\
            &= r_n > 0.
\end{align}
Hence we completed the proof.
\end{proof}

\begin{example}\label{ex:gcd}
Compute $\gcd (308, 119)$.
\begin{proof}[Solution]
  We can write
  \begin{align}
    \hspace{5.4cm}308 &= 2 \times 119 + 70 \\
    119 &= 1 \times 70 + 49 \\
    70  &= 1 \times 49 + 21 \\
    49  &= 2 \times 21 + \underline{7} & \btfact{\leftarrow \text{last nonzero remainder}}  \\
    21  &= 3 \times 7 + 0 
  \end{align}
  By Theorem~\ref{thm:euclid_gcd}, we have $\gcd (308, 119) = 7$.
\end{proof}
\end{example}

Recall by Proposition~\ref{prop:gcd_min_lin}, there are integers $x, y \in \mathbb{Z}$ such that
\[
  \gcd (a, b) = ax + by.
\]
We can use Euclidean algorithm to get such $x, y$ by back substitution.
\begin{align}
  \gcd (a, b) &= r_n \\
              &= r_{n-2} - q_n r_{n-1} .
\end{align}
Plug this into
\[
\hspace{9.2mm}  r_{n-1} = r_{n-3} - q_{n-1} r_{n-2} ,
\]
then continue with
\[
\hspace{8.2mm}  r_{n-2} = r_{n-4} - q_{n-2} r_{n-3} 
\]
etc., until we reach the first of this series of equations $a = q_1 b + r_1$. The we have a linear combination in terms of $a$ and $b$.

\begin{example}\label{ex:gcd2}
  Follow up Example~\ref{ex:gcd}, write 7 as a linear combination of 308 and 119.
\end{example}
\begin{proof}[Solution]
  We know from Example~\ref{ex:gcd} that $7 = \gcd (308, 119)$. By back substitution,
  \begin{align}
    7 &= 49 - 2 \times 21 \\
      &= 49 - 2 \times (70 - 1 \times 49) \\
      &= 49 \times 3 - 2 \times 70 \\
      &= (119 - 1 \times 70) \times 3 - 2 \times 70 \\
      &= 119 \times 3 - 70 \times 5 \\
      &= 119 \times 3 - (308 - 2 \times 119) \times 5 \\
      &= 308 \times (-5) + 119 \times 13. 
  \end{align}
\end{proof}
\subsection*{Exercise}
\begin{enumerate}
\item Show that $x^3$ and $x^3 + x + 1$ are coprime. (\emph{Hint}: Note that $\gcd (x^3, x^3 + x + 1) = \gcd (x^3, x + 1)$.)
% ko80, no. 35
\item Let $n \in \Z$, $n > 0$. Show there exists a unique pair $k, \ell \in \Z$ with $0 \leqslant \ell < k$ such that
  \[
    n = \frac{k(k - 1)}2 + \ell.
  \]
  
\end{enumerate}
\chapter[Lecture Seven]{Day Seven \hfill {\footnotesize \rm --- 01.02.2017}}\label{chp:diophantine_two_var}

Recall Euclid algorithm from last lecture: the algorithm allows us to compute $\gcd(a, b)$, $a, b \in \mathbb{Z}$ and find $x, y \in \mathbb{Z}$ such that $ax + by = \gcd (a, b)$. Hence it is natural to ask

Q\underline{uestion}: Given $a, b, c \in \mathbb{Z}$ with $a, b > 0$. Consider the equation
\[
  ax + by = c.
\]
When can we solve it for $x, y \in \mathbb{Z}$? If this is the case, can we find all solutions?

This is an example of a linear Diophantine equation (in two variables).

\begin{proposition}\label{prop:diophantine_two_var}
  Let $a, b, c \in \mathbb{Z}$ with $a, b$ not both zero. Then there are $x, y \in \mathbb{Z}$ such that $ax + by = c$ if and only if  $\gcd(a, b) \mid c$.
\end{proposition}
\begin{proof}
Assume there are $x, y \in \mathbb{Z}$ such that 
\[
ax + by = c.
\]  

Let $d = \gcd (a, b)$. Since $d \mid a$, $d \mid b$, we have $d \mid (ax + by)$ by Proposition~\ref{prop:div_linear_combi}, i.e., $d = \gcd(a, b) \mid c$.

Now assume $d \mid c$, we want to show there are $x, y \in \mathbb{Z}$ such that 
\[
ax + by = c.
\]
We proved previously (in Proposition~\ref{prop:gcd_min_lin}) that there are $m, n \in \Z$ such that 
\[
am + bn = d.
\]
Then there is an integer $e \in \Z$ such that $c = de$ by $d \mid c$, i.e., 
\[
a(me) + b(ne) = de = c.
\]
So we can take $x = me$ and $y = ne$ as desired.
\end{proof}

\begin{lemma}\label{lem:lem1_day7}
Let $a, b, c \in \Z$, each nonzero. If $a \mid bc$ and $\gcd (a, b) = 1$, then $a \mid c$.
\end{lemma}
\begin{proof}
Since $\gcd (a, b) = 1$, we have there exist $m, n \in \Z$ such that
\begin{align}
1 &= am + bn.
\end{align}
Multiply both sides by $c$,
\begin{align}
c = m(ac) + n(bc).
\end{align}
But $a \mid ac$ trivially and $a \mid bc$ by our assumption. So we have $a \mid c$.
\end{proof}
\begin{remark}
Alternatively, we can prove Lemma~\ref{lem:lem1_day7} using prime factorization and fundamental theorem of arithmetic. Since $\gcd (a, b) = 1$, $a, b$ share no common divisors. Therefore we must have $a \mid c$ if $a \mid bc$.

It also looks similar to the setting when we proved that if $p$ is prime, $p \mid bc$ and $p \nmid b$ then $p \mid c$. Actually the same proof works here too. 
\end{remark}

\begin{proposition}\label{prop:lin_diophantine}
Let $a, b \in \Z$, not both zero. Let $c \in \Z$. Assume we have $x_0, y_0 \in \Z$ such that $ax_0 + by_0 = c$. Then there exist $x, y \in \Z$ such that $ax + by = c$ if and only if 
\[
x = x_0 + \frac{b}{\gcd (a, b)}k \text{ and } y = y_0 - \frac{a}{\gcd (a, b)}k
\] 
for $k \in \Z$.
\end{proposition}
\begin{proof}
Let $d = \gcd (a, b)$. Let $a = dr$ and $b = ds$ for some $r, s \in \Z$.

Let $x = x_0 + sk$, $y = y_0 - rk$ for some $k \in \Z$. Then 
\begin{align}
ax + by &= a(x_0 + sk) + b(y_0 - rk) \\
        &= ax_0 + by_0 + \cancelto{0}{(ask\footnotemark - brk)} \\
        & = c.
\end{align}
\footnotetext{Not the word ``ask''.}
Now assume we have $x, y \in \Z$ such that $ax + by = c$. Substituting $ax_0 + by_0 = c$, we have 
\begin{align}
  a(x - x_0) + b(y - y_0) &= 0, \\
  dr(x - x_0) + ds(y - y_0) &= 0 \implies \\
  r(x - x_0) + s(y - y_0) &= 0.
\end{align}
Since $\gcd (r, s) = 1$, $r \mid s(y - y_0) \implies r \mid (y_0 - y)$ by Lemma~\ref{lem:lem1_day7}. Therefore $y_0 - y = kr$ for some $k \in \Z$. It follows that $y = y_0 - kr$.

Substitute $y = y_0 - rk$ in $ax + by = c$, we have
\[
r(x - x_0) = srk.
\]
So $x = x_0 + sk$.
\end{proof}

\begin{example}
We saw last time in Example~\ref{ex:gcd2}
\[
308 \times (-5) + 119 \times 13 = 7.
\]
By Proposition~\ref{prop:lin_diophantine}, the set of integer solutions to 
\[
308 x + 119 y = 7
\] 
is given by 
\begin{align}
\left\{ \begin{array}{l} 
          x = -5 + \displaystyle \frac{119}7 k \\
          \\ % add more space
          y = 13 - \displaystyle \frac{308}7 k
          \end{array}\right. , \text{ for } k \in \Z.
\end{align}
\end{example}

Q\underline{uestion}: How efficient is Euclidean algorithm? For run time, how many steps does it take?

Let $a, b \in \Z$ with $a \geqslant b > 0$. We saw 
\begin{align}
a &= q_1 b   + r_1, \hspace{6.7mm} 0 < r_1 < b \\
b &= q_2 r_1 + r_2, \hspace{5mm} 0 < r_2 < r_1 \\
& \hspace{2mm} \vdots \\ 
r_{n-2} &= q_n r_{n-1} + r_n, \, 0 < r_n < r_{n-1} \\
r_{n-1} &= q_{n+1} r_n + 0, \hspace{2.5mm} r_{n+1} = 0.
\end{align}
So it takes $(n+1)$ steps. Can we bound $(n+1)$ in terms of $a$ and $b$?

A naive analysis: $r_1 < b$ so 
\begin{align}
r_1 &\leqslant b - 1, \\
r_2 &\leqslant r_1 - 1 \leqslant b - 2, \\
& \hspace{2mm} \vdots \\ 
r_n &\leqslant b - n,\\
0 &= r_{n+1} \leqslant b - (n+1).
\end{align}
Hence $n \leqslant b - 1$. The naive bound $(n+1) \leqslant b$.

\chapter[Lecture Eight]{Day Eight \hfill {\footnotesize \rm --- 03.02.2017}}

Recall from last time, the naive bound of Euclid algorithm is $(n+1) \leqslant b$. We can improve this bound however, 

\begin{claim}\label{clm:euclidean_alg_steps}
The \#steps of Euclidean algorithm satisfies $(n+1) \leqslant 2\log_2 b$.
\end{claim}
\begin{proof}
We can analyze Euclidean algorithm two steps at a time. 

First, we look at the two equations
\begin{align}
a &= q_1 b + r_1, \\
b &= q_2 r_1 + r_2. 
\end{align}
\begin{enumerate}[label=(\roman*)]
\item If $r_1 \leqslant \frac{b}2$, then $r_2 < r_1 \leqslant \frac{b}2$.
\item If $\frac{b}2 < r_1 < b$, then $r_2 = b - q_1 r_1 < \frac{b}2$ by $q_1 \geqslant 1$, $a, b$ positive integers.
\end{enumerate}
In either case, $r_2 < \frac{b}2$. Same arguments show that 
\[
r_{k+1} < \frac{r_{k-1}}2 
\]
for any $k \in \Z$, $k > 0$.
Therefore, $r_2 < \frac{b}2$, $r_4 < \frac{b}4$, $r_6 < \frac{b}8$, \ldots, i.e., we have
\[
r_{2k} < \frac{b}{2^k}
\]
for any $k \in \Z$, $k > 0$. If $\frac{b}{2^k} = 1$ then $r_{2k} < \frac{b}{2^k} = 1 \implies r_{2k} = 0$.

Hence $(n+1) \leqslant 2k = 2\log_2 b$.
\end{proof}
\begin{remark}
Up to constants, $(n+1) \leqslant 2\log_2 b$ is the best possible.
\end{remark}

\begin{example}
Consider the Fibonacci sequence, defined as follows,
\[
F_0 = 0, ~F_1 = 1, ~F_2 = 1, \ldots, F_{n+1} = F_n + F_{n-1} \text{ for all $n \geqslant 2$}.
\]
Run Euclidean algorithm to compute $\gcd (F_{n+1}, F_n)$.
\begin{align}
  F_{n+1} &= 1 \cdot F_n + F_{n-1}, \\
   & \hspace{2mm} \vdots\\
  F_4 &= 1 \cdot F_3 + F_2, \\
  F_3 &= 1 \cdot F_2 + F_1 = 2 \cdot 1 + 0.
\end{align}
We can see the run time in this computation is \#steps $= n - 1$. One can show\footnote{No proof was given in the lecture. It is known $\displaystyle F_n = \frac{1}{\sqrt 5} \left( \left( \frac{1 + \sqrt 5}2 \right)^n - \left( \frac{1 - \sqrt 5}2 \right)^n \right)$.} there are constants $b_1, c_1$ such that for each $n$
\[
  c_1b_1^n \leqslant F_n \leqslant c_1 b_1^n + 1.
\]
Hence $n \geqslant b_2 \log_2 F_n + c_2$ for some $b_2, c_2$. Therefore the \#steps it takes to compute $\gcd (F_{n+1}, F_n)$ satisfies $n-1$ steps $\geqslant b_2 \log_2 F_n + c_2 -1$.
\end{example}
But maybe there is a better algorithm?
\begin{conjecture}
  For any algorithm computing $\gcd$ using only ``remainder function'', i.e., given $a$ and $b$, output $r$ such that $a = bq + r$ with $0 \leqslant r < b$, there is a positive constant $c$ such that
  \begin{align}
    \text{\#steps to compute $\gcd(a, b)$ } \geqslant c \log_2 a
  \end{align}
  for infinitely many pairs $a$ and $b$ with $a \geqslant b > 0$.
\end{conjecture}
\begin{theorem}[Moschovakis--van den Dries]
  For any algorithm using only $+$, $-$, ``quotient function and remainder functions'', i.e., given $a, b$ output $r$ such that $a = bq + r$ with $0 \leqslant r < b$, then there are infinitely many $a \geqslant b > 0$ such that
  \begin{align}
    \text{\#steps to compute $\gcd(a, b)$ } \geqslant \frac{1}6 \log_2 \log_2 a.
  \end{align}
\end{theorem}
\begin{remark}
  For the proof, the class is referred to \cite{Moschovakis2004}. This holds whenever $a^2 - 2b^2 = 1$ and $b > 2$. The right hand side of the inequality is independent of $b$ meaning there are infinitely many such pairs.
\end{remark}

\underline{Con}g\underline{ruences}
\begin{definition}
  Let $a, b, n \in \Z$ with $n > 0$. We say \textbf{$a$ is congruent to $b$ modulo $n$}, written as $a \equiv b \bmod n$, if $n \mid (a - b)$; if $a$ is not congruent to $b$ modulo $n$, we write $a \not \equiv b \bmod n$.
\end{definition}
\begin{example}
  $212 \equiv 0 \bmod 2$. $212 \equiv 2 \bmod 5$. $212 \not \equiv 13 \bmod 5$. 
\end{example}
\begin{proposition}
  Let $n > 0$ be a positive integer. Congruence modulo $n$ is an equivalence relation on $\Z$, i.e., congruences modulo $n$ satisfies the following:
  \begin{enumerate}[label=(\arabic*)]
  \item It is reflexive. For all $a \in \Z$, $a \equiv a \bmod n$;
  \item It is symmetric. For all $a, b \in \Z$, if $a \equiv b \bmod n$ then $b \equiv a \bmod n$;
  \item It is transitive. For any $a, b, c \in \Z$, if $a \equiv b \bmod n$, $b \equiv c \bmod n$, then $a \equiv c \bmod n$.
  \end{enumerate}
\end{proposition}
\begin{proof}
  We need to check the above three conditions.
  \begin{enumerate}[label=(\roman*)]
  \item \underline{Reflexivit}y. $a - a = 0$ and $n \mid 0$.  
  \item \underline{S}y\underline{mmetr}y. $a \equiv b \bmod n$ implies $n \mid (a - b)$. We can write $a - b = cn$ for some $c \in \Z$. Thus $b - a = (-c)n$, i.e., $n \mid (b - a)$.
  \item \underline{Transitivit}y. By $a \equiv b \bmod n$, $b \equiv c \bmod n$ we can write $a - b = xn$, $b - c = yn$ for some $x, y \in \Z$. Then $a - c = (x+y)n$, i.e., $n \mid (a - c)$.
  \end{enumerate}
\end{proof}
\subsection*{Exercise}
\begin{enumerate}
% ko80, no. 8
\item Compared with Claim~\ref{clm:euclidean_alg_steps}, show the \#steps of Euclidean algorithm to compute $\gcd (a, b)$, $a, b \in \Z$, $a > b > 0$, also satisfies $(n+1) \leqslant 5 \ell$, where $\ell =\text{\#digits of $b$}$.
% ko80, no. 54  
\item Let $a, n \in \Z$ with $a, n > 1$. $b = a^n$ is called a perfect power. Let $p$ be a prime. Show that $2^p + 3^p$ is not a perfect power.
% ko80, no. 56
\item Let $n \in \Z$. Show that
  \begin{enumerate}
  \item if $n > 1$, there does not exist an odd prime $p$ and a positive integer $m$ such that $p^n + 1 = 2^m$;
  \item if $n > 2$, there does not exist an odd prime $p$ and a positive integer $m$ such that $p^n - 1 = 2^m$.
  \end{enumerate}
% ko80, no. 64(1)
\item Given any integer $n \in \Z$, it must satisfy one of these congruences: $n \equiv 0 \bmod 2$, $n \equiv 0 \bmod 3$, $n \equiv 1 \bmod 4$, $n \equiv 5 \bmod 6$, or $n \equiv 7 \bmod 12$. (\emph{Hint}: Consider congruence class of any odd integer modulo $12$.)
\end{enumerate}
\chapter[Lecture Nine]{Day Nine \hfill {\footnotesize \rm --- 06.02.2017}}

Recall from last time, let $a, b, n \in \Z$ with $n > 0$, we say $a$ is a congruent to $b$ modulo $n$ if $n \mid (a - b)$. \\[-4mm]
\begin{definition}
  Let $n \in \Z$, $n > 0$ and let $a \in \Z$. The set of all integers congruent to $a$ modulo $n$ is called a \textbf{congruence class modulo $n$}. And we denote  this class by $[a]$.
\end{definition}
\begin{example}
  What are the congruence classes modulo 2?
  \begin{align}
    \{\text{even integers}\} &= [0] = [2] = \cdots, \\
    \{\text{odd integers}\} &= [1] = [-1] = \cdots.
  \end{align}
\end{example}
Say $n \in \Z$, $n > 0$ and $a, b \in \Z$, when is $[a] = [b]$?
\begin{claim}\label{clm:clm1_day9}
  $[a] = [b]$ if and only if $a \equiv b \bmod n$.
\end{claim}
\begin{proof} ($\Rightarrow$) If $[a] = [b]$, then $b \in [a]$. So $a \equiv b \bmod n$.

  ($\Leftarrow$) If $a \equiv b \bmod n$, for $c \in [a]$, it follows that by transitivity, $c \equiv a \bmod n$ and $a \equiv b \bmod n$ imply $c \equiv b \bmod n$. So $[a] \subseteq [b]$. Similarly, we have $[b] \subseteq [a]$.
\end{proof}
Consequently,
\begin{align}
  [a] = [b] &\iff a \equiv b \bmod n \\
            &\iff b \in [a].
\end{align}
Then if $a \not\equiv b \bmod n$, what can be said about $[a], [b]$?
\begin{claim}
  If $[a] \neq [b]$, then $[a] \cap [b] = \varnothing$.
\end{claim}
\begin{proof}
  If $[a] \cap [b] \neq \varnothing$, then there is a $c \in \Z$ such that $c \in [a]$ and $c \in [b]$, meaning $c \equiv a \bmod n$ and$ c \equiv b \bmod n$. Hence $[a] = [c] = [b]$ by Claim~\ref{clm:clm1_day9}, a contradiction.
\end{proof}
Aside, a contrapositive of a statement ``$p \implies q$'' is ``not $ q \implies$ not $p$.''
\begin{example}
  What do the congruence classes modulo 3 look like?

  They are $[0], [1], [2]$. Generally, fix $a$,
  \begin{align}
    [a] &= \{ b : b \in \Z, b \equiv a \bmod 3 \} \\
        &= \{ a + 3k : k \in \Z \}.
  \end{align}
\end{example}

\begin{definition}
  Let $n \in \Z$, $n > 0$, a set $\mathcal{S}$ of integers is called a \textbf{complete residue system modulo $n$}, if every integer is congruent to \emph{precisely} one element of $\mathcal{S}$.
\end{definition}
\begin{proposition}\label{prop:res_sys}
  Let $n \in \Z$, $n > 0$. Then $\{0, 1, \ldots, n - 1\}$ is a \textbf{complete} residue system modulo $n$.
\end{proposition}
\begin{proof}
  Let $a \in \Z$. We first show $a$ is congruent to some element in $\{ 0, 1, \ldots, n -1 \}$. By division algorithm, there are unique $q, r \in \Z$ such that $a = q n + r$ with $0 \leqslant r < n$. So $n \mid (a - r)$ and $r \in \{0, 1, \ldots, n-1\}$. We then have $a \equiv r \bmod n$.

  Now assume $r' \in \{ 0, 1, \ldots, n-1 \}$ satisfies $a \equiv r' \bmod n$. Then $a = q'n + r'$ for some $q' \in \Z$. But $0 \leqslant r' < n$, the uniqueness of $q$ and $r$ in the division algorithm implies $r' = r$.
\end{proof}
Consequently, another way to think of congruences is that $a \equiv b \bmod n \iff $ dividing $a$ and $b$ by $n$, \`a la division algorithm, gives the same remainder.

\begin{restatable}{proposition}{proptwodaynine}\label{prop:prop2_day9}
  Let $a, b, c, d \in \Z$ and let $n \in \Z$, $n > 0$. Assume $a \equiv b \bmod n$ and $c \equiv d \bmod n$. Then
  \begin{enumerate}[label=(\arabic*)]
  \item $a + c \equiv b + d \bmod n$;
  \item $ac \equiv bd \bmod n$.
  \end{enumerate}
\end{restatable}
\begin{proof} We have to prove two statements.
  \begin{enumerate}[label=(\roman*)]
  \item $a \equiv b \bmod n$ implies $n \mid (a - b)$, i.e., $a = b + k_1 n$ for some $k_1 \in \Z$. Similarly, $c = d + k_2 n$ for some $k_2 \in \Z$.    
    Then we have
    \begin{align}
      (b + d) - (a + c) &= (b + d) - ((b + k_1 n) + (d + k_2 n)) \\
                        &= -(k_1 + k_2) n.
    \end{align}
    Therefore $n \mid (b+d) - (a+c) \bmod n$, i.e., $a + c \equiv b + d \bmod n$. \label{itm:itm1_day9} 
  \item Follow ~\ref{itm:itm1_day9}, we can check
    \begin{align}
      ac - bd &= (b + k_1 n) (d + k_2 n) - bd \\
              &= (k_1d + k_2b + nk_1k_2)n.
    \end{align}
    Therefore $n \mid (ac - bd)$, i.e., $ac \equiv bd \bmod n$.
  \end{enumerate}

  We completed the proof.
\end{proof}

Consequently, we can define ``addition'' and ``multiplication'' on residue classes modulo $n$ by
\begin{align}
  [a] + [c] &= [a + c], \\
  [a][c]    &= [ac].
\end{align}
They make sense only when Proposition~\ref{prop:prop2_day9} holds.

\subsection*{Exercise}
\begin{enumerate}
% ko80, no. 42
\item Let $p$ be a prime. Show that
  \begin{enumerate}
  \item $\displaystyle \binom{n}{p} \equiv \left\lfloor \frac{n}p \right\rfloor \bmod p$;
  \item if $\displaystyle p^s \bigmid \left\lfloor \frac{n}p \right\rfloor$ for some $s \in \Z$, $s > 0$, then $\displaystyle p^s \bigmid \binom{n}p$.
  \end{enumerate}
% ko80, no. 52
\item Let $n \in \Z$. Show that $\frac{1}5 n^5 + \frac{1}3 n^3 + \frac{7}{15} n$ is also an integer.
% ko80, no. 53 
\item Let $p$ be an odd prime with $p > 3$. Show for any $a, b \in \Z$, $a b^p - b a^a \equiv 0 \bmod 6p$. (\emph{Hint}: Note that $6 \mid b(b^2 - 1)$.)
\end{enumerate}
\chapter[Lecture Ten]{Day Ten \hfill {\footnotesize \rm --- 08.02.2017}}

Recall Proposition~\ref{prop:prop2_day9} from last time,
\proptwodaynine*
Thus \emph{addition} and \emph{multiplication} modulo $n$ depends only on congruence classes. Addition and multiplication on congruence classes are well defined by
\begin{align}
  [a] + [c] &= [a + c], \\
  [a][c]    &= [ac].
\end{align}
\begin{example}
  $2 + 5 \equiv 7 \equiv 1 \bmod 6$. $8 + 23 \equiv 31 \equiv 1 \bmod 6$.
\end{example}
\begin{example}
  Working with modulus 6, we have $[2] = [8]$ and $[5] = [23]$. Then $[2] + [5] = [8] + [23]$ since $[7] = [1] = [31]$.
\end{example}
\begin{example}
  Working with modulus 2, we have the familiar assertion in grade school,
\begin{align}
  \text { even + even } & \text{= even, } \\
  \text { even + odd } & \text{= odd, } \\
  \text { odd + odd } & \text{= even.}
\end{align}
Similarly with multiplication, 
\begin{align}
  \text { even $\times$ even } & \text{= even, } \\
  \text { even $\times$ odd } & \text{= even, } \\
  \text { odd $\times$ odd } & \text{= odd.}
\end{align}
\end{example}

Another (down-to-earth) way to think about this: from last lecture, we saw in Proposition~\ref{prop:res_sys} that $\{ 0, 1, \ldots, n -1 \}$ is a complete residue system modulo $n$. So $[0], [1], \ldots, [n-1]$ are distinct congruence classes. Then if $0 \leqslant a, c \leqslant n - 1$, $[a] + [c] = [r]$ with $a + c = qn + r$ and $0 \leqslant r < n$ by division algorithm.

Similar argument applies to $[a] [c] = [r']$ with $a c = q'n + r'$ and $0 \leqslant r' < n$.

\begin{notation}
The set of congruence classes with the above operations of addition and multiplication is denoted by $\Z_n$ or $\Z / n\Z$.
\end{notation}
\begin{example}
  Consider addition of elements in $\Z_3$,
  \begin{align}
    \begin{array}{c|ccc}
       +  & [0] & [1] & [2] \\[0mm] \hline % here \\ needs [0mm] argument since otherwise [0] in the next row will be treated as length argument
      [0] & [0] & [1] & [2] \\[0mm] 
      [1] & [1] & [2] & [0] \\[0mm]
      [2] & [2] & [0] & [1]
    \end{array}.
  \end{align}
  Consider multiplication of elements in $\Z_3$,
  \begin{align}
    \begin{array}{c|ccc}
  \times  & [0] & [1] & [2] \\[0mm] \hline
      [0] & [0] & [0] & [0] \\[0mm] 
      [1] & [0] & [1] & [2] \\[0mm]
      [2] & [0] & [2] & [1]
    \end{array}.
  \end{align}
\end{example}

\begin{example}\label{ex:z4}
  Consider addition of elements in $\Z_4$,
  \begin{align}
    \begin{array}{c|cccc}
       +  & [0] & [1] & [2] & [3] \\[0mm]\hline
      [0] & [0] & [1] & [2] & [3] \\[0mm] 
      [1] & [1] & [2] & [3] & [0] \\[0mm]
      [2] & [2] & [3] & [0] & [1] \\[0mm]
      [3] & [3] & [0] & [1] & [2]
    \end{array}.
  \end{align}
  Consider multiplication of elements in $\Z_4$,
  \begin{align}
    \begin{array}{c|cccc}
  \times  & [0] & [1] & [2] & [3] \\[0mm]\hline
      [0] & [0] & [0] & [0] & [0] \\[0mm] 
      [1] & [0] & [1] & [2] & [3] \\[0mm]
      [2] & [0] & [2] & {\color{red}[0]} & [2] \\[0mm]
      [3] & [0] & [3] & [2] & [1]
    \end{array}.
  \end{align}
Note that the multiplication of two nonzero elements $[2], [2]$ gives zero in the multiplication table.
\end{example}
\begin{remark}
  Usual arithmetic rules hold modulo $n$ except for $[a], [b] \neq 0$, we could still have $[a][b]=0$ (or $[a][b] = 0$ does not necessarily imply $[a] = 0$ or $[b] = 0$.) Such nonzero $a, b$ that $ab = 0$ are called \textbf{zero divisors} (in a \textbf{ring}). Consequently, cancellation modulo $n$ is not trivial. In the above Example~\ref{ex:z4}, $2 \times 2 \equiv 0 \bmod 4$ but $2 \not\equiv 0 \bmod 4$.
\end{remark}

\begin{proposition}\label{prop:cong_cancel}
  Let $a, b, c, n \in \Z$, $n > 0$. Then $ca \equiv cb \bmod n$ if and only if $a \equiv b \bmod \frac{n}{\gcd (c, n)}$.
\end{proposition}
\begin{proof}
($\Rightarrow$) Assume $ca \equiv cb \bmod n$. Then $n \mid (ca - cb)$. Let $d = \gcd(c, n)$, we have $\frac{n}d \mid \frac{c}d (a - b)$. But $\frac{n}d, \frac{c}d$ are coprime, by Lemma~\ref{lem:lem1_day7}, we have $\frac{n}d \mid (a - b)$, i.e., $a \equiv b \bmod \frac{n}d$.

($\Leftarrow$) If $a \equiv b \bmod n$, then $a - b = e \left( \frac{n}d \right)$ for some $e \in \Z$. It further implies $d(a - b) = en$. We also know that $c = fd$ for some $f \in \Z$. So $(fd)(a - b) = c(a - b) = (fe)n$, i.e., $ca \equiv cb \bmod n$.
\end{proof}

For $a, b, n \in \Z$, $n > 0$, an equation $ax \equiv b \bmod n$ is called a \textbf{linear congruence equation modulo $n$}.

Q\underline{uestion}: When can we find a solution $x \in \Z$ to this equation?

If a solution exists, it means $\exists \, x \in \Z$ such that $n \mid (ax - b)$, i.e., $ax - b = ny$ for some $y \in \Z$. Thus if a solution exists it implies $\exists \, x, y \in \Z$ such that $ax - ny = b$ (Diophantine equation in two variables). Therefore, $ax \equiv b \bmod n$ has an integer solution if and only if $b$ is a linear combination of $a$ and $n$. Further, it has a solution if and only if $\gcd (a, n) \mid b$.

\begin{example}
$10x \equiv 5 \bmod 12$ has no solutions since $\gcd (10, 12) = 2 \nmid 5$. $10x \equiv 8 \bmod 12$ has solutions however, since $\gcd (10, 12) = 2 \mid 8$.
\end{example}

We saw in Proposition~\ref{prop:lin_diophantine} that if $x_0, y_0 \in \Z$ satisfy $a x_0 +  n y_0 = b$, then the complete set of solutions to $ax + ny = b$ is given by
\begin{align}
\left\{ \begin{array}{l}
         x = x_0 + \displaystyle \frac{n}{\gcd (a, n)}k \\
          \\
         y = y_0 - \displaystyle \frac{a}{\gcd (a, n)}k 
        \end{array} , \text{ for } k \in \Z \right. .
\end{align}

The set of solutions to $ax \equiv b \bmod n$ 
\[
         x = x_0 + \displaystyle \frac{n}{\gcd (a, n)}k, \text{ for } k \in \Z,
\]
when modulo $n$, these $x$'s repeat and there are $\gcd (a, n)$ many distinct solutions modulo $n$. And they are given by
\[
x_0, \, x_0 + 1\left( \frac{n}a \right), \, \ldots, \, x_0 + (d-1)\left( \frac{n}a \right), 
\]
where $d = \gcd(a, n)$. So we have the following theorem.

\begin{theorem}\label{thm:num_soln_lin_cong}
The linear congruence $ax \equiv b \bmod n$ has a solution if and only if $\gcd (a, n) \mid b$. If this is the case, it has $\gcd (a, n)$ many distinct solutions modulo $n$.
\end{theorem}

\begin{corollary}\label{cor:lin_cong_soln}
If $\gcd (a, n) = 1$, then $ax \equiv b \bmod n$ has a \textbf{unique} solution modulo $n$.
\end{corollary}
\begin{corollary}
$ax \equiv 1 \bmod n$ has a solution if and only if $\gcd (a, n) = 1$.
\end{corollary}
If $ax \equiv 1 \bmod n$ for some $x \in \Z$, we say $x$ is the \textbf{multiplicative inverse of $a$ modulo $n$}. If such an inverse of $a$ modulo $n$ exists, written as $a^{-1}$, it is unique modulo $n$.

\subsection*{Exercise}
\begin{enumerate}
% ko80, no. 22  
\item Let $n \in \Z$. Show $504 \mid n^9 - n^3$.
% ko80, no. 28
\item Solve $(n - 1)! = n^k - 1$ for all $n, k \in \Z$. (\emph{Hint}: There are three solutions: $(n, k) = (2,1)$, $(3,1)$, and $(5,2)$. Show when $n > 5$, there are no integer solutions.)
\end{enumerate}
\chapter[Lecture Eleven]{Day Eleven \hfill {\footnotesize \rm --- 10.02.2017}}

Recall last time we saw $a \in \Z$ has a multiplicative inverse modulo $n$, i.e., $ax \equiv 1 \bmod n$ has an integer solution $x$ if and only if $\gcd (a, n) = 1$.
\begin{definition}
  The \textbf{Euler $\varphi$-function}, denoted $\varphi$, is given by
  \[
    \varphi(n) = | \left\{ a \in \Z : 1 \leqslant a \leqslant n, ~\gcd(a, n) = 1\right\} | 
  \]
for any positive integer $n$.
\end{definition}
\begin{example}\label{ex:ex1_day11}
  Quick examples.
  \begin{itemize}
  \item $\varphi(6) = 2$ with $n = 6$, $a = 1, 5$.
  \item $\varphi(10) = 4$ with $n = 10$, $a = 1, 3, 7 \text{ and } 9$.
  \item $\varphi(15) = 8$ with $n = 15$. Later we will know that $\varphi(15) = \varphi(3)\varphi(5) = (3-1) \times (5-1)$.
  \item If $p$ is prime, $\varphi(p) = p - 1$.
  \end{itemize}
\end{example}
For a fixed $n \in \Z$, $n > 0$ and a prime $p$, either $p \mid n$ or $\gcd (p, n) = 1$. 
\begin{theorem}[Dirichlet]
  For any $a, n \in \Z$, $n > 0$ and $\gcd (a, n) = 1$, there are infinitely many primes congruent to $a$ modulo $n$. Phrased another way, it says there are infinitely many primes in the arithmetic progression,
  \[
    a, \, a + n, \, \ldots , a + kn, \, \ldots 
  \]
  where $k \in \Z$, $k > 0$.
\end{theorem}
\begin{remark}
  Dirichlet's theorem on arithmetic progressions is one of the ``biggest'' theorems in analytic number theorem. The proof is beyond this course therefore it will not be given in the lecture. A related theorem about primes in arithmetic progression is Green-Tao~\cite{Green2004}, which states that the sequence of primes contains arbitrarily long arithmetic progressions.
\end{remark}

\begin{example}\label{ex:ex2_day11}
  There are infinitely many primes of the form $4k+1$ and $4k+3$, where $k \in \Z$, $k > 0$.
\end{example}

Recall the prime counting function
\[
\pi (x) = | \left\{ p : \text{$p$ is prime, $p \leqslant x$} \right\} |
\]
and the prime number theorem which states 
\[
\pi (x) \sim \frac{x}{\log x}.
\] 

Let $a, n \in \Z$, $n > 0$. Define 
\[
\pi (x; n, a) = | \left\{ p \leqslant x : \text{$p$ is prime, } p \equiv a \bmod n \right\} |.
\]

\begin{theorem}[Dirichlet's theorem, strong form]
If $\gcd (a, n) = 1$, then 
\[
\pi (x; n, a) \sim \frac{1}{\varphi(n)} \frac{x}{\log x}.
\]
\end{theorem}

\begin{remark}
  The moral of the strong form of Dirichlet's theorem is that to the first order, primes are evenly distributed among the residue classes modulo $n$ that are coprime with $n$. Computationally, it seems some residue classes actually have ``more'' primes, i.e., the second order term of the distribution of primes among the residue classes modulo $n$ is different.
\end{remark}

\begin{conjecture}
   Let $a, b, n \in \Z$, $0 < a, b < n$ with $\gcd (a, n) = \gcd (b, n) = 1$. If $a$ is a quadratic residue\footnote{We will discuss ``quadratic residue/nonresidue modulo $n$'' in detail later in the course.} modulo $n$ while $b$ is a quadratic nonresidue modulo $n$, then $\pi (x; n, b) > \pi (x; n, a)$ more often than not.
\end{conjecture}
\begin{remark}
  $\pi (x; 4, 3) > \pi (x; 4, 1)$ more often than not. The density of such $x$ is greater than 0.99. This phenomenon was first observed by P. Chebyshev in the 1850s. The problem is called \textbf{Chebyshev bias} or \textbf{prime number race}.
\end{remark}

We restate the second half of Example~\ref{ex:ex2_day11} as a theorem and prove it below without using Dirichlet's theorem but using an elementary way instead.

\begin{theorem}
  There are infinitely many primes congruent to 3 modulo 4. Equivalently, there are infinitely many primes of the form $4k+3$, where $k \in \Z$, $k \geqslant 0$.
\end{theorem}
\begin{proof}
  Assume the contrary, there are only finitely many primes of the form $4k+3$, where $k \in \Z$, $ k \geqslant 0$. Label them as 
\[
p_0, p_1, \ldots, p_m \text{ with $p_0 = 3$, $m \in \Z$, $m > 0$.}
\]
Consider the number $N = 4p_1p_2\cdots p_m + 3$. Obviously $N \equiv 3 \bmod 4$.

\begin{enumerate}[label=(\roman*)]
\item $2 \nmid N$ since otherwise $2 \mid N - 4p_1p_2 \cdots p_m = 3 \implies 2 \mid 3$, a contradiction. 
\item Similarly, $p_i \nmid N$ for $1 \leqslant i \leqslant m$. (Otherwise $p_i \mid 3$ for $1 \leqslant i \leqslant m$.) 
\item Lastly, $3 \nmid N$ since otherwise $3 \mid N - 3 \implies 3 \mid 4p_1p_2 \cdots p_m$. Then $3 \mid p_i$ for some $1 \leqslant i \leqslant m$ by 3 being a prime, a contradiction.
\end{enumerate}
Therefore, $N$ has no divisors $d$, $d \in \Z$ such that $d \equiv 0, 2 \bmod 4$ (by $2 \nmid N$) or $d \equiv 3 \bmod 4$ (by $N > 1$ has a prime divisor but those primes of the form $4k+3$ have been ruled out; there are finitely many and none of them divides $N$.) So the only divisors of N are congruent to 1 modulo 4. 

We can write $N = q_1q_2 \cdots q_n$, $q_j \equiv 1 \bmod 4$ for $1 \leqslant j \leqslant n$, $n \in \Z$ and $n > 0$. Then we have 
\begin{align}
  N &= q_1 q_2 \cdots q_n \\
    &\equiv 1 \cdot 1 \, \cdots\, \cdot 1 \bmod 4 \\
    &\equiv 1 \bmod 4,
\end{align}
a contradiction.
\end{proof}

Back to linear congruence equations, clearly, $x \equiv b \bmod n$ has a unique solution modulo $n$.

Q\underline{uestion}: Can we find $x \in \Z$ such that
\begin{align}
  \left\{ \begin{array}{l}
            x \equiv a \bmod m \\[1mm]
            x \equiv b \hspace{.6mm} \bmod n
          \end{array}\right. 
\end{align} 
with $a, b, m, n \in \Z$, $m, n > 0$?

\begin{example}\label{ex:ex3_day11}
  Solve linear congruences 
\begin{align}
  \left\{ \begin{array}{l}
            x \equiv 4 \bmod 7 \\[1mm]
            x \equiv 5 \bmod 9.
          \end{array}\right.
\end{align} 
\end{example}
\begin{proof}[Solution]
  We can write $x - 4 = 7\ell$ and $x - 5 = 9q$, where $\ell, q \in \Z$. If we plug the second equation into the first one, then we have $7\ell + 4 = 9q + 5$, i.e., $7\ell \equiv 1 \bmod 9$.
  We use Euclidean algorithm to solve for $\ell$,
  \begin{align}
    9 &= 1 \times 7 + 2 \\
    7 &= 3 \times 2 + 1 \\
    2 &= 2 \times 1 + 0.
  \end{align}
  By back substitution,
  \begin{align}
    1 &= 7 - 3 \times 2 \\
      &= 7 - 3 \times (9 - 7) \\
      &= 4 \times 7 - 3 \times 9.
  \end{align}
  So we can take $\ell = 4$. $x = 4 + 7 \times 4 = 32$. $x = 32$ solves
  \begin{align}
  \left\{ \begin{array}{l}
            x \equiv 4 \bmod 7 \\[1mm]
            x \equiv 5 \bmod 9.
          \end{array}\right.
\end{align} 
\end{proof}

\subsection*{Exercise}
\begin{enumerate}
% ko80, no. 2
\item Let $a, m, n \in \Z$, $m, n > 0$ with $\gcd (a, n) = 1$. Consider arithmetic progression
  \[
    a, \, a + n, \, \ldots , a + kn, \, \ldots 
  \]
  where $k \in \Z$, $k > 0$. Show there are infinitely many $k$'s such that $\gcd (a + k n, m) = 1$.
\end{enumerate}

\chapter[Lecture Twelve]{Day Twelve \hfill {\footnotesize \rm --- 13.02.2017}}

\underline{Chinese remainder theorem} (CRT)
\begin{theorem}
  Let $n_1, n_2, \ldots, n_k$ be coprime positive integers, i.e, $\gcd (n_i, n_j) = 1$ if $i \neq j$ for $1 \leqslant i, j \leqslant k$ where $k \in \Z$, $k > 0$. Let $b_1, b_2, \ldots, b_k \in \Z$. The system of congruences
  \begin{align}
    x &\equiv b_1 \bmod n_1 \\
    x &\equiv b_2 \bmod n_2 \\
    & \hspace{1.7mm} \vdots \\
    x &\equiv b_k \bmod n_k
  \end{align}
  has a \textbf{unique} solution modulo $(n_1n_2 \cdots n_k)$.
\end{theorem}
\begin{proof}\footnote{This proof is based on the technique used in Example~\ref{ex:ex3_day11} then apply induction. The proof is different from~\cite{Strayer2001}*{\S 2.3, Theorem~2.9}, which is quite tricky.}We induct on $k \geqslant 1$. 

To prove the \underline{existence} of such an $x$, if $k = 1$, it is clear that
  \[
    x_1 \equiv b_1 \bmod n_1     
  \]
  has a unique solution modulo $n_1$.

  Assume the result is true for $k - 1 \geqslant 1$. By our induction hypothesis, there is an $x_0 \in \Z$ such that
    \begin{align}
    x_0 &\equiv b_1 \bmod n_1 \\
    x_0 &\equiv b_2 \bmod n_2 \\
    & \hspace{1.7mm} \vdots \\
    x_0 &\equiv b_{k-1} \bmod n_{k-1} ,
  \end{align}
  and it is unique modulo $N = n_1n_2 \cdots n_{k-1}$. Now consider the system of two congruences
    \begin{align}
\label{eqn:crt1}      x &\equiv x_0 \bmod N \\
\label{eqn:crt2}      x &\equiv b_k \bmod n_k .
    \end{align}
    Equation~\eqref{eqn:crt2} implies $x - b_k = n_k \ell$ for some $\ell \in \Z$. 

    Plug $x = b_k + n_k \ell$ into Equation~\eqref{eqn:crt1}, we have 
    \begin{align}
\label{eqn:crt3}      n_k \ell \equiv x_0 - b_k \bmod \prod_{i=1}^{k-1} n_i .
    \end{align}
    Since $\gcd (n_k, n_i) = 1$ for $1 \leqslant i \leqslant k-1$, it implies $\gcd (n_k, \prod_{i=1}^{k-1} n_i) = 1$. Indeed, if there is a prime $p$ such that $p \mid n_k$ and $p \mid \prod_{i=1}^{k-1} n_i$, then $p \mid n_i$ for some $1 \leqslant i \leqslant k-1$. Hence $\gcd(n_k, n_i) \neq 1$ for some $1 \leqslant i \leqslant k-1$.

    Since $\gcd (n_k, \prod_{i=1}^{k-1} n_i) = 1$ in Equation~\eqref{eqn:crt3}, we can find $\ell \in \Z$ such that 
\[
n_k \ell \equiv x_0 - b_k \bmod N.
\]
Set $x = b_k + n_k\ell$. Clearly $x \equiv b_k \bmod n_k$. And by construction 
\[
x \equiv x_0 \bmod \prod_{i=1}^{k-1}n_i,
\]
therefore by reducing the modulus to its divisor\footnote{Trick!},
\begin{align}
\begin{split}\label{eqn:crt4}
x &\equiv x_0 \bmod n_i \\
&\equiv b_i \hspace{.8mm}\bmod n_i
\end{split}
\end{align}
for each $1 \leqslant i \leqslant k-1$. The second congruence in Equation~\eqref{eqn:crt4} is by our induction hypothesis.

To prove the \underline{uni}q\underline{ueness} of this $x$, let $y \in \Z$ be another solution such that $y \equiv b_i \bmod n_i$ for each $1 \leqslant i \leqslant k$. Our induction hypothesis ($x$ is unique up to $k-1$ congruences) gives 
\begin{align}
  y &\equiv x_0 \bmod n_1n_2 \cdots n_{k-1} \\
    &\equiv x \hspace{1.5mm}\bmod n_1n_2 \cdots n_{k-1}.
\end{align}
Together with $y \equiv x \bmod n_k$, we have 
\begin{align}
n_1n_2 \cdots n_{k-1} &\mid y - x, \\
               n_k   &\mid y - x.
\end{align}
Therefore\footnote{Similar to what we have seen for the greatest common divisors $\{ \text{$d:$ $d =$ common divisors of $a, b$} \} = \{ \text{multiples of $\gcd (a, b)$} \}$, we have $\{ \text{$e:$ $e=$ common multiples of $a, b$} \} = \{ \text{multiples of $\text{lcm} (a, b)$} \}$.} $\text{lcm} (\prod_{i=1}^{k-1} n_i, n_k) \mid y - x$. But $\gcd (\prod_{i=1}^{k-1} n_i, n_k) = 1$, so $\text{lcm} (\prod_{i=1}^{k-1} n_i, n_k) = n_k \prod_{i=1}^{k-1} n_i = \prod_{i=1}^k n_i$. Hence $n_1n_2 \cdots n_k \mid y - x$, i.e., $y \equiv x \bmod n_1n_2 \cdots n_k$.
\end{proof}
\begin{remark}
In practice, we can solve these congruences inductively by first finding a solution to the first $k - 1$ congruences. Then solve the two congruences 
\begin{align}
  x &\equiv x_0 \bmod n_1n_2 \cdots n_{k-1} \\
  x &\equiv b_k \hspace{0.35mm}\bmod n_k.
\end{align}
\end{remark}

\begin{example}
Find a solution to 
\begin{align}
  x &\equiv 1 \bmod 3 \\
  x &\equiv 2 \bmod 4 \\
  x &\equiv 4 \bmod 5.
\end{align}
\end{example}
\begin{proof}[Solution]
  The first congruence implies $x = 3k + 1$ for some $k \in \Z$. Plug it into the second congruence, we have
  \begin{align}
    1 + 3k &\equiv 2 \bmod 4 \implies \\
    3k &\equiv 1 \bmod 4.
  \end{align}
  Quick inspection tells us we can take $k = 3$. So $x = 3\times 3 + 1 = 10$ solves the first two congruences
  \begin{align}
    x &\equiv 1 \bmod 3 \\
    x &\equiv 2 \bmod 4.
  \end{align}
  By Chinese remainder theorem, the first two congruences are equivalent to $x \equiv 10 \bmod 12$. Hence we reduce the original system of congruences to
  \begin{align}
    x &\equiv 10 \bmod 12 \\
    x &\equiv 4 \hspace{1.7mm}\bmod 5.
  \end{align}
  Repeat the first step in this example to solve the new system of two congruences. The first congruence in the new system implies $x = 10 + 12 \ell$ for some $\ell \in \Z$. Plug it into the second equation $x \equiv 4 \bmod 5$, we have
  \begin{align}
    10 + 12 \ell &\equiv 4 \bmod 5 \implies \\
    0 + 2 \ell &\equiv 4 \bmod 5.
  \end{align}
  So we can take $\ell = 2$. Thus $x = 10 + 2\times 2 = 34$ solves the original three congruences. It is unique modulo $3\times 4 \times 5 = 60$. 
\end{proof}
Notice that in the previous example we used/saw
\[
  3^2 \equiv 1 \bmod 4.
\]
Actually, $\gcd (3, 4) = 1$ and $\varphi(4) = 2$ in this case.

Recall the Euler $\varphi$-function,
\[
  \varphi (n) = | \left\{ a \in \Z : 1 \leqslant a \leqslant n, \, \gcd (a, n) = 1 \right\} |.
\]

\begin{restatable}[Euler's theorem]{theorem}{eulerthm}\label{thm:euler_thm}
  Let $a, n \in \Z$, $n > 0$ and $\gcd (a, n) = 1$. Then
  \[
    a^{\varphi(n)} \equiv 1 \bmod n.
  \]
\end{restatable}
\begin{proof}
  Euler's theorem will be proved in Lecture~\ref{chp:euler_thm}.
\end{proof}
\begin{corollary}[Fermat's little theorem]
  For any prime $p$ and $a \in \Z$ with $p \nmid a$, we have
  \[
    a^{p-1} \equiv 1 \bmod p.
  \]
\end{corollary}
\begin{proof}
  Assuming Euler's theorem, if $p \nmid a$, then $\gcd(a, n) = 1$. It follows from Euler's theorem that $a^{p-1} \equiv 1 \bmod p$ since $\varphi(p) = p - 1$.
\end{proof}
\begin{corollary}[of Fermat's little theorem]
  For any prime $p$ and any $a \in \Z$, we have
  \[
    a^p \equiv a \bmod p.
  \]
\end{corollary}
\begin{proof}
  We consider two cases.
  \begin{enumerate}[label=(\roman*)]
  \item If $p \nmid a$, we can apply Fermat's little theorem. Then $a^{p-1} \equiv 1 \bmod p \implies a^{p-1}\cdot a \equiv 1\cdot a \bmod p$, i.e., $a^p \equiv a \bmod p$.
  \item If $p \mid a$, then $p \mid a^p$. Therefore
    \begin{align}
      a^p &\equiv 0 \bmod p \\
          &\equiv a \bmod p.
    \end{align}
  \end{enumerate}

  We completed the proof.
\end{proof}
\begin{example}
  Find the smallest positive integer in the congruence class of $3^{537} \bmod 11$.
\end{example}
\begin{proof}[Solution]
  By Fermat's little theorem,
  \[
    3^{10} \equiv 1 \bmod 11.
  \]
  Therefore
  \begin{align}
    3^{537} &= (3^{10})^{53} \times 3^7 \\
            &\equiv 1^{53} \times 3^7 \hspace{2.3mm}\bmod 11 \\
            &\equiv (3^2)^3 \times 3 \hspace{1.1mm}\bmod 11 \\
            &\equiv (-2)^3 \times 3 \bmod 11 \\
            &\equiv -24 \bmod 11 \\
            &\equiv 9 \hspace{4.5mm}\bmod 11.
  \end{align}
  Note that even with $3^7$, we still want to reduce it to a smaller number that is manageable.
\end{proof}

\subsection*{Exercise}
\begin{enumerate}
 % ko80, no. 44
\item Let $(a_n)$ be a sequence in $\Z_{>0}$ with $a_{n + 1} = 2 a_n + 1$ for $n = 1, 2, \ldots\, $ Show that it cannot be the case that all $a_n$'s are prime. (\emph{Hint}: Let $p$ be an odd prime. Then $p \mid 2^{p-1} p + 2^{p-1} - 1$.)
% ko80, no. 55
\item Find the smallest positive integer $n$ such that $\frac{1}2 n$ is a square; $\frac{1}3 n$ is a cube; and $\frac{1}5 n$ is a fifth power. 
\end{enumerate}

\chapter[Lecture Thirteen]{Day Thirteen \hfill {\footnotesize \rm --- 15.02.2017}}\label{chp:euler_thm}

Recall Euler's theorem from last time,
\eulerthm*
Recall the notation $\varphi (n) $, which is the Euler $\varphi$-function counting all the positive integers up to $n$ that are coprime with $n$.
\begin{proof}[Proof {\rm (of Euler's theorem)}]
  Let $r_1, r_2, \ldots, r_{\varphi(n)}$ be the $\varphi (n)$ distinct integers satisfying $1 \leqslant r_i \leqslant n$ and $\gcd(r_i, n) = 1$. Consider the $\varphi (n)$ integers
\[
ar_1, \, ar_2, \, \ldots, \, ar_{\varphi(n)}.
\]
We claim $\gcd (ar_i, n) = 1$ for each $1 \leqslant i \leqslant \varphi(n)$. Indeed, if $p$ is a prime such that $p \mid ar_i$ for some $1 \leqslant i \leqslant \varphi(n)$ then $p \mid a$ or $p \mid r_i $. But $p \nmid n$ since $\gcd (a, n) = \gcd(r_i, n) = 1$.

Moreover we also claim $ar_i \not\equiv ar_j \bmod n$ if $i \neq j$. Indeed, assume $ar_i \equiv ar_j \bmod n$, then $n \mid a(r_i - r_j)$. But $\gcd (a, n) = 1$ so we have $n \mid r_i - r_j$, i.e., $r_i \equiv r_j \bmod n$. By our choice of $r_i$ for $1 \leqslant i \leqslant \varphi(n)$, they are distinct if $i \neq j$.\footnote{Another way to see this: Since $\gcd (a, n) = 1$, there is a unique $b \bmod n$ such that $ba \equiv 1 \bmod n$. Assume $ar_i \equiv ar_j \bmod n$, then $ba r_i \equiv ba r_j \bmod n$, which again implies $r_i = r_j$.} 

Therefore each $ar_i$ is congruent to precisely one $r_j$.\footnote{How this works in the proof: By $\gcd (ar_i,n) = 1$ and $ar_i \equiv c \bmod n$ for some $c \in \Z$ up to $n$, we have $\gcd (c, n) = \gcd (ar_i, n) = 1$, meaning $c$ is coprime with $n$. But we have selected all those integers up to $n$ and coprime with $n$, i.e, the $r_i$'s. So $ar_i \equiv c \equiv r_j \bmod n$.} In particular,
\begin{align}
  (ar_1)(ar_2) \cdots (ar_{\varphi(n)}) &\equiv r_1 r_2 \cdots r_{\varphi(n)} \bmod n \\
\label{eqn:eqn1_day13}  a^{\varphi(n)} r_1 r_2 \cdots r_{\varphi(n)} &\equiv r_1 r_2 \cdots r_{\varphi(n)} \bmod n .
\end{align}
But $\gcd (r_1 r_2 \cdots r_{\varphi(n)}, n) = 1$ since $\gcd (r_i, n) = 1$ for each $1 \leqslant i \leqslant \varphi(n)$. We can cancel $r_1 r_2 \cdots r_{\varphi(n)}$ on both sides in Equation~\eqref{eqn:eqn1_day13} then we have the result as desired.
\end{proof}
\begin{remark}
  We did use in the proof that if $a \equiv b \bmod n$ and $\gcd (a, n) = 1$, then $\gcd (b, n) = \gcd (a, n) = 1$.
\end{remark}
\begin{example}
  Find the smallest positive integer in the congruence class of $2^{209} \bmod 15$.
\end{example}
\begin{proof}[Solution]
  Recall $\varphi(15) = 8$ in Example~\ref{ex:ex1_day11} and $\gcd (2, 15) = 1$. We can apply Euler's theorem, then we have
  \[
    2^8 \equiv 1 \bmod 15.
  \]
  Also $209 = 26\times 8 + 1$. So
  \begin{align}
    2^{209} &= (2^8)^{26}\times 2 \\
            &\equiv 2 \bmod 15.
  \end{align}
\end{proof}
Q\underline{uestion}: How do we compute $\varphi(n)$?

\begin{proposition}\label{prop:phi_multi}
  Let $m, n \in \Z$ be positive coprime integers. Then $\varphi(mn) = \varphi(m) \varphi(n)$.
\end{proposition}
\begin{proof}
  Let $1 \leqslant a \leqslant mn$ satisfy $\gcd (a, mn) = 1$. Then there is a unique $1 \leqslant b \leqslant m $ and a unique $1 \leqslant c \leqslant n $ such that
  \[
    a \equiv b \bmod m \text{ and } a \equiv c \bmod n.
  \]
Indeed, consider $a \equiv b \bmod m$ and $a \equiv b' \bmod m$ with $1 \leqslant b, b' \leqslant m$, $b, b' \in \Z$. $b \equiv a \equiv b' \bmod m \implies m \mid b - b'$. Since $b, b'$ are restricted between 1 and $m$, we have $b' = b$. Also
\begin{align}
\gcd (a, mn) = 1 & \implies \gcd (a, m) = 1 \\ 
                 & \implies \gcd (b, m) = \gcd (a, m)  = 1. & \btfact{\text{(by $a \equiv b \bmod m$)}} 
\end{align}  
Similarly, $\gcd (a, mn) = 1 \implies \gcd (c, n) = \gcd (a, n) = 1$.
  This shows given $1 \leqslant a \leqslant mn$ with $\gcd (a, mn) = 1$, we get unique $1 \leqslant b \leqslant m$ and $1 \leqslant c \leqslant n$ with $\gcd (b, m) = 1$ and $\gcd (c, n) = 1$.

  On the other hand, if we have $1 \leqslant b \leqslant m$ and $1 \leqslant c \leqslant n$ satisfying $\gcd (b, m) = 1$ and $\gcd (c, n) = 1$, since $\gcd (m, n) = 1$, by Chinese remainder theorem, there is a unique $1 \leqslant a \leqslant mn$ such that 
  \[
    a \equiv b \bmod m \text{ and } a \equiv c \bmod n.
  \]
$\gcd (b, m) = 1$ and $a \equiv b \bmod m$ imply $\gcd (a, m) = \gcd (b, m) = 1$. Similarly $\gcd (c, n) = 1$ and $a \equiv c \bmod n$ imply $\gcd(a, n) = \gcd (c, n) = 1$. Hence $\gcd (a, mn) = 1$.

Therefore we have constructed a bijection between 
\[
\left\{ a \in \Z : 1 \leqslant a \leqslant mn, \, \gcd(a, mn) = 1\right\}
\]
and 
\[
\left\{ (b,c) \in \Z \times \Z : 1 \leqslant b \leqslant m, \, 1 \leqslant c \leqslant n, \, \gcd(b, m) = \gcd(c,n) = 1\right\}.
\]
Thus the two sets have the same cardinality (size), i.e., $\varphi(mn) = \varphi(m)\varphi(n)$.
\end{proof}
\begin{remark}
  This proposition is not true without the coprime assumption on $m, n > 0$. For example, $\varphi(4) = 2 \neq 1 \times 1 = \varphi(2) \varphi(2)$.
\end{remark}
\begin{theorem}\label{thm:phi_n}
  Let $n \in \Z$, $n \geqslant 2$ and write $n = p_1^{e_1}p_2^{e_2}\cdots p_k^{e_k}$, $k \in \Z$, $k > 0$, where $p_i$'s are distinct primes with $e_i \geqslant 1$ for $1 \leqslant i \leqslant k$. Then 
\[
\varphi(n) = (p_1 - 1)p_1^{e_1-1}(p_2 - 1)p_2^{e_2-1}\cdots (p_k - 1)p_k^{e_k-1}.
\]
\end{theorem}
Let us state a lemma before we prove Theorem~\ref{thm:phi_n}.
\begin{lemma}\label{lem:phi_p}
  Let $e$ be a positive integer and $p$ a prime. Then $\varphi(p^e) = (p-1)p^{e-1}$.
\end{lemma}
\begin{proof}
  Let $a \in \Z$, $1 \leqslant a \leqslant p^e$. We know
  \[
    \gcd (a, p^e) \neq 1 \iff p \mid a.
  \]
  Thus the integers $1 \leqslant a \leqslant p^e$ such that $\gcd (a, p^e) \neq 1$ are multiples of $p$, 
  \[
    p, \, 2p, \, \ldots, \, (p^{e-1}-1)p, \, p^{e-1}p.
  \]
  The number of multiples of $p$ in this range is $p^{e-1}$. Hence
  \begin{align}
    \varphi(p^e) &= p^e - p^{e-1} \\
                 &= (p-1) p^{e-1}.
  \end{align}
\end{proof}

\begin{proof}[Proof {\rm (of Theorem~\ref{thm:phi_n})}]
  Since $p_1, p_2, \ldots, p_k$ are distinct primes, $p_1^{e_1}, p_2^{e_2}, \ldots, p_k^{e_k}$ are pairwise coprime. By Proposition~\ref{prop:phi_multi}, Euler $\varphi$-function is multiplicative.\footnote{We will discuss multiplicative arithmetic functions in detail later in the course.} We can repeatedly apply Proposition~\ref{prop:phi_multi},
  \begin{align}
    \varphi(n) &= \varphi(p_1^{e_1}p_2^{e_2}\cdots p_k^{e_k}) \\
               &= \varphi(p_1^{e_1}) \varphi(p_2^{e_2}\cdots p_k^{e_k}) \\
               & \hspace{1.8mm} \vdots  \\
               &= \varphi(p_1^{e_1})\varphi(p_2^{e_2}) \cdots \varphi(p_k^{e_k}) \\
               &= (p_1 - 1)p_1^{e_1-1}(p_2 - 1)p_2^{e_2-1}\cdots (p_k - 1)p_k^{e_k-1}.
  \end{align}
  The last equality is due to Lemma~\ref{lem:phi_p} we just proved.
\end{proof}
\begin{example}
  Compute $\varphi(90)$.
\end{example}
\begin{proof}[Solution]
  \begin{align}
    \varphi(90) &= \varphi(2 \times 3^2 \times 5) \\
                &= (2-1) \times (3-1)3 \times (5-1) \\
                &= 24.
  \end{align}
\end{proof}
\subsection*{Exercise}
\begin{enumerate}
% ko80, no. 74
\item Let $k, n \in \Z$, $k, n > 0$. If $k \varphi(n) = n - 1$ with $k \geqslant 2$ has integer solutions in $n$, then $n$ is the product of at least four different odd primes.
% ko80, no. 75
\item Let $n \in \Z$ with $n > 0$. If $\varphi (n+3) = \varphi (n) + 2$, then either $n = 2 p^\alpha$ or $n + 3 = 2 p^\alpha$ for some prime $p$ with $p \equiv 3 \bmod 4$ and $\alpha \in \Z$, $\alpha \geqslant 1$.
% ko80, no. 80
\item Let $a, b \in \Z$, $a, b > 0$ with $\gcd (a, b) = 1$. Show there exist $m, n \in \Z$, $m, n > 0$ such that $a^m + b^n \equiv 1 \bmod ab$.
\end{enumerate}
-- End of Part~\ref{part1}.\footnote{Administrative announcement: Midterm 1 covers material up to this point.}

\part{February to March, 2017}\label{part2}

\chapter[Lecture Fourteen]{Day Fourteen \hfill {\footnotesize \rm --- 17.02.2017}}\label{chp:RSA}

\underline{Real a}pp\underline{lication: RSA cr}yp\underline{tos}y\underline{stem}
\begin{problem}
  Alice and Bob want to send secret messages but eavesdropper Eve cannot intercept the message.
\end{problem}
\begin{proof}[Solution]
  Alice and Bob developed an elaborate code that only they share to encrypt and decrypt messages. This is called \emph{symmetric key cryptosystem}. It requires both Alice and Bob to have the encoding and decoding data.

  However this approach does not work on large scale. (e.g., banking system etc.) It requires either different schemes for every person (i.e., too much work, impractical) or to reuse scheme for different person (thus defeats its original purpose.)

  The effective solution to this problem is \emph{public key cryptosystem}: a cryptosystem where the encoding data and decoding data are different. The encoding data is called the \textbf{public key} and published to the world. The decoding key data is called the \textbf{private key} and is kept secret. The key ingredient of this approach is that it is difficult to determine the private key from the data of public key.
\end{proof}
\begin{lemma}\label{lem:rsa}
  Let $n$ be a positive, square free integer. Let $k$ be a positive integer such that $k \equiv 1 \bmod \varphi(n)$. Then $a^k \equiv a \bmod n$.
\end{lemma}
\begin{proof}
  A square free integer $n$ is an integer $n = p_1 p_2 \cdots p_r$ with $p_1, p_2, \ldots, p_r$ distinct primes for $1 \leqslant i \leqslant r$, $r \in \Z$, $r > 0$. By the Chinese remainder theorem, for each $1 \leqslant i \leqslant r$,
\[
a^k \equiv a \bmod n \iff a^k \equiv a \bmod p_i.
\]
Indeed, ($\Rightarrow$) direction is by the definition of congruence so we can reduce the modulus $n$ to its divisor $p_i$. ($\Leftarrow$) direction is actually due to the Chinese remainder theorem. For all $r$ congruences
\begin{align}
  \begin{split}\label{eqn:eqn1_day14}
    x &\equiv a \bmod p_1 \\
    x &\equiv a \bmod p_2 \\
    & \hspace{1.7mm} \vdots \\
    x &\equiv a \bmod p_r,
  \end{split}
\end{align}
there is a unique $x_1$ modulo $n = \prod_{i=1}^r p_i$ such that $x_1$ solves congruences in Equations~\eqref{eqn:eqn1_day14}. Similarly, for all $r$ congruences
\begin{align}
  \begin{split}\label{eqn:eqn2_day14}
    x &\equiv a^k \bmod p_1 \\
    x &\equiv a^k \bmod p_2 \\
    & \hspace{1.7mm} \vdots \\
    x &\equiv a^k \bmod p_r,
  \end{split}
\end{align}
there is a unique $x_2$ modulo $n = \prod_{i=1}^r p_i$ such that $x_2$ solves these congruences in Equations~\eqref{eqn:eqn2_day14}. Therefore $a^k \equiv x_2 \equiv x_1 \equiv a \bmod n$, i.e., $a^k$ and $a$ are in the same congruence class modulo $n$.

Fix some $1 \leqslant i \leqslant r$. We consider two cases.
\begin{enumerate}[label=(\roman*)]
\item If $p_i \mid a$, then $p_i \mid a^k$. So $a^k \equiv 0 \equiv a \bmod p_i$.
\item If $p_i \nmid a$, then $\gcd (a, p_i) = 1$. By Fermat's little theorem, $a^{p_i - 1} \equiv 1 \bmod p_i$.

By our assumption, $k \equiv 1 \bmod \varphi(n)$ and $n$ is square free,
\[
\varphi(n) = (p_1 - 1) (p_2 - 1) \cdots (p_r - 1),
\]
we have $k \equiv 1 \bmod (p_i - 1)$ for each $1 \leqslant i \leqslant r$. Hence $k = 1 + m(p_i - 1)$ for some $m \in \Z$. Then
\begin{align}
  \hspace{4.3cm} a^k &= a^{1 + m(p_i-1)} \\
      &\equiv a (a^{p_i - 1})^m  \bmod p_i \\
      &\equiv a (1)^m  \bmod p_i & \btfact{\text{(by Fermat's little theorem\footnotemark)}} \\
      &\equiv a \bmod p_i.
\end{align}
\end{enumerate}
% use footnote in math mode: use command pair up \footnotemark in math & \footnotetext in text
\footnotetext{What if $m <0$? (It cannot happen since $k > 0$.)}

Then apply the argument at the beginning of this proof. 
\end{proof}

The RSA cryptosystem is one of the first and still widely used public key cryptosystem.

\underline{RSA cr}yp\underline{tos}y\underline{stem set-u}p (Rivest--Shamir--Adleman, 1977; Cocks, 1973\footnote{Not declassified until 1997.}): 
\begin{itemize} 
\item Choose two large primes $p, q$ and compute $n = pq$; 
\item Compute $\varphi(n) = (p-1)(q-1)$;
\item Choose $1 \leqslant e \leqslant \varphi(n)$ with $\gcd (e, \varphi(n)) = 1$;
\item Compute the multiplicative inverse of $e$ modulo $\varphi(n)$, i.e., $1 \leqslant d \leqslant \varphi(n)$ such that $de \equiv 1 \bmod \varphi(n)$;
\item Then the \emph{public key} is $(e, n)$ and the \emph{private key} is $(d,n)$.
\end{itemize}

\chapter[Lecture Fifteen]{Day Fifteen \hfill {\footnotesize \rm --- 20.02.2017}}\label{chp:miller_rabin}

Recall the set-up of RSA cryptosystem from last time,
\begin{itemize}
\item $p$ and $q$ are two large primes; compute $n = pq$;
\item An integer $1 \leqslant e \leqslant \varphi(n)$, is coprime with $\varphi(n) = (p-1)(q-1)$;
\item Compute $1 \leqslant d \leqslant \varphi(n)$, the multiplicative inverse of $e$ modulo $\varphi(n)$ such that $de \equiv 1 \bmod \varphi(n)$;
\item $(e,n)$ is the public key and $(d, n)$ is the private key.
\end{itemize}

\underline{Encr}yp\underline{tion}: Say $m \in \Z$, $1 \leqslant m \leqslant n$, is the message. Compute the smallest positive (or nonnegative) integer $c$ in the residue class of $m^e \bmod n$. This $c$ is the coded message.

\underline{Decr}yp\underline{tion}: Since $de \equiv 1 \bmod \varphi(n)$, Lemma~\ref{lem:rsa} from last lecture gives
\begin{align}
  c^d &= m^{ed} \\
      &\equiv m \bmod n.
\end{align}
So $m$ is the smallest nonnegative integer in the residue class of $c^d$.
\begin{remark}
  Why is this secure? To compute $d$ from $e$, one needs to first compute $\varphi(n)$. For this, one also needs to factor $n$. This is difficult (as of now). 
\end{remark}
\begin{remark}
  Currently, the largest RSA successfully factored was an $n$ with 232 digits in 2009. It took hundreds of computers two years. In practice, RSA uses n with between 300 and 1200 digits (or 1024 bits and 4096 bits).
\end{remark}
\begin{remark}
  In 2012, Lenstra et al. gathered millions of public keys and used Euclidean algorithm to computed $\gcd$ of pairs of $n$'s. They successfully factored 0.2\% of the keys they gathered. They concluded people were reusing primes.
\end{remark}

\underline{Technical details}: How do we generate \emph{large primes} in the first place? In practice, we generate integers that are \emph{probable primes}, i.e., integers that are prime with a probability greater than a fixed probability (on the same order, say, a server will be hit by a tornado or a lightening, twice.)

One such algorithm is (roughly) as follows:
\begin{itemize}
\item Choose a random large odd number $b$;
\item Check if $b$ is divisible by a list of small primes. If so, start again with $b+2$; if not, continue to the next step;
\item Perform several iterations of the Miller--Rabin test. If $b$ fails one of these tests, start again with $b+2$; if not, then $b$ is a probable prime.
\end{itemize}
\begin{remark}
  The prime number theorem says that
  \[
    | \left\{ p \leqslant x : p \text{ is prime} \right\} | \sim \frac{x}{\log x}.
  \]
  So primes have density roughly ``$\frac{1}{\log x}$ around $x$.'' One expects a prime in the interval $b \leqslant x \leqslant b + \log b$. If $b$ is 300 digits, $\log b \approx 690$. We should expect about 350 iterations using the above algorithm.
\end{remark}
\underline{Miller--Rabin test}: Note if $p$ is prime, for $x \in \Z$ such that  $x^2 \equiv 1 \bmod p$, $x$ is its own multiplicative inverse if and only if $x \equiv \pm 1 \bmod p$. This is not necessarily true for composite modulus, e.g., we know $3^2 \equiv 1 \bmod 8$ but $3 \not\equiv \pm 1 \bmod 8$.

Say $p$ is an odd prime with $p - 1 = 2^km$, where $k \geqslant 1$ and $m$ odd positive integer. Take $1 \leqslant a \leqslant p - 1$, then by Fermat's little theorem,
\[
  a^{2^km} \equiv a^{p-1} \equiv 1 \bmod p.
\]
By our observation from the above quadratic congruence modulo $p$,
\begin{align}
  a^{2^km} \equiv \left( a^{2^{k-1}m} \right)^2 &\equiv 1 \hspace{2.7mm} \bmod p \implies \\
  a^{2^{k-1}m} &\equiv \pm 1 \bmod p.
\end{align}
If $a^{2^{k-1}m} \equiv 1 \bmod p$ and $k > 1$, we repeat and get 
\[
  a^{2^{k-2}m} \equiv 1 \bmod p.
\]
Continue this we obtain that either
\[
  a^m \equiv 1 \bmod p \text{ (take away all powers of 2)}
\]
or 
\[
  a^{2^rm} \equiv - 1 \bmod p \text{ for some $0 \leqslant r \leqslant k - 1$.}
\]

We can turn the above observation regarding a prime $p$ into a contrapositive argument. For any odd integer $n > 2$, writing $n - 1 = 2^{k-1}m$ with $m$ odd, $m \in \Z$, $m > 0$, if there is an integer $1 \leqslant a \leqslant n-1$ such that
\begin{align}
  a^m &\not\equiv 1 \hspace{2.7mm} \bmod n \\
&\hspace{-55mm} \text { \underline{and} }\\
  a^{2^r m} &\not\equiv -1 \bmod n \text{ \hspace{1mm}~for any $0 \leqslant r \leqslant k-1$},
\end{align}
then $n$ is not prime.
Hence the Miller--Rabin test: 
\begin{itemize}
\item Choose a random $1 \leqslant a \leqslant n-1$ and compute $a^m \bmod n$ and $a^{2^rm} \bmod n$ for each $0 \leqslant r \leqslant k - 1$; 
\item Check against the aforementioned contrapositive argument about $n$;
\item For a single Miller--Rabin test, more than $\frac{3}4$ of the choices $1 \leqslant a \leqslant n - 1$ will show $n$ is composite under Miller--Rabin test if $n$ is indeed composite.
\end{itemize}

Q\underline{uestion}: How do we compute large primes \emph{for fun}?

\begin{definition}
  A \textbf{Mersenne prime} is a prime of the form $2^p - 1$ with $p$ prime.
\end{definition}
\begin{example}
  $2^2-1 = 3$, $2^3-1 = 7$, $2^5-1 = 31$, $2^7-1 = 127$ are all Mersenne primes. However $2^{11}-1 = 23\times 89$ is not a Mersenne prime. $2^{13}-1 = 8191$ is Mersenne.  
\end{example}
The largest known prime (as of now, also Mersenne) is
\[
  2^{74,207,281} - 1
\]
found in 2016 and it has 22,338,618 digits. The discovery of this Mersenne prime was a result of GIMPS (Great Internet Mersenne Prime Search).

\chapter[Lecture Sixteen]{Day Sixteen \hfill {\footnotesize \rm --- 22.02.2017}}

Recall from last time, a Mersenne prime is a prime of the form $2^p - 1$ with $p$ prime.

\begin{conjecture}\label{conj:inf_mersenne}
  There are infinitely many Mersenne primes.
\end{conjecture}
\begin{definition}
  A \textbf{Fermat prime} is a prime of the form $2^{2^n} + 1$ with $n \in \Z$, $n \geqslant 0$.
\end{definition}
\begin{example}
  $2^{2^0} + 1 = 3$, $2^{2^1} + 1 = 5$, $2^{2^2} + 1 = 17$, $2^{2^3} + 1 = 257$, $2^{2^4} + 1 = 65537$ are all Fermat primes. But Euler found $2^{2^5} + 1 = 641 \times 6700417$. 
\end{example}
So far we do not know whether there are other Fermat primes. It is believed these are the only ones.

Also from last time, we saw that if $p$ is prime, then\footnote{Proof. ($\Rightarrow)$ If $a^2 \equiv 1 \bmod p$ then $p \mid (a+1)(a-1)$. Therefore either $p \mid a+1$ or $p \mid a - 1$ since $p$ is prime, i.e., $a \equiv \pm 1 \bmod p$. ($\Leftarrow)$ If $a \equiv \pm 1 \bmod p$, then $a \cdot a \equiv (\pm 1) \cdot (\pm 1) \bmod p$, i.e., $a^2 \equiv 1 \bmod p$.

  Alternative proof: $p$ is prime so the ring $(\Z_p, +, \times)$ is a finite field $\mathbb{F}_p$, hence it is an integral domain. In this integral domain $\Z_p$, we have $x^2 = 1 \iff (x+1)(x-1) = 0 \iff x + 1 = 0 \text{ or } x - 1 = 0$, i.e., $x = \pm 1$.}
\[
  a^2 \equiv 1 \bmod p \iff a \equiv \pm 1 \bmod p.
\]

So if $a^2 \equiv 1 \bmod p$ for $1 \leqslant a \leqslant p - 1$, then $a = 1$ or $a = p - 1$.
We also know from our discussion of multiplicative inverse that if $1 \leqslant a \leqslant p - 1$, $\gcd(a, p) = 1$, then there is a unique $1 \leqslant b \leqslant p - 1$ such that $ab \equiv 1 \bmod p$. If $a^2 \equiv 1 \bmod p$, it  means $b = a$, then we have $b = 1 \text{ or } p-1$. Consequently, we have the following lemma.

\begin{lemma}\label{lem:wilson_pf}
  Let $p$ be an odd prime, $p \geqslant 5$. For each $2 \leqslant a \leqslant p - 2$, there is a unique $2 \leqslant b \leqslant p - 2$ such that $ab \equiv 1 \bmod p$ and $b \neq a$.
\end{lemma}
\begin{proof}
  By the discussion right before we stated this lemma, it follows immediately.
\end{proof}
\begin{example}
  For a prime $p = 7$, modulo 7, we have
  \begin{align}
    1 \times 1 = 1 \hspace{1.7mm} &\equiv 1 \bmod 7 \\
    2 \times 4 = 8 \hspace{1.7mm} &\equiv 1 \bmod 7 \\
    3 \times 5 = 15 &\equiv 1 \bmod 7 \\
    6 \times 6 = 36 &\equiv 1 \bmod 7.
  \end{align}
\end{example}

\begin{theorem}[Wilson's theorem]
  Let $p$ be a prime, then $(p-1)! \equiv -1 \bmod p$.
\end{theorem}
\begin{proof}
  If $p = 2$, then $(p - 1)! = 1! \equiv -1 \bmod 2$.
  
  If $p = 3$, then $(p - 1)! = 2! \equiv -1 \bmod 3$.

  If $p \geqslant 5$, by Lemma~\ref{lem:wilson_pf}, there is a unique $2 \leqslant b \leqslant p - 2$ for each $2 \leqslant a \leqslant p - 2$ such that $ab \equiv 1 \bmod p$ with $b \neq a$. Hence the numbers $2, 3, \ldots, p - 2$ can be grouped into $\left( \frac{p-3}2 \right)$ pairs of $(a, b)$, $2 \leqslant a, b \leqslant p - 2$ such that $ab \equiv 1 \bmod p$, meaning the elements of the product $2 \cdot 3 \cdot \, \cdots \, \cdot (p - 2)$ can be reordered so that
  \begin{align}
    (p-1)! &= 1 \cdot \left( 2 \cdot 3 \cdot \, \cdots \, \cdot p-2 \right) \cdot (p - 1) \\
           &\equiv 1 \cdot (\, \underbrace{ 1 \cdot 1 \cdot \, \cdots \, \cdot 1 }_{\frac{p-3}2 {\rm ~copies}} \,) \cdot \, (p - 1) \bmod p \\
           &\equiv p-1 \bmod p \\
           &\equiv -1 \hspace{3.3mm} \bmod p.
  \end{align}

  We completed the proof.
\end{proof}

\begin{proposition}
  If $n \in \Z$, $n \geqslant 2$ such that $(n-1)! \equiv -1 \bmod n$, then $n$ is prime.
\end{proposition}
\begin{proof}
  Recall $a \equiv b \bmod n$ then $\gcd (a,n) = \gcd(b, n)$. Therefore
  \[
    (n - 1)! \equiv -1 \bmod n \implies \gcd ((n-1)!, n) = \gcd (-1, n).
  \]
  But $\gcd (-1, n) = 1$, and $d \mid (n-1)!$ for each $2 \leqslant d \leqslant n-1$, $d \in \Z$. It follows that $d \nmid n$ for $2 \leqslant d \leqslant n-1$ since $\gcd((n-1)!, n) = 1$. Then the only possible divisors of $n$ is just 1 and $n$, hence $n$ is prime.
\end{proof}

\subsection*{Exercises}

\begin{enumerate}
\item Find the residue of $50!$ modulo $47^2$. (\emph{Hint}: $\frac{50!}{47} \equiv 41 \bmod 47$.)
% ko80, no. 41
\item Let $p$ be an odd prime. Show that
  \begin{align}
    1^2 \cdot 3^2 \cdot \, \cdots \, \cdot (p-2)^2 &\equiv (-1)^{\frac{p+1}2} \bmod p, \\
    2^2 \cdot 4^2 \cdot \, \cdots \, \cdot (p-1)^2 &\equiv (-1)^{\frac{p+1}2} \bmod p.
  \end{align}
% ko80, no. 66
\item Let $a_1, a_2, \ldots, a_n$ and $b_1, b_2, \ldots, b_n$ are two complete residue systems modulo $n$. Show that
  \begin{enumerate}
  \item if $2 \mid n$, then $a_1 + b_1, a_2 + b_2, \ldots, a_n + b_n$ is not a complete residue system modulo $n$;
  \item if $n > 2$, then $a_1 b_1, a_2 b_2, \ldots, a_n b_n$ is not a complete residue system modulo $n$.
  \end{enumerate}
% ko80, no. 69
\item Show $61! + 1 \equiv 0 \bmod 71$ and $63! + 1 \equiv 0 \bmod 71$. (\emph{Hint}: First show if $(-1)^r r! \equiv 1 \bmod p$ for an odd prime $p$ and $1 \leqslant r \leqslant p - 1$, $r \in \Z$, then $(p - r - 1)! + 1 \equiv 0 \bmod p$.) 
% ko80, no.70
\item Let $p > 3$ be an odd prime satisfying
  \[
    1 + \frac{1}2 + \cdots + \frac{1}{p - 1} + \frac{1}p = \frac{a}{p^b}
  \]
  with $a,b \in \Z$, $a, b > 0$ and $\gcd (a, b) = 1$. Show that $p^3 \mid a - b$.
\end{enumerate}
\chapter[Lecture Seventeen]{Day Seventeen \hfill {\footnotesize \rm --- 27.02.2017}}

Q\underline{uadratic Residues}: When an integer is a square modulo $p$ with $p$ prime.

\begin{definition}\label{def:qr}
  Let $a, n \in \Z$ with $n > 0$ and $\gcd (a, n) = 1$. We say $a$ is a \textbf{quadratic residue modulo $n$} if $\exists~ x \in \Z$ such that
  \[
    x^2 \equiv a \bmod p.
  \]
  Otherwise we say $a$ is a \textbf{quadratic nonresidue modulo $n$}.
\end{definition}
\begin{remark}\label{rmk:qr}
  Let $a, n$ be the same as in Definition~\ref{def:qr}.
  \begin{itemize}
  \item If $a \equiv b \bmod n$, then
    \[
      \text{$a$ is a quadratic residue modulo $n$} \iff \text{$b$ is a quadratic residue modulo $n$}.
    \]
  \item If $x^2 \equiv a \bmod n$ and $\gcd (a, n) = 1$, then $\gcd (x, n) = 1$. Indeed,
    \begin{align}
      \gcd (a, n) > 1 &\iff \text{$\exists~\text{prime }p$ such that $p \mid x$ and $p \mid n$}\\
                      &\iff \text{$\exists~\text{prime }p$ such that $p \mid x^2$ and $p \mid n$} \\
                      &\iff \gcd (x^2, n) \geqslant p > 1 \\
                      &\iff \gcd (a, n) \geqslant p > 1.
    \end{align}
  \end{itemize}
\end{remark}
 
\begin{example}\label{ex:ex1_day17}
  Find all quadratic residues modulo 7.
\end{example}
\begin{proof}[Solution]
  Note that if $x^2 \equiv a \bmod 7$ and $x \equiv r \bmod 7$ then $r^2 \equiv a \bmod 7$. Indeed, $x \equiv r \bmod 7 \implies x = 7q + r$ for some unique $q, r \in \Z$, $0 \leqslant r < 7$ by division algorithm. Therefore
  \begin{align}
    x^2 &= (7q + r)^2 \\
        &\equiv r^2 \bmod 7.
  \end{align}
  We can loop over $r \in \{1, 2, \ldots, 6\}$ to compute residue classes modulo 7. Note that $r = 0$ is not included since $\gcd (r, 7) = \gcd (x, 7) = 1$ by Remark~\ref{rmk:qr}.
  \begin{align}
    1^2 &\equiv 1 \bmod 7 \\
    2^2 &\equiv 4 \bmod 7 \\
    3^2 &\equiv 2 \bmod 7 \\
    4^2 &\equiv 2 \bmod 7 \\
    5^2 &\equiv 4 \bmod 7 \\
    6^2 &\equiv 1 \bmod 7.
  \end{align}
  So we see that $a \in \Z$ with $\gcd (a, 7) = 1$ is a quadratic residue modulo 7 if and only if $a \equiv 1, 2, \text{ or } 4 \bmod 7$.
\end{proof}
\begin{remark}
  We could have seen easily that $6^2 \equiv 1 \bmod 7$ since $6^2 \equiv (-1)^2 = 1 \bmod 7$. Similarly, $5^2 \equiv (-2)^2 = 2^2 \equiv 4 \bmod 7$.
  In general, if $\gcd (a, n) = 1$, $n > 0$ and $x \in \Z$ such that $x^2 \equiv a \bmod n$, then $(n - x)^2 \equiv (-x)^2 = x^2 \equiv a \bmod n$.
\end{remark}

\begin{proposition}\label{prop:quad_res_no_soln}
  Let $p$ be an odd prime and let $a \in \Z$ with $p \nmid a$. The congruence $x^2 \equiv a \bmod p$ either has no solutions in $x \in \Z$ or has precisely two incongruent solutions modulo $p$.
\end{proposition}
\begin{proof}[Sketch of Proof]\renewcommand*{\qedsymbol}{\ensuremath{\blacksquare}}
  It suffices to show that if $x^2 \equiv a \bmod p$ has a solution $x \in \Z$, then it has \underline{exactl}y two solutions. Obviously, if $x^2 \equiv a \bmod p$ then $(-x)^2 \equiv a \bmod p$. We can show that $x \not\equiv -x \bmod p$.\footnote{Assume the contrary, $x \equiv -x \bmod p \implies p \mid 2x$. But $p$ is odd and $\gcd (x, p) = 1$, a contradiction.} So when it has solutions, it has at least two two incongruent solutions $\pm x$ modulo $p$.

  On the other hand, for any other solution $y \in \Z$ with $y^2 \equiv a \bmod p$, we have
  \begin{align}
   \hspace{4.7cm} y^2 &\equiv a \hspace{1.7mm} \bmod p \\
        &\equiv x^2 \bmod p \\
\iff    &p \mid (y^2 - x^2)  \\
\iff    &p \mid y+x \text{ or } p \mid y-x & \btfact{\text{(by $p$ is prime)}}
  \end{align}
  i.e., either $y \equiv x \bmod p$ or $y \equiv -x \bmod p$. So it has at most two solutions.
  Hence when $x^2 \equiv a \bmod p$ has solutions, it has exactly two solutions.
\end{proof}
\begin{remark}\label{rmk:qr_comp_mod}
  It fails for composite moduli, e.g, for $n = 8$, $1^2 \equiv 3^2 \equiv 5^2 \equiv 7^2 \equiv 1 \bmod 8$. For $n = 15$, $1^2 \equiv 4^2 \equiv 11^2 \equiv 14^2 \equiv 1 \bmod 15$. In both cases the quadratic congruence has four incongruent solutions. 
\end{remark}

\subsection*{Exercises}

\begin{enumerate}
\item If $5x^2 \equiv 1 \bmod p$ with $p$ prime, show 5 is a quadratic residue modulo $p$.
\end{enumerate}
\chapter[Lecture Eighteen]{Day Eighteen \hfill {\footnotesize \rm --- 01.03.2017}}

\begin{proposition}\label{prop:prop1_day18}
  Let $p$ be an odd prime. There are \textbf{exactly} $\frac{p-1}2$ quadratic residues modulo $p$ and the same amount of quadratic nonresidues modulo $p$.
\end{proposition}
\begin{proof}
  For any $x \in \{ 1, 2, \ldots, p-1 \}$, $\exists~a \in \{ 1, 2, \ldots, p-1 \}$ such that $x^2 \equiv a \bmod p$.

  On the other hand, we saw in Proposition~\ref{prop:quad_res_no_soln} if $a \in \{ 1, 2, \ldots, p-1 \}$ is a quadratic residue, then there are exactly two different $x \in \{ 1, 2, \ldots, p-1 \}$ such that $x^2 \equiv a \bmod p$. As $x$ ranges over the elements of the set $\{ 1, 2, \ldots, p-1 \}$, the congruence classes of $x^2$ take $\frac{p-1}2$ values. Thus there are $\frac{p-1}2$ quadratic residues. And there are $(p-1) - \frac{p-1}2 = \frac{p-1}2$ quadratic nonresidues.
\end{proof}
\begin{example}
  We saw that 1, 2, 4 are all incongruent quadratic residues modulo 7 and 3, 5, 6 are quadratic nonresidues modulo 7 in Example~\ref{ex:ex1_day17}.
\end{example}
\begin{example}
  Note Proposition~\ref{prop:prop1_day18} fails for composite moduli, e.g., modulo 8. Any $x \in \Z$ with $\gcd (x, 8) = 1$ satisfies $x^2 \equiv 1 \bmod 8$; 1 is a quadratic residue modulo 8. But we saw in Remark~\ref{rmk:qr_comp_mod}, 3, 5, 7 are quadratic nonresidues modulo 8.
\end{example}
\begin{definition}\label{def:legendre_sym}
  Let $p$ be an odd prime and let $a \in \Z$ with $p \nmid a$. Then the \textbf{Legendre symbol} for $p$ and $a$, denoted $\left( \frac{a}p \right)$ is defined by
  \begin{align}
    \left( \frac{a}p \right) = \left\{ \begin{array}{rl}
                                         1 & \text{if $a$ is a quadratic residue mod $p$,} \\[2mm]
                                         -1 & \text{if $a$ is a quadratic residue mod $p$.}
                                       \end{array} \right.
  \end{align}                                               
\end{definition}
\begin{example}
  We rephrase Example~\ref{ex:ex1_day17} in the language of Legendre symbol modulo $7$.
  \begin{align}
    \left( \frac{1}7 \right) &= \left( \frac{2}7 \right) = \left( \frac{4}7 \right) = 1\\
    \left( \frac{3}7 \right) &= \left( \frac{5}7 \right) = \left( \frac{5}7 \right) = -1.
    \end{align}
\end{example}
\begin{remark}
  The Legendre symbol gives no information about computing the solutions to $x^2 \equiv a \bmod p$, only whether or not a solution exists.
\end{remark}
\begin{restatable}[Euler's criterion]{theorem}{eulercriterion}\label{thm:euler_criterion}
  Let $p$ be an odd prime and let $a \in \Z$ with $p \nmid a$. Then
  \[
    \left( \frac{a}p \right) \equiv a^{\frac{p-1}2} \bmod p.
  \]
\end{restatable}
\begin{proof}
  First assume $\left( \frac{a}p \right) = 1$. By Definition~\ref{def:legendre_sym}, there is an $x \in \Z$ such that $x^2 \equiv a \bmod p$ and $p \nmid x$. Then
  \begin{align}
  \hspace{5.5cm}  a^{\frac{p-1}2} &\equiv \left( x^2 \right)^{\frac{p-1}2} \bmod p \\
                    &\equiv x^{p-1} \hspace{5mm} \bmod p \\
                    &\equiv 1 \hspace{10.5mm} \bmod p. & \btfact{\text{(by Fermat's little theorem)}}
  \end{align}
  So $\left( \frac{a}p \right) \equiv a^{\frac{p-1}2} \bmod p$ when $\left( \frac{a}p \right) = 1$.

  Now assume $\left( \frac{a}p \right) = -1$. For any $b \in \{1, 2, \ldots, p-1\}$, we have $\gcd (b, p) = 1$. There is a unique $c \in \{1, 2, \ldots, p-1\}$ such that $bc \equiv a \bmod p$. Further by $\left( \frac{a}p \right) = -1$, $a$ is a quadratic nonresidue modulo $p$, hence for each $b$, the associated $c$ cannot be the same as $b$. We can pair up the $p - 1$ elements $1, 2, \ldots, p-1$ into $\left( \frac{p-1} 2\right)$ pairs $(b, c)$, $\forall~b, c \in \{1,2,\ldots,p-1\}$ such that $bc \equiv a \bmod p$. Then
  \begin{align}
    (p-1)! &\equiv \underbrace{a \cdot a \cdot \, \cdots \, \cdot a}_{\frac{p-1}2 {\rm ~copies}}\, \bmod \, p \\
           &\equiv a^{\frac{p-1}2} \bmod p.
  \end{align}
  At the same time, by Wilson's theorem, $(p-1)! \equiv -1 \bmod p$. So we have $\left( \frac{a}p \right) = -1 \equiv a^{\frac{p-1}2} \bmod p$.
\end{proof}
\begin{theorem}\label{thm:qr_1st_supp}
  Let $p$ be an odd prime. Then
  \begin{align}
    \left( \frac{-1}p \right) = (-1)^{\frac{p-1}2} = \left\{ \begin{array}{rl}
                                                               1 & \text{if $p \equiv 1 \bmod 4$,} \\[2mm]
                                                               -1 & \text{if $p \equiv 3 \bmod 4$.}
                                                             \end{array} \right.
  \end{align}
\end{theorem}
\begin{proof}
  By Euler's criterion,
  \[
    \left( \frac{-1}p \right) \equiv (-1)^{\frac{p-1}2} \bmod p.
  \]
  Since both sides are $\pm 1$ and $1 \not\equiv -1 \bmod p$ ($p$ is an odd prime; otherwise $p \mid 2$), we have
  \[
    \left( \frac{-1}p \right) = (-1)^{\frac{p-1}2}.
  \]
\end{proof}
\begin{remark}
  Without Euler's criterion, it is easy to see
  \[
    \text{quadratic residue $\times$ quadratic residue = quadratic residue,}
  \]
  not too hard to get 
  \[
    \text{quadratic residue $\times$ quadratic nonresidue = quadratic nonresidue.}
  \]
  However it is not obvious that
  \[
    \hspace{-11mm} \text{quadratic nonresidue $\times$ quadratic nonresidue = quadratic residue.}
  \]
\end{remark}

\chapter[Lecture Nineteen]{Day Nineteen \hfill {\footnotesize \rm --- 03.03.2017}}

\begin{proposition}\label{prop:legendre_sym_prod}
  Let $p$ be an odd prime and let $a, b \in \Z$ with $p \nmid a$ and $p \nmid b$. Then
  \[
    \left( \frac{ab}p \right) = \left( \frac{a}p \right) \left( \frac{b}p \right).
  \]
\end{proposition}
\begin{proof}
  Since $p \nmid a$ and $p \nmid b$, we have $p \nmid ab$. By Euler's criterion,
  \begin{align}
    \left( \frac{ab}p \right) &\equiv \left( ab \right)^{\frac{p-1}2} \hspace{ 4.6mm}\bmod p \\
                              &\equiv a^{\frac{p-1}2} b^{\frac{p-1}2} \hspace{2mm}\bmod p \\
                              &\equiv \left( \frac{a} p\right) \left( \frac{a} p\right) \bmod p.
  \end{align}
  Since $\left( \frac{ab}p \right)$, $\left( \frac{a}p \right)$ and $\left( \frac{b}p \right)$ are all $\pm 1$, $p$ is an odd prime, we have
  \[
    \left( \frac{ab}p \right) = \left( \frac{a}p \right) \left( \frac{b}p \right).
  \]
\end{proof}
\begin{theorem}\label{thm:qr_2nd_supp}
  Let $p$ be an odd prime. Then
  \begin{align}
    \left( \frac{2}p \right) =  \left\{ \begin{array}{rl}
                                                               1 & \text{if $p \equiv 1 \text { or } 7 \bmod 8$,} \\[2mm]
                                                               -1 & \text{if $p \equiv 3 \text { or } 5 \bmod 8$.}
                                                             \end{array} \right.
  \end{align}
\end{theorem}
\begin{proof}
  Let $s = \frac{p-1}2$ and consider the equalities
  \begin{align}\label{eqn:eqn1_day19}
    \begin{split}
      1 &= (-1)(-1) \\
      2 &= (+2)(-1)^2 \\
      3 &= (-3)(-1)^3 \\
      4 &= (+4)(-1)^4 \\
      & \hspace{2mm}\vdots \\
      s &= (\pm s)(-1)^s.
    \end{split}
  \end{align}
  The plus or minus sign in the last term of Equations~\eqref{eqn:eqn1_day19} depends on whether $s$ is even or odd. Multiply the left hand sides and right hand sides of Equations~\eqref{eqn:eqn1_day19} respectively, we have
  \[
    s! = \Big( (-1)(+2)(-3)\cdots(\pm s) \Big) (-1)^{1+2+\cdots +s},
  \]
  where the positive elements in the product $(-1)(+2)(-3)\cdots(\pm s)$ are all the even integers between 1 and $s = \frac{p-1}2$ while the negative elements are negations of odd integers between 1 and $s = \frac{p-1}2$.

  Notice that if $1 \leqslant a \leqslant \frac{p-1}2$ is odd then $p - a$ is even, and $p - a \equiv -a \bmod p$ for $\frac{p+1}2 \leqslant p - a \leqslant p - 1$. Moreover,
  \[
    \left\{p-a : 1 \leqslant a \leqslant \frac{p-1}2, \text{ $a$ is odd} \right\} = \left\{ b : \frac{p+1}2 \leqslant b \leqslant p-1, \text{ $b$ is even} \right\},
  \]
  where $a, b$ of the two sets are symmetric with respect to the midpoint number $\frac{p}2 \in \R$.
  Thus the product
  \begin{align}
    (-1)(+2)(-3)\cdots(\pm s) &\equiv 2 \cdot 4 \cdot \, \cdots \, \cdot (p-3)(p-1) \bmod p \\
                              &\equiv (2\cdot 1)(2\cdot 2) \cdots (2\cdot (s-1)) (2\cdot s) \bmod p \\
                              &\equiv 2^s s! \bmod p.
  \end{align}
  Next $1+2+\cdots +s = \frac{s(s+1)}2$. Therefore we have
  \[
    s! \equiv 2^s s! (-1)^{\frac{s(s+1)}2} \bmod p.
  \]
  Since $p \nmid s!$, we can cancel out (by Proposition~\ref{prop:cong_cancel}) $s!$ on both sides to get
  \begin{align}
    1 \equiv 2^s (-1)^{\frac{s(s+1)}2} &\bmod p \implies \\
    2^s \equiv (-1)^{\frac{s(s+1)}2} &\bmod p. \footnotemark
  \end{align}\footnotetext{Multiply both sides by $\frac{s(s+1)}2$ and note that $s(s+1)$ is even. This trick will be used again so we can move one term to the other side when we consider quadratic reciprocity law.}
  But $s = \frac{p-1}2$ and $\gcd (2, p) = 1$, so by Euler's criterion,
  \begin{align}
    2^s &= 2^{\frac{p-1}2} \\
        &\equiv \left( \frac{2}p \right) \bmod p.
  \end{align}
  Hence
  \[
    \left( \frac{2}p \right) \equiv (-1)^{\frac{s(s+1)}2}.
  \]
  The equality follows from that they are $\pm 1$ on both sides of the above congruence and $p$ is an odd prime.

  Finally,
  \begin{align}
    (-1)^{\frac{s(s+1)}2} & = \left\{ \begin{array}{rl}
                                      1 & \text{if $s = 4k$ \hspace{6mm}or $s = 4k+3$ for some $k \in \Z$,} \\[2mm]
                                        -1 & \text{if $s = 4k+1$ or $s = 4k+2$ for some $k \in \Z$.}
                                      \end{array} \right. \\
    & = \left\{ \begin{array}{rl}
                                      1 & \text{if $p = 8k+1$ or $p = 8k+7$ for some $k \in \Z$,} \\[2mm]
                                        -1 & \text{if $p = 8k+3$ or $p = 8k+5$ for some $k \in \Z$.}
                \end{array} \right.
  \end{align}
\end{proof}
\begin{example}
  Consider the prime 257. It satisfies
  \begin{align}
    257 &= 240 + 17 \\
        &\equiv 17 \bmod 8 \\
        &\equiv 1 \hspace{1.7mm}\bmod 8.
  \end{align}
  Thus $\left( \frac{2}p \right) = 1$, i.e., 2 is a quadratic residue modulo 257.
\end{example}

Q\underline{uestion}: For which primes is $(-2)$ a quadratic residue?

Let $p$ be an odd prime, by Proposition~\ref{prop:legendre_sym_prod},
\[
  \left( \frac{-2}p \right) =   \left( \frac{-1}p \right)   \left( \frac{2}p \right) = \left\{ \begin{array}{rl}
                                                               1 & \text{if $p \equiv 1 \text { or } 3 \bmod 8$,} \\[2mm]
                                                               -1 & \text{if $p \equiv 5 \text { or } 7 \bmod 8$.}
                                                                                               \end{array} \right.
\]
\begin{remark}
  Theorem~\ref{thm:qr_1st_supp} and Theorem~\ref{thm:qr_2nd_supp} are called the \textbf{first supplement} and the \textbf{second supplement to quadratic reciprocity} respectively.
\end{remark}

\chapter[Lecture Twenty]{Day Twenty \hfill {\footnotesize \rm --- 06.03.2017}}

Q\underline{uadratic reci}p\underline{rocit}y (QR law)

\begin{theorem}\label{thm:qr}
  Let $p, q$ be odd primes. Then
  \[
    \left( \frac{p}q \right) \left( \frac{q}p \right) = (-1)^{\frac{p-1}2 \cdot \frac{q-1}2}.
  \]
\end{theorem}
The proof itself takes more than a single lecture and we shall see some numerical evidences before the proof is given.
\begin{example}
  Is $5$ quadratic residue modulo $77797$?
\end{example}
\begin{proof}[Solution]
  No. Note that $77797$ is an odd prime as is $5$. By quadratic reciprocity (Theorem~\ref{thm:qr}),
  \[
    \left( \frac{5}{77797} \right) \left( \frac{77797}{5} \right) = (-1)^{\frac{5-1}2\times \frac{77797-1}2} = 1.
  \]
  We do not need to compute $\frac{77797-1}2$ since $\frac{5-1}2$ is even.

  Therefore
  \begin{align}
  \hspace{4.5cm}  \left( \frac{5}{77797} \right) &= \left( \frac{77797}{5} \right) \\
                                   &= \left( \frac{2}5 \right ) &\btfact{\text{(by $77797\equiv 2 \bmod 5$)}}\\
                                   &= -1.
  \end{align}
  Hence $5$ is not a quadratic residue modulo $77797$.
\end{proof}
\begin{example}
  Determine whether or not $(-30)$ is a quadratic residue modulo $257$.
\end{example}
\begin{proof}[Solution]
  $257$ is an odd prime. We can break $\left( \frac{-30}{257} \right)$ into the product of several Legendre symbols by Proposition~\ref{prop:legendre_sym_prod},
  \[
    \left( \frac{-30}{257} \right) = \left( \frac{-1}{257} \right) \left( \frac{2}{257} \right)\left( \frac{3}{257} \right)\left( \frac{5}{257} \right).
  \]

  By the first supplement to quadratic reciprocity (Theorem~\ref{thm:qr_1st_supp}), $\left( \frac{-1}{257} \right) = 1$ since $257 \equiv 1 \bmod 4$. By the second supplement to quadratic reciprocity (Theorem~\ref{thm:qr_2nd_supp}), $\left( \frac{2}{257} \right) = 1$ since $257 \equiv 1 \bmod 8$. We also know\footnote{These are based on exercise problems in~\cite{Strayer2001}*{\S 4.3}. See Problems 35, 36.} $\left( \frac{3}{257} \right) = -1$ since $257 \equiv 5 \bmod 12$, not $\pm 1 \bmod 12$; $\left( \frac{5}{257} \right) = -1$ since $257 \equiv 2 \bmod 5$, not $1 \text{ or } 4 \bmod 5$.

  Therefore
  \[
    \left( \frac{-30}{257} \right) = 1 \times 1 \times (-1) \times (-1) = 1,
  \]
  i.e., $-30$ is a quadratic residue modulo $257$.
\end{proof}
\begin{example}
  Determine whether or not $193$ is a quadratic residue modulo $293$.
\end{example}
\begin{proof}[Solution]
  $193$ and $273$ are both odd primes. By quadratic reciprocity,
  \[
    \left( \frac{193}{273} \right) \left( \frac{273}{193} \right) = (-1)^{\frac{193-1}2 \times \frac{273-1}2} = 1.
  \]

  And 
  \[
  \left( \frac{273}{193} \right) = \left( \frac{273 - 193}{193} \right) = \left( \frac{80}{193} \right) = \left( \frac{2^4\times 5}{193} \right) = \left( \frac{5}{193} \right) = -1.\footnotemark
  \]
\footnotetext{Again by quadratic reciprocity, $\left( \frac{5}{193} \right) \left( \frac{193}5 \right) = (-1)^{\frac{5-1}2 \times \frac{193-1}2} = 1 \implies \left( \frac{5}{193} \right) =  \frac{1}{\left( \frac{193}5 \right)} = \frac{1}{-1} = -1$ since $\left( \frac{193}5 \right) = \left( \frac{3}5 \right) = -1$.} 

Therefore $\left( \frac{193}{273} \right) = \frac{1}{-1} = -1$. Hence $193$ is not a quadratic residue modulo $273$.
\end{proof}

The proof of quadratic reciprocity presented here is due to G. Rousseau~\cite{Rousseau1991}, however treated in such an elementary way that the language of abstract algebra is not used. This proof is also different from the lattice points counting proof used in~\cite{Strayer2001}. The rough idea of this proof is to consider multiplying together ``half'' of the elements of $1 \leqslant a \leqslant pq$ with $\gcd (a, pq) = 1$. In three ways, using Chinese remainder theorem, we show that these three different ways are equal modulo $pq$ up to a sign. We then show these differences in signs are $\left( \frac{p}q \right)$, $\left( \frac{q}p \right)$ and $(-1)^{\frac{p-1}2 \cdot \frac{q-1}2}$ using Euler's criterion. 

\begin{notation}
  Let $\sA$ be some finite subset of $\Z$. We use $\prod_{a \in \sA} a$ to denote the product of all elements in $\sA$, e.g, $n! = \prod_{1 \leqslant j \leqslant n} j$.
\end{notation}



\begin{proof}[Proof {\rm (of Theorem~\ref{thm:qr}, quadratic reciprocity)}]\renewcommand*{\qedsymbol}{}
Let $\sA = \left\{ a \in \Z : 1 \leqslant a < \frac{pq}2, \, \gcd (a, pq) = 1 \right\}$. Let $R = \prod_{a \in \sA} a$. We want to evaluate $R \bmod p$ and $R \bmod q$. $R$ can be written explicitly as
\begin{align}\label{eqn:qr_R}
  \begin{split}
    R &= \left(\frac{1}{ 1 \cdot q \cdot \, \cdots \, \cdot \left(\frac{p-1}2\right) q } \right) \big(\overbrace{ 1 \cdot 2 \cdot \, \cdots \, \cdot (p-1) }^{(p-1) {\rm ~terms}} \big)
    \big( \overbrace{(p+1) \cdot (p+2) \cdot \, \cdots \, \cdot (p + (p-1)) }^{(p-1) {\rm ~terms}} \big)
    \cdots \\
    & \hspace{7.8mm} \Bigg( \overbrace{ \left(p\left(\frac{q-1}2\right)+1\right) \cdot \left(p\left(\frac{q-1}2\right)+2\right) \cdot \, \cdots \, \cdot \left(p\left(\frac{q-1}2\right) + \frac{p-1}2\right) }^{{\rm only~}\frac{p-1}2 {\rm ~terms}} \Bigg) \\
    &=\frac{\displaystyle \left(\prod_{i=1}^{p-1} i \right) \left(\prod_{i=1}^{p-1} p+i\right) \cdots \left(\prod_{i=1}^{p-1} p\left( \frac{q-3}2 \right) + i\right) \left( \prod_{i=1}^{\frac{p-1}2} p\left( \frac{q-1}2 \right) + i \right)}{\displaystyle \prod_{i=1}^{\frac{p-1}2} iq} \\
    &=\frac{\displaystyle \left( \prod_{j=0}^{\frac{q-3}2} \prod_{i=1}^{p-1} pj + i \right) \left( \prod_{i=1}^{\frac{p-1}2} p\left( \frac{q-1}2 \right) + i \right) }{\displaystyle \prod_{i=1}^{\frac{p-1}2} i} . 
  \end{split}
\end{align}
Here we have written in the numerator the product of all integers between $1$ and $\frac{pq-1}2$ that are coprime with $p$, and in the denominator all multiples of $q$ in the same range.\footnote{As in $ \prod_{i=1}^{\frac{p-1}2} p\left( \frac{q-1}2 \right) + i $ of Equation~\eqref{eqn:qr_R}, we needed to calculate $m, n$ in $\left( \prod_{i=1}^m pn + i \right) $ in order to write down the correct last term in the numerator. To get $n = \frac{q-1}2$, consider the largest $n \in \Z$ such that $np \leqslant \frac{pq-1}2$. It implies $n \leqslant \frac{pq-1}{2p} = \frac{q}2 - \frac{1}{2p} = \lfloor \frac{q}2 \rfloor + \frac{1}2 - \frac{1}{2p}$ since $q$ is odd. And $p > 1$, hence $\frac{1}2 - \frac{1}{2p} \in \left(0, \frac{1}2\right)$. Therefore $n = \lfloor \frac{q}2 \rfloor = \frac{q-1}2$ as desired. As for $m$, consider the largest $m \in \Z$ such that $p\left( \frac{q-1}2 \right) + m < \frac{pq}2$. Then $m < \frac{p}2$. Hence $m = \lfloor \frac{p}2 \rfloor = \frac{p-1}2 $.} Denote the numerator and the denominator $N$ and $D$ respectively in Equation~\eqref{eqn:qr_R}. We have $DR = N$. (To be continued.)
\end{proof}

\chapter[Lecture Twenty-One]{Day Twenty-One \hfill {\footnotesize \rm --- 08.03.2017}}

\begin{proof}[Proof {\rm (of Theorem~\ref{thm:qr}, quadratic reciprocity, continued)}]
  The denominator in Equation~\eqref{eqn:qr_R}
  \begin{align}
    D &= \prod_{i=1}^{\frac{p-1}2} iq \\
      &= q^{\frac{p-1}2} \prod_{i=1}^{\frac{p-1}2} i \\
      &=\left(\frac{p-1}2\right)! \, q^{\frac{p-1}2}.
  \end{align}
  Thus $DR = N$ implies
  \begin{align}\label{eqn:dr_n}
    \begin{split}
      \left( \frac{p-1}2 \right)! \, q^{\frac{p-1}2} R &\equiv \left( \prod_{j=0}^{\frac{q-3}2} \prod_{i=1}^{p-1} i \right) \left( \prod_{i=1}^{\frac{p-1}2}  i \right) \bmod p \\
      &\equiv \left( \prod_{i=1}^{p-1} i \right)^{\frac{q-1}2} \left( \prod_{i=1}^{\frac{p-1}2}  i \right) \hspace{1mm} \bmod p \\
      &\equiv (p-1)!^{\frac{q-1}2}\left( \frac{p-1}2 \right)! \hspace{.5mm}\bmod p.
    \end{split}
  \end{align}
  Since $p \nmid \left( \frac{p-1}2 \right)!$, we can cancel $\left( \frac{p-1}2 \right)!$ from both sides of Equation~\eqref{eqn:dr_n} . Further by Euler's criterion, we have $q^{\frac{p-1}2} \equiv \left( \frac{q}p \right) \bmod p$. Then Equation~\eqref{eqn:dr_n} becomes
  \[
    \left( \frac{q}p \right) R \equiv (p-1)!^{\frac{q-1}2} \bmod p.
  \]
  By a symmetric argument, switching the roles of $p$ and $q$, we get
  \[
    \left( \frac{p}q \right) R \equiv (q-1)!^{\frac{p-1}2} \bmod q.
  \]

  Now let $B$ be the set of $1 \leqslant b \leqslant pq$ with $\gcd (b, pq) = 1$ such that 
\begin{align}
b &\equiv i \bmod p \text{ for some } 1 \leqslant i \leqslant p-1, \text{ and } \\
b &\equiv j \bmod q \text{ for some } 1 \leqslant j \leqslant \frac{q-1}2.
\end{align}
By Chinese remainder theorem, there is a unique $1 \leqslant b \leqslant pq-1$ coprime with $pq$ for every such pair $(i,j)$. Set $Q = \prod_{b \in \sB} b$, where 
\[
\sB = \left\{ 1 \leqslant b \leqslant pq-1 : b \equiv i \bmod p \text{ for $1 \leqslant i \leqslant p-1$} \text{ and } b \equiv j \bmod q \text{ for $1 \leqslant j \leqslant \frac{q-1}2 $} \right\}.
\]
Thus
\begin{align}\label{eqn:Q_mod_p}
\begin{split}
  Q &= \prod_{b \in \sB} b \\
    &\equiv \prod_{i=1}^{p-1} i^{\frac{q-1}2} \hspace{4.5mm}\bmod p\\
    &\equiv (p-1)!^{\frac{q-1}2} \bmod p,
\end{split}
\end{align}
and
\begin{align}\label{eqn:Q_mod_q}
  Q &\equiv \prod_{j=1}^{\frac{q-1}2} j^{p-1} \hspace{13.3mm}\bmod q\\
    &\equiv \Bigg(\left(\frac{q-1}2\right)!\Bigg)^{p-1} \bmod q.
\end{align}
\begin{claim}\label{clm:clm1_qr}
\[
\left( \frac{q}p \right) R \equiv Q \bmod pq.
\]
\end{claim}
\begin{proof}[Proof {\rm (of Claim~\ref{clm:clm1_qr})}]
For any $1 \leqslant a \leqslant \frac{pq-1}2$ with $\gcd (a, pq) = 1$, precisely one of following two must happen: 
\begin{enumerate}[label=(\roman*)]
\item $a \equiv i \bmod p$ for some $1 \leqslant i \leqslant p - 1$ and 
 $a \equiv j \bmod q$ for some $1 \leqslant j \leqslant \frac{q - 1}2$; \label{itm:itm1_clm1_qr}
\item $a \equiv i \bmod p$ for some $1 \leqslant i \leqslant p - 1$ and 
 $a \equiv j \bmod q$ for some $\frac{q+1}2 \leqslant j \leqslant q-1$. \label{itm:itm2_clm1_qr}
\end{enumerate}
The second case ~\ref{itm:itm2_clm1_qr} is equivalent to 
\begin{enumerate}[label=(\roman*)]
% prime item 
% https://tex.stackexchange.com/questions/320667/enumitem-how-to-put-a-prime-into-certain-parenthesized-labels
\item[(ii')]\def\@currentlabel{(ii')} $a \equiv -i \bmod p$ for some $1 \leqslant i \leqslant p - 1$ and 
 $a \equiv -j \bmod q$ for some $1 \leqslant j \leqslant \frac{q - 1}2$. \label{itm:itm2a_clm1_qr}
\end{enumerate}
In the first case ~\ref{itm:itm1_clm1_qr}, $a \equiv b \bmod pq$ for a unique $b \in \sB$ and in the second case ~\ref{itm:itm2a_clm1_qr}, $a \equiv -b \bmod pq$ for a unique $b \in \sB$ by Chinese remainder theorem. Thus
\begin{align}
  R &= \prod_{a \in \sA} a \\
    &\equiv \pm \prod_{b \in \sB} b \bmod pq \\
    &\equiv \pm Q \hspace{5.4mm} \bmod pq.
\end{align}
But we knew already $\left( \frac{q}p \right) R \equiv (p-1)!^{\frac{q-1}2} \equiv Q \bmod p$. Further by $p$ is an odd prime, this implies\footnote{First $R \equiv \pm Q \bmod pq \implies R \equiv \pm Q \bmod p$ since we can reduce the modulus. Then together with $\left( \frac{q}p \right)R \equiv Q \bmod p$, we have $\left( \frac{q}p \right) R \equiv \pm R \bmod p$. The fact that $p$ is an odd prime will force the congruence to become equality given that $\left( \frac{q}p  \right)$ can only be $\pm 1$.} the sign $\pm$ equals $\left( \frac{q}p \right)$. Hence the Claim~\ref{clm:clm1_qr}.
\end{proof}

Now let $\sC$ be the set of $1 \leqslant c \leqslant pq - 1$ with $\gcd (c, pq) = 1$ such that 
\begin{align}
c &\equiv i \bmod p \text{ for some } 1 \leqslant i \leqslant \frac{p-1}2, \text{ and } \\
c &\equiv j \bmod q \text{ for some } 1 \leqslant j \leqslant q-1.
\end{align}
Set $P = \prod_{c \in \sC} c$, where
\[
\sC = \left\{ 1 \leqslant c \leqslant pq-1 : c \equiv i \bmod p \text{ for $1 \leqslant i \leqslant \frac{p-1}2$} \text{ and } c \equiv j \bmod q \text{ for $1 \leqslant j \leqslant q-1 $} \right\}.
\]
By a similar argument to what we have done to $\sB$ and $Q = \prod_{b \in \sB} b$, we get 
\begin{align}
P &\equiv \Bigg(\left( \frac{p-1}2 \right)!\Bigg)^{q-1} \bmod p, \text{ and }\label{eqn:P_mod_p}\\
P &\equiv (q-1)!^{\frac{p-1}2} \hspace{9.5mm}\bmod q. \label{eqn:P_mod_q}
\end{align}
Repeat the argument in the proof of Claim~\ref{clm:clm1_qr} with switched roles of $p$ and $q$ for $P$, we can claim
\begin{claim}\label{clm:clm2_qr}
\[
\left( \frac{p}q \right) R \equiv P \bmod pq.
\]
\end{claim}

\begin{claim}\label{clm:clm3_qr}
\[
Q \equiv (-1)^{\frac{p-1}2 \cdot \frac{q-1}2} P \bmod pq.
\]
\end{claim}
\begin{proof}[Proof {\rm (of Claim~\ref{clm:clm3_qr})}]
It suffices to show the congruence holds modulo $p$ \underline{and} modulo $q$.

First off,
\begin{align}
(p-1)! &= 1 \cdot 2 \cdot \, \cdots\, \cdot \left(\frac{p-1}2 \right) \left(\frac{p-1}2 + 1 \right) \cdot \, \cdots\, \cdot (p-2) (p-1) \\
       &= 1 \cdot 2 \cdot \, \cdots\, \cdot \left(\frac{p-1}2 \right) \left(p - \frac{p-1}2 \right) \cdot \, \cdots\, \cdot (p-2) (p-1) \\
       &\equiv \Bigg( \left(\frac{p-1}2 \right)!\Bigg)^2 (-1)^{\frac{p-1}2} \bmod p.
\end{align}
Plug this in to Equation~\eqref{eqn:Q_mod_p} and raise it to the $\left( \frac{q-1}2 \right)$-th power, further we get 
\begin{align}
\hspace{2.2cm} Q &\equiv \Bigg( \left(\frac{p-1}2 \right)!\Bigg)^{q-1} (-1)^{\frac{p-1}2\cdot \frac{q-1}2} \bmod p \\
    &\equiv P (-1)^{\frac{p-1}2\cdot \frac{q-1}2} \bmod p. & \btfact{\text{(by Equation~\eqref{eqn:P_mod_p})}}
\end{align}
Similarly,
\[
  (q-1)! \equiv \Bigg( \left(\frac{q-1}2 \right)!\Bigg)^2 (-1)^{\frac{q-1}2} \bmod q.
\]
Plug it into Equation~\eqref{eqn:Q_mod_q} and use Equation~\eqref{eqn:P_mod_q}, then 
\begin{align}
  P &\equiv (-1)^{\frac{q-1}2 \cdot \frac{p-1}2} Q \bmod q \iff \\
  Q &\equiv (-1)^{\frac{q-1}2 \cdot \frac{p-1}2} P \bmod q.
\end{align}
This proves Claim~\ref{clm:clm3_qr}.
\end{proof}
Finally, putting Claims~\ref{clm:clm1_qr}, \ref{clm:clm2_qr} and \ref{clm:clm3_qr} together, we have
\begin{align}
\hspace{3.6cm}  \left( \frac{q}p \right) R &\equiv Q \hspace{27.3mm}\bmod pq & \btfact{\text{(by Claim~\ref{clm:clm1_qr})}}\\
                             &\equiv (-1)^{\frac{p-1}2 \cdot \frac{q-1}2} P \hspace{8.7mm}\bmod pq & \btfact{\text{(by Claim~\ref{clm:clm3_qr})}}\\
  &\equiv (-1)^{\frac{p-1}2 \cdot \frac{q-1}2} \left( \frac{p}q \right) R \bmod pq & \btfact{\text{(by Claim~\ref{clm:clm2_qr})}}
\end{align}
Since $\gcd (R, pq) = 1$, we can cancel $R$ on both sides to get
\[
  \left( \frac{q}p \right) \equiv (-1)^{\frac{p-1}2 \cdot \frac{q-1}2} \left( \frac{p}q \right) \bmod pq.
\]
All terms in the above congruence are $\pm 1$ and $pq$ is odd and $pq > 2$, therefore this forces the congruence to be an equality. We completed the proof of quadratic reciprocity. Phew!\footnote{The fact that the proof of quadratic reciprocity without the vocabulary of modern algebra ran page after page because of our elementary exposition, indirectly says how powerful the language of abstract algebra is.}
\end{proof}
\subsection*{Exercise}
\begin{enumerate}
% ko80, no. 89
\item Let $n \in \Z$, $n > 1$. Show that $2^n - 1 \nmid 3^n - 1$.
% ko80, no. 97
\item Let $p$ be a prime with $p \neq 2, 3, 5, 11, 17$. Show there exist three different quadratic residues modulo $p$, denoted $r_1, r_2, r_3$, such that $r_1 + r_2 + r_3 \equiv 0 \bmod p$.
% ko80, no. 98  
\item Let $p = 2k + 1$ be a prime, where $k \in \Z$, $k > 0$. If $p \equiv 7 \bmod 8$, then $\displaystyle \sum_{i = 1}^k i \left( \frac{i} p \right) = 0$, where $\displaystyle \left( \frac{i}p \right)$ is the usual Legendre symbol. (\emph{Hint}: Try computing $\displaystyle \sum_{i = 1}^{p-1} i \left( \frac{i}p \right)$ in two different ways.)
\end{enumerate}
\chapter[Lecture Twenty-Two]{Day Twenty-Two \hfill {\footnotesize \rm --- 10.03.2017}}\label{lec22}

\underline{Arithmetic functions}

\begin{definition}
  An \textbf{arithmetic function} is a function valued in $\Z$, $\Q$, $\R$ or $\C$, where the domain of the function is the set of positive integers $\Z_{\geqslant 1}$.
\end{definition}
\begin{example}
  The Euler $\varphi$-function $\varphi(n)$, counts the number of integers $1 \leqslant a \leqslant n$ such that $\gcd (a, n) = 1$.
\end{example}
\begin{example}\label{ex:nu_func}
 The number of divisors function $\nu(n)$, counts the number of p\underline{ositive} divisors of $n$, i.e., 
\[
  \nu (n) = \sum_{\substack{d \, \mid \, n \\ d \, > \, 0}} 1
\]for $n \in \Z$, $n > 0$. For example, $\nu(p) = 2$ for any prime $p$. It is also easy to check $\nu(6) = 4$.
\end{example}
\begin{example}\label{ex:sigma_func}
  The sum of divisors function $\sigma(n)$, sums up all p\underline{ositive} divisors of $n$, i.e., \[
  \sigma (n) = \sum_{\substack{d \, \mid \, n \\ d \, > \, 0}} d
\]
for $n \in \Z$, $n > 0$. For example, $\sigma (p) = 1 + p$ for any prime $p$. Another example $\sigma(6) = 1 + 2 + 3 + 6 = 12$.
\end{example}

\begin{definition}
  An arithmetic function $f$ is \textbf{multiplicative} if $f(mn) = f(m)f(n)$ for $m,n \in \Z$, $m,n > 0$ when $\gcd (m, n) = 1$. $f$ is \textbf{completely multiplicative} if $f(mn) = f(m)f(n)$ for all $m,n \in \Z$, $m,n > 0$.
\end{definition}

\begin{example}
  $\varphi(n)$ is multiplicative (See Proposition~\ref{prop:phi_multi}) but it is not completely multiplicative, e.g., $\varphi(4) = 2 \neq 1 = \varphi(2) \varphi(2)$.
\end{example}
\begin{example}
  Fix a prime $p$. Define an arithmetic function $\chi (n)$ by
\begin{align}
\chi (n) = \left\{  \begin{array}{cl} 
                      0 & \text{if $p \mid n$,} \\[1mm]
                      \left(\frac{n}p \right) & \text{if $p \nmid n$.} 
                    \end{array} \right.
\end{align}
for $n \in \Z$, $n > 0$. This function is called \textbf{Dirichlet character}.   Show the Dirichlet character $\chi (n)$ is completely multiplicative. 
\end{example}
\begin{proof}
  If a prime $p \mid mn$ for $m,n \in \Z$, $m,n > 0$, then $p \mid m$ or $p \mid n$. $\chi (mn) = 0$ since $p \mid mn$. On the other hand, $\chi (m) = 0$ or $\chi (n) = 0$. Therefore $\chi (mn) = \chi (m) \chi (n)$ if $p \mid mn$.

  If $p \nmid mn$, then $p \nmid m$ and $p \nmid n$. 
\begin{align}
  \chi (mn) &= \left( \frac{mn}p \right) \\
            &= \left( \frac{m}p \right) \left( \frac{n}p \right) \\
            &= \chi (m) \chi (n).
\end{align}
The second equality is due to Proposition~\ref{prop:legendre_sym_prod} of Legendre symbol.
\end{proof}

\begin{remark}
If $f$ is a multiplicative function, then to compute it, it suffices to know its value on prime powers. If $n \in \Z$, $n > 0$, consider its prime factorization
\[
n = p_1^{e_1}p_2^{e_2} \cdots p_k^{e_k},
\]
where $p_i$'s are distinct primes, $e_i \in \Z$, $e_i \geqslant 1$ for each $1 \leqslant i \leqslant k$ with $k \in \Z$, $k > 0$. Then 
\[
  f(n) = f(p_1^{e_1}p_2^{e_2} \cdots p_k^{e_k}) 
       = f(p_1^{e_1}) f(p_2^{e_2}) \cdots f(p_k^{e_k}).
\]
\end{remark}
\begin{example}
  We saw in Lemma~\ref{lem:phi_p} that for any prime $p$ and $e \in \Z$, $e \geqslant 1$,  for the prime power $p^e$, we have
\[
\varphi (p^e) = (p-1)p^{e-1}.
\]
Hence $\varphi (p_1^{e_1}p_2^{e_2} \cdots p_k^{e_k}) = (p_1-1)p^{e_1-1} (p_2-1)p^{e_2-1} \cdots (p_k-1)p^{e_k-1}$, where $p_i$'s are distinct primes and $e_i \in \Z$, $e_i \geqslant 1$. In this way, Theorem~\ref{thm:phi_n} is immediate.
\end{example}

\begin{theorem}\label{thm:f_F_multi}
  Let $f$ be an arithmetic function. Define an arithmetic function $F$ by 
\[
F(n) = \sum_{\substack{d \, \mid \, n \\ d \, > \, 0}} f(d)
\]
for any $n \in \Z$, $n > 0$. If $f$ is multiplicative, then $F$ is also multiplicative.
\end{theorem}
We state a lemma first.
\begin{lemma}\label{lem:lem1_day22}
  Let $m, n \in \Z$, $m, n > 0$ with $\gcd (m, n) = 1$. Then for any $d \mid mn$, $d > 0$, there are \textbf{unique} $d_1 \mid m$ and $d_2 \mid n$, $d_1, d_2 > 0$ such that $d = d_1 d_2$. Moreover, if $d_1 \mid m$ and $d_2 \mid n$ with $d_1, d_2 > 0$, then $\gcd (d_1, d_2) = 1$ and $d_1d_2 \mid mn$.
\end{lemma}
\begin{proof}[Proof {\rm (of Lemma~\ref{lem:lem1_day22})}]\renewcommand*{\qedsymbol}{\ensuremath{\blacksquare}}
This is a homework question. See~\cite{Strayer2001}*{\S 3.1, Problem 8}.
\end{proof}
\begin{proof}[Proof {\rm (of Theorem~\ref{thm:f_F_multi})}]
Let $m, n \in \Z$, $m, n > 0$ with $\gcd (m, n) = 1$. Then 
\[
F(mn) = \sum_{\substack{d \, \mid \, mn \\ d \, > \, 0}} f(d).
\]
By Lemma~\ref{lem:lem1_day22}, for any $d \mid mn$, $d$ factors uniquely as $d = d_1d_2$ with $d_1 \mid m$, $d_2 \mid n$, $d_1, d_2 > 0$ and $\gcd (d_1, d_2) = 1$. Conversely, such $d_1, d_2$ pair gives a $d = d_1 d_2$ such that $d \mid mn$. 

Therefore,
\begin{align}
\hspace{4.7cm}  \sum_{\substack{d \, \mid \, mn \\ d \, > \, 0}} f(d) &= \sum_{\substack{d_1 \, \mid \, m \\ \hspace{-.8mm}d_2 \, \mid \, n \\ d_1,\,d_2 \, > \, 0}} f(d_1d_2) \\
                                                        &=\sum_{\substack{d_1 \, \mid \, m \\ \hspace{-.8mm}d_2 \, \mid \, n \\ d_1,\,d_2 \, > \, 0}} f(d_1)f(d_2) & \btfact{\text{(by multiplicativity of $f$)}}\\
                                                        &= \sum_{\substack{d_1 \, \mid \, m \\ d_1 \, > \, 0}} f(d_1) \sum_{\substack{d_2 \, \mid \, n \\ d_2 \, > \, 0}} f(d_2) \\
&= F(m)F(n).
\end{align}
\end{proof}

\begin{example}
 Consider the sum of number of divisors functions $\nu (n) = \sum_{\substack{d \, \mid \, n \\ d \, > \, 0}} 1$. 

The constant function $f(n) = 1$ is (completely) multiplicative for all $n \in \Z$, $n > 0$. Then $\nu (n) $ is also multiplicative by Theorem~\ref{thm:f_F_multi}. For any prime $p$ and $e \in \Z$, $e \geqslant 0$, $\nu (p^e) = 1+e$ since $ d \mid p^e \iff d = p^f$ for some $0 \leqslant f \leqslant e$, $f \in \Z$.

Thus if $p_1, p_2, \ldots, p_k$ are distinct primes and $e_1, e_2, \ldots, e_k$ are nonnegative integers, $k \in \Z$, $k > 0$, then
\[
\nu (p_1^{e_1}p_2^{e_2}\cdots p_k^{e_k}) = \prod_{i = 1}^k (1 + e_i).
\]
This is an application of fundamental theorem of arithmetic. 
\end{example}

\begin{example}
The sum of divisors function $\sigma (n) = \sum_{\substack{d \, \mid \, n \\ d \, > \, 0}} d$ is multiplicative since identify function $f(n) = n$ is multiplicative for all $n \in \Z$, $n > 0$. Let $p$ be a prime and let $e \in \Z$, $e \geqslant 1$. Then 
\[
\sigma (p^e) = 1 + p + p^2 + \cdots + p^e = \frac{p^{e+1} - 1}{p-1}.
\]
Note this also holds when $e = 0$. If $p_1, p_2, \ldots, p_k$ are distinct primes and $e_1, e_2, \ldots, e_k$ are nonnegative integers, $k \in \Z$, $k > 0$, then
\[
\sigma (p_1^{e_1}p_2^{e_2}\cdots p_k^{e_k}) = \prod_{i = 1}^k \frac{p_i^{e_i+1} - 1}{p_i-1}.
\]
\end{example}
\begin{remark}
  If $f$ is completely multiplicative, we do not necessarily have $F(n) = \sum_{\substack{d \, \mid \, n \\ d \, > \, 0}} f(d)$ is also completely multiplicative, e.g., $f(n) = 1$ is completely multiplicative but $\nu (n) = \sum_{\substack{d \, \mid \, n \\ d \, > \, 0}} 1$ is not since $\nu (4) = 3 \neq 4 = \nu (2) \nu (2)$. (Or use another counter example $\sigma (4) = 7 \neq 9 = \sigma (2) \sigma (2)$. Note $f(n) = n$ is completely multiplicative in $\sigma (n) = \sum_{\substack{d \, \mid \, n \\ d \, > \, 0}} d$.)
\end{remark}

\subsection*{Exercise}
\begin{enumerate}
\item Show $\nu (n) \leqslant 2\sqrt n$. (\emph{Hint}: Cut numbers $1$ through $n$ at the point $\sqrt n$ and count divisors of $n$ from both sides.)
\item Show $\varphi (n) \nu (n) \geqslant n$.
% ko80, no. 58
\item Let $n \in \Z$ with $n > 0$ be a given positive integer. Find the number of integer solutions of
  \[
    \frac{1}n = \frac{1}x + \frac{1}y \text{ with $x, y > 0$ and $x \neq y$.}
  \]
% ko80, no. 76
\item Find all positive integers $n$ such that $\nu (n) = \varphi (n)$. (\emph{Hint}: All such $n$'s are $n = 1$, $3$, $8$, $10$, $18$, $24$, and $30$.)
% ko80, no. 77  
\item Let $p, q$ be primes and let $a, b \in \Z$, $a, b > 0$ with $p^a > q^b$. If $p^a \mid \sigma (p^a) \sigma (q^b)$, then $p^a = \sigma (q^b)$.
% ko80, no. 78  
\item Solve $\varphi (xy)  = \varphi (x) + \varphi (y)$ for all $x, y \in \Z$, $x, y > 0$.
% ko80, no. 81
\item Show that there are infinitely many odd integers $n$ such that $\sigma (n) > 2 n$. (\emph{Hint}: $945$ is the smallest such positive $n$.)
\end{enumerate}
\chapter[Lecture Twenty-Three]{Day Twenty-Three \hfill {\footnotesize \rm --- 13.03.2017}}

Recall from last time, the Euler $\varphi$-function satisfies
\begin{itemize}
\item the Euler $\varphi$-function is multiplicative, i.e., if $m, n \in \Z$, $m, n > 0$ with $\gcd (m, n) = 1$, then $\varphi (mn) = \varphi (m) \varphi (n)$;
\item if $p$ is a prime and $e \in \Z$ with $e \geqslant 1$, then $\varphi (p^e) = (p-1)p^{e-1}$;
\item if $n = p_1^{e_1} p_2^{e_2} \cdots p_k^{e_k}$ with $p_i$'s distinct primes and $e_i$'s positive integers for $1 \leqslant i \leqslant k$, $k \in \Z$, $k > 0$, then $\varphi (n) = \prod_{i=1}^k (p_i - 1)p_i^{e_i - 1}$.
\end{itemize}

\begin{example}
  Find \underline{all} positive integers $n$ such that $\varphi (n) = 42$.\footnote{The original example in the lecture was $\varphi (n) = 28$. $n = 29 \text{ or } 58$ in that case. The one used here is an example from a homework session.}
\end{example}
\begin{proof}[Solution]
First, note that $42 = 2 \cdot 3 \cdot 7$. Start with the largest divisor $7$. Since $7$ is a prime divisor of $\varphi (n)$, consider the prime factorization of $n$,
\[
 n = p_1^{e_1}p_2^{e_2} \cdots p_k^{e_k},
\]
where $p_i$'s are distinct primes, and $e_i \geqslant 1$ for $1 \leqslant i \leqslant k$, $k \in \Z$, $k > 0$. By the formula
\[
\varphi (n) = \prod_{i=1}^k (p_i - 1) p_i^{e_i -1},
\]
 we have either $7 \mid p_i -1$ or $7 \mid p_i^{e_i -1}$.
\begin{enumerate}[label=(\roman*)]
\item If $7 \mid p_i -1$, note that $p_i - 1$ is bounded by $42$ so write down all primes that are smaller than $43$ and that are also multiples of $7$ plus $1$. There are only two possible primes, namely $29, 43$.
\begin{enumerate}
\item If $ p_i = 29 $, now write $n = 29m$ with $\gcd (29,m) = 1$, $m \in \Z$, $m > 0$. We can rewrite $\varphi(n) = \varphi(29)\varphi(m) = (29 - 1) \varphi(m)$, by Euler $\varphi$-function being multiplicative. It is easy to see that $28 \varphi(m) = 42$ has no integer solutions.
\item If $p_i = 43$, again by $\varphi(n) = \varphi(43) \varphi(m) = (43 - 1) \varphi (m)$, we have $\varphi(m) = 1$. It has two solutions $m = 1, 2$. So $n = 43$ or $86$.
\end{enumerate}
\item If $7 \mid p_i^{e_i -1}$, note that $7^{e_i -1 }$ is also bounded by $42$, so $e_i \leqslant 2$. At the same time $e_i \geqslant 2$ since the power of $7$ survived in the formula of $\varphi(n)$. By a similar argument we have $\varphi(m) = 1$. $m = 1, 2$. The possible $n$ in this case is $49$ or $98$.
\end{enumerate}          
 In conclusion, $\varphi(n) = 42 \iff n = 43, 86, 49 \text{ or } 98$.  
\end{proof}

Recall from last lecture, if $f$ is a multiplicative arithmetic function, then $F(n) = \sum_{\substack{d\, \mid \, n \\ d \, > \, 0}} f(d)$ is also a multiplicative arithmetic function. In particular, $F(n) = \sum_{\substack{d\, \mid \, n \\ d \, > \, 0}} \varphi (d) = n$ is multiplicative where $f(n) = \varphi (n)$. $\sum_{\substack{d\, \mid \, n \\ d \, > \, 0}} \varphi (d) = n$ is called \textbf{Gauss's identity}.

\begin{theorem}[Gauss's theorem on divisor sum]\label{thm:gauss_id}
  For any $n \in \Z$, $n > 0$, we have $n = \sum_{\substack{d\, \mid \, n \\ d \, > \, 0}} \varphi (d)$.
\end{theorem}
\begin{proof}
  For a positive divisor $e$ of $n$, let $\sS_e = \left\{ 1 \leqslant a \leqslant n : \gcd (a, n) = e \right\}$. 

For all $a \in \sS_e$, we have $\gcd (\frac{a}e, \frac{n}e) = 1$ and $1 \leqslant \frac{a}e \leqslant \frac{n}e$. On the other hand, if $1 \leqslant b \leqslant \frac{n}e$ with $\gcd (b, \frac{n}e) = 1$, then $1 \leqslant be \leqslant n$ and $\gcd (be, n) = e$. Thus there is a bijection between $\sS_e$ and the set $\left\{ 1 \leqslant b \leqslant \frac{n}e : \gcd (b, \frac{n}e) = 1 \right\}$. Therefore 
\[
| \sS_e | = \left| \left\{ 1 \leqslant b \leqslant \frac{n}e : \gcd (b, \frac{n}e) = 1 \right\} \right| = \varphi \left(\frac{n}e \right).
\]

Since every $1 \leqslant a \leqslant n$ lies in an $\sS_e$ for some $e \mid n$. And these $\sS_e$'s are disjoint for different $e$'s, then we have 
\begin{align} 
  n &= \left| \left\{ 1 \leqslant a \leqslant n \right\} \right| \\
    &= \sum_{\substack{e\, \mid \, n \\ e \, > \, 0}} \left| \sS_e \right | \\
    &= \sum_{\substack{e\, \mid \, n \\ e \, > \, 0}} \varphi \left( \frac{n}e \right) \\
    &= \sum_{\substack{d\, \mid \, n \\ d \, > \, 0}} \varphi (d),
\end{align}
by setting $d = \frac{n}e$, as $e$ ranges over all positive divisors of $n$, so does $\frac{n}e$.\footnote{Administrative announcement: Midterm 2 covers material up to this point.}
\end{proof}

\chapter[Lecture Twenty-Four]{Day Twenty-Four \hfill {\footnotesize \rm --- 15.03.2017}}

\begin{definition}
A positive integer $n$ is called \textbf{a perfect number} if $\sigma (n) = 2n$. Equivalently, $\sigma (n) - n = n$, i.e., the sum of all positive divisors other than $n$ (or the sum of all proper divisors of $n$) equals $n$.
\end{definition}

\begin{example} 
Quick examples.
\begin{itemize}
\item  $6$ is a perfect number since $\sigma(6) - 6 = 1 + 2 + 3 = 6$. 
\item $28$ is a perfect number since $\sigma(28) - 28 = 1 + 2 + 4 + 7 + 14 = 28$.
\item No prime number $p$ is perfect since $\sigma (p) - p = 1 < p$ for all prime $p$. 
\end{itemize}
\end{example}

\begin{lemma}\label{lem:lem1_day24}
  Let $a$ be a positive integer. If $2^a - 1$ is prime, then $a$ is prime. 
\end{lemma}
\begin{proof}
We prove the lemma by the contrapositive. If $a$ is not prime, then $a$ is composite. Hence we can write $a = bc$ with $b, c \in \Z$, $1 < b < a, 1< c < a$. Then we have 
\begin{align}
  2^a - 1 &= 2^{bc} - 1 \\
          &= (2^b - 1)(2^{bc - b} + 2^{bc - 2b} + \cdots + 2^b + 1).
\end{align}
Since $b > 1$, $2^b - 1 > 2^1 - 1 = 1$. Since $c > 1$, $2^{bc - b} + 2^{bc - 2b} + \cdots + 2^b + 1 > 1$. ($c$ determines how many terms are in the sum.)

Then $2^a - 1$ is composite. 
\end{proof}
\begin{theorem}\label{thm:perfect_mersenne}
  Let $n \in \Z$, $n > 0$ be even. $n$ is a perfect number if and only if $n = 2^{p-1}(2^p-1)$ with $p$ prime and $2^p-1$ a Mersenne prime.
\end{theorem}
\begin{proof}
  ($\Leftarrow$) (Euclid) Assume $n = 2^{p-1}(2^p-1)$ with $2^p-1$ a Mersenne prime. Since $2^p-1$ is odd, $\gcd (2^{p-1}, 2^p-1) = 1$. By multiplicativity of $\sigma$, we have 
  \begin{align}
\hspace{4.9cm}   \sigma (n) &= \sigma \big(2^{p-1}(2^p-1)\big) \\
               &= \sigma (2^{p-1}) \sigma (2^p-1) \\
               &= \left( \sum_{i = 0}^{p-1} 2^i \right)  (1 + 2^p -1) & \btfact{\text{(by $2^p-1$ is prime)}} \\
               &= \frac{2^p-1}{2-1} \cdot 2^p \\
               &= 2\cdot 2^{p-1}(2^p-1) \\
               &= 2n.
  \end{align}
  ($\Rightarrow$) (Euler) Now assume that $n$ is an even perfect number. We can write $n = 2^a m$ with $2 \nmid m$ and $a \in \Z$, $a \geqslant 1$. Since $n$ is perfect, and $\sigma$ is multiplicative, 
\begin{align}
  \sigma (n) &= \sigma (2^a m) \\
             &= \sigma (2^a) \sigma (m) \\
             &= \frac{2^{a+1} - 1}{2-1} \sigma (m) \\
             &= (2^{a+1} - 1) \sigma (m).
\end{align}
On the other hand, by $n$ a perfect number, $\sigma (n) = 2n = 2\cdot 2^a m$. So $2^{a+1}m = (2^{a+1} - 1)\sigma (m)$. Note $\gcd (2^{a+1}, 2^{a+1}-1) = 1$ since two consecutive positive integers are coprime. Therefore $2^{a+1} \mid \sigma (m)$. 

Write $\sigma (m) = 2^{a+1} k$ with $k \in \Z$, $k > 0$. Then $2^{a+1}m = (2^{a+1} - 1)2^{a+1}k \implies m = (2^{a+1}-1)k$. We can show $k = 1$. Otherwise, if $1 < k < m$ and $k \mid m$, then 
\begin{align}
  \sigma (m) &\geqslant 1 + k + m \\
             &= 1 + k + (2^{a+1}-1)k \\
             &= 1 + \sigma (m),
\end{align}
a contradiction. Substitute $k = 1$ into $m$, then we have $m = 2^{a+1}-1$. 

We now show $2^{a+1} - 1$ is prime, then by Lemma~\ref{lem:lem1_day24}, $a + 1$ is prime and we are done. Indeed, we saw that $\sigma (m) = 2^{a+1}k = \sigma (2^{a+1} - 1) = 1 + (2^{a+1} - 1)$, since $k = 1$, i.e., $2^{a+1}-1$ has no other positive divisors than $1$ and itself. Hence $2^{a+1} - 1$ is prime. Take $p = a + 1$ we proved the ($\Rightarrow$) direction.
\end{proof}

We stated earlier in Conjecture~\ref{conj:inf_mersenne} that there are infinitely many Mersenne primes. By Theorem~\ref{thm:perfect_mersenne}, Conjecture~\ref{conj:inf_mersenne} is equivalent to
\begin{conjecture}
  There are infinitely many even perfect numbers. 
\end{conjecture}
Another conjecture about perfect number is 
\begin{conjecture}
  There are no odd perfect numbers.
\end{conjecture}

\chapter[Lecture Twenty-Five]{Day Twenty-Five \hfill {\footnotesize \rm --- 17.03.2017}}

In Lecture~\ref{lec22}, we saw in Theorem~\ref{thm:f_F_multi} that given an arithmetic function $f$, we can build an new arithmetic function $F(n) = \sum_{\substack{d \, \mid \, n \\ d \, > \, 0}} f(d)$. If $f$ is multiplicative, then so is $F$. 

Q\underline{uestion}: Can we recover the function $f$ upon which $F$ is built?

First let us try a few examples.
\begin{align}
  F(1) &= f(1) \hspace{10.7mm}\implies \\
  f(1) &= F(1). \\
  F(2) &= f(1) + f(2) \implies \\
  f(2) &= F(2) - F(1).
\end{align}
For any prime $p$, since $F(p) = f(1) + f(p)$, it implies $f(p) = F(p) - F(1)$. If $n = pq$ with $p, q$ prime and $p \neq q$, then
\begin{align}
  F(n) &= F(pq) \\
       &= f(1) + f(p) + f(q) + f(pq) \implies \\
  f(pq)&= F(pq) - \big(F(p)-F(1)\big) - \big(F(q)-F(1)\big) - F(1).
\end{align}
We can continue this. Therefore $F$ does retain all the information about $f$.

However, is there a clear way of recovering $f$ from $F$? Yes, recovering $f$ from $F$ is called \textbf{M\"obius inversion}.
\begin{definition}
  The \textbf{M\"obius function}, denoted $\mu$, is defined by
  \begin{align}
    \mu (n) = \left\{ \begin{array}{rl}
                        1 & \text{if $n = 1$,} \\[1mm]
                        0 & \text{if $p^2 \mid n$, $p$ is prime,} \\[1mm]
                        (-1)^k & \text{if $n = p_1p_2\cdots p_k$ with $p_1, p_2, \ldots, p_k$ distinct primes, $k \in \Z_{\geqslant 1}$.}
                      \end{array} \right.
  \end{align}
  for any $n \in \Z$, $n > 0$.
\end{definition}
\begin{example}
  $\mu (1) = 1$, $\mu (2) = -1$, $\mu (3) = -1$, $\mu (4) = 0$, $\mu (6) = 1$, $\mu (12) = 0$, $\mu (30) = -1$.
\end{example}
\begin{theorem}
  The M\"obius function is multiplicative.
\end{theorem}
\begin{proof}
  Let $m, n \in \Z$, $m, n > 0$. If $m = 1$, then $\mu (mn) = \mu (1\cdot n) = 1 \cdot \mu (n) = \mu (m) \mu (n)$. Similarly, if $n = 1$, $\mu (n) \mu (m) = \mu (mn)$. Either way, $\gcd (1, n) = 1$ or $\gcd (m, 1) = 1$, i.e., with $\gcd (m, n) = 1$ we have $\mu (mn) = \mu (m) \mu (n)$.

  Assume $m, n > 1$. If there is a prime $p$ such that $p^2 \mid m$ or $p^2 \mid n$, then $p^2 \mid mn$ and $\mu (mn) = 0 = \mu (m) \mu (n)$.

  Now assume $n = p_1 p_2 \cdots p_k$ with $p_1, p_2, \ldots, p_k$ distinct primes and $m = q_1 q_2 \cdots q_{\ell}$ with $q_1, q_2, \ldots, q_{\ell}$ distinct primes, $k, \ell \in \Z$, $k, \ell > 0$. If $\gcd (m, n) = 1$, then $p_i \neq q_j$ for any $1 \leqslant i \leqslant k$, $1 \leqslant j \leqslant \ell$. Therefore
  \[
    mn = p_1 p_2 \cdots p_k q_1 q_2 \cdots q_{\ell} \\
  \]
  is a product of distinct primes. Then $\mu (mn) = (-1)^{\ell+k} = (-1)^{\ell}(-1)^k = \mu (m) \mu (n)$.
\end{proof}
\begin{remark}
  $\mu$ is not completely multiplicative. For example, $\mu (n\cdot n) = 0 \neq (-1)^{2k} = 1$ for $n$ with $\mu (n) = (-1)^k$.
\end{remark}
Then by Theorem~\ref{thm:f_F_multi}, $F(n) = \sum_{\substack{d \, \mid \, n \\ d \, > \, 0}} \mu (d)$ is multiplicative.
\begin{proposition}\label{prop:sum_mu}
  Let $n \in \Z$, $n > 0$. Then $\sum_{\substack{d \, \mid \, n \\ d \, > \, 0}} \mu (d) = \left\{ \begin{array}{rl}
                                                                                                     1 & \text{if $n = 1$,} \\[1mm]
                                                                                                     0 & \text{otherwise.}
                                                                                                         \end{array} \right.$
\end{proposition}
\begin{proof}
  If $n = 1$, then $\sum_{\substack{d \, \mid \, n \\ d \, > \, 0}} \mu (d) = \mu (1) = 1$.

  Let $p$ be a prime and let $e \in \Z$, $e > 0$, then
  \[
    F(p^e) = \sum_{\substack{d \, \mid \, p^e \\ d \, > \, 0}} \mu (d) = \sum_{i=1}^e \mu (p^i) 
    = 1 - 1 + 0 + 0 + \cdots + 0 = 0.
  \]
  Then if $n \in \Z$, $n > 1$, writing $n = p_1^{e_1} p_2^{e_2} \cdots p_k^{e_k}$ with $p_1, p_2, \ldots, p_k$ distinct primes and $e_1, e_2, \ldots, e_k$ positive integers, then
  \[
    F(n) = \prod_{i=1}^k F(p_i^{e_i}) 
         = \prod_{i=1}^k 0,
  \]
  since $F(n) = \sum_{\substack{d \, \mid \, n \\ d \, > \, 0}} \mu (d)$ is multiplicative.
\end{proof}

\begin{theorem}[M\"obius inversion]\label{thm:mobius_inv}
  Let $f$ and $g$ be arithmetic functions. Then $f(n) = \sum_{\substack{d \, \mid \, n \\ d \, > \, 0}} g(d)$ if and only if
  \[
    g(n) = \sum_{\substack{d \, \mid \, n \\ d \, > \, 0}} \mu \left(\frac{n}d\right) f(d) = \sum_{\substack{d \, \mid \, n \\ d \, > \, 0}} \mu (d) f\left(\frac{n}d\right).
  \]
\end{theorem}

M\"obius inversion shows how to recover function $g$ from function $f$ (and vice versa), where $\mu$ tracks the sign of $f(d)$ (or of $f\left( \frac{n}d \right)$ in the second equality). Before we prove the M\"obius inversion, we shall see some immediate consequences of Theorem~\ref{thm:mobius_inv}.

\begin{example}
  Let $n \in \Z$, $n > 0$.
  \begin{itemize}
  \item We saw in Theorem~\ref{thm:gauss_id} that $n = \sum_{\substack{d \, \mid \, n \\ d \, > \, 0}} \varphi (d)$. M\"obius inversion implies
    \[
      \varphi (n) = \sum_{\substack{d \, \mid \, n \\ d \, > \, 0}} \mu \left( \frac{n}d \right) d.
    \]
  \item We saw in Example~\ref{ex:nu_func} that $\nu (n) = \sum_{\substack{d \, \mid \, n \\ d \, > \, 0}} 1$. By M\"obius inversion, we have
    \[
      1 = \sum_{\substack{d \, \mid \, n \\ d \, > \, 0}} \mu \left( \frac{n}d \right) \nu (d).
    \]
  \item We saw in Example~\ref{ex:sigma_func} that $\sigma (n) = \sum_{\substack{d \, \mid \, n \\ d \, > \, 0}} d$. By M\"obius inversion, we have
    \[
      n = \sum_{\substack{d \, \mid \, n \\ d \, > \, 0}} \mu \left( \frac{n}d \right) \sigma (d).
    \]
  \item By the third example and Gauss's identity, we have
    \[
      \sum_{\substack{d \, \mid \, n \\ d \, > \, 0}} \mu \left( \frac{n}d \right) \sigma (d) = \sum_{\substack{d \, \mid \, n \\ d \, > \, 0}} \varphi (d).
    \]
  \end{itemize}
\end{example}

\begin{proof}[Proof {\rm (of M\"obius inversion)}]
  First note that $\sum_{\substack{d \, \mid \, n \\ d \, > \, 0}} \mu \left(\frac{n}d\right) f(d) = \sum_{\substack{d \, \mid \, n \\ d \, > \, 0}} \mu (d) f\left(\frac{n}d\right)$ holds because as $d$ ranges over all positive divisors of $n$, so does $\frac{n}d$.

  ($\Rightarrow$) Assume $f(n) = \sum_{\substack{d \, \mid \, n \\ d \, > \, 0}} g(d)$ for $n \in \Z$, $n > 0$. Then
  \[
    \sum_{\substack{d \, \mid \, n \\ d \, > \, 0}} \mu (d) f\left( \frac{n}d \right) = \sum_{\substack{d \, \mid \, n \\ d \, > \, 0}} \Bigg( \mu (d) \sum_{\substack{c \, \mid \, \frac{n}d \\ c \, > \, 0}} g(c) \Bigg).
  \]

  This double sum is over all pairs $d$ and $c$ with $d \mid n$ and $c \mid \frac{n}d$.

  This is the same as summing over pairs $d$ and $c$ with $dc \mid n$. (This argument is the bridge between the claims before and after.)

  Thus it is the same as summing over pairs $d$ and $c$ with $c \mid n$ and $d \mid \frac{n}c$. Therefore
  \[
    \sum_{\substack{d \, \mid \, n \\ d \, > \, 0}} \Bigg( \mu (d) \sum_{\substack{c \, \mid \, \frac{n}d \\ c \, > \, 0}} g(c) \Bigg) =
    \sum_{\substack{c \, \mid \, n \\ c \, > \, 0}} \Bigg( g (c) \sum_{\substack{d \, \mid \, \frac{n}c \\ d \, > \, 0}} \mu (d) \Bigg).
    \]
    Note that by Proposition~\ref{prop:sum_mu},
    \[
      \sum_{\substack{d \, \mid \, \frac{n}c \\ d \, > \, 0}} \mu (d) =
      \left\{
      \begin{array}{rl}
        1 & \text{if $n = c$,} \\
        0 & \text{otherwise.}
            \end{array} \right.
    \]
    Thus
    \[
      \sum_{\substack{c \, \mid \, n \\ c \, > \, 0}} \Bigg( g (c) \sum_{\substack{d \, \mid \, \frac{n}c \\ d \, > \, 0}} \mu (d) \Bigg) =
      g(n),
    \]
    which is only term that survives in the sum.

    ($\Leftarrow$) Now assume $g(n) = \sum_{\substack{d \, \mid \, n \\ d \, > \, 0}} \mu \left(\frac{n}d\right) f(d)$. Then
    \begin{align}\label{eqn:g2f}
      \begin{split}
      \sum_{\substack{d \, \mid \, n \\ d \, > \, 0}}  g (d) &=
                                                             \sum_{\substack{d \, \mid \, n \\ d \, > \, 0}} \, \sum_{\substack{c \, \mid \, d \\ c \, > \, 0}} \mu \left( \frac{d}c \right) f(c) \\
                                                           &= \sum_{\substack{c \, \mid \, n \\ c \, > \, 0}} f(c)
      \sum_{\substack{m \, \mid \, \frac{n}c \\ m \, > \, 0}} \mu (m) \\
      &= f(n).
      \end{split}
    \end{align}
    The second equality of Equation~\eqref{eqn:g2f} is because of a similar argument about pairs $\frac{d}c$ and $c$ as above about pairs $d$ and $c$. Indeed, summing over $\frac{d}c$ and $c$ pairs with $d \mid n$ and $c \mid d$ is the same as summing over $d = \frac{d}c \cdot c$ with $d \mid n$. Let $m = \frac{d}c$, then it is the same as summing over pairs $c$ and $m$ with $c \mid n$ and $m \mid \frac{n}c$. The last equality follows since $\frac{n}c$ has to be $1$ in order for $f(c)$ to survive in the sum according to Proposition~\ref{prop:sum_mu}.
\end{proof}

-- End of Part~\ref{part2}.

\part{March to May, 2017}\label{part3}

\chapter[Lecture Twenty-Six]{Day Twenty-Six \hfill {\footnotesize \rm --- 27.03.2017}}

\underline{Primitive roots}\\

The rough goal of our study of primitive roots is to understand the modulo arithmetic of powers. Recall Euler's theorem, Theorem~\ref{thm:euler_thm} says if $a, n \in \Z$ with $n \geqslant 1$ and $\gcd (a,n) = 1$, then
\[
  a^{\varphi (n)} \equiv 1 \bmod n.
\]
However, for any $a$ and $n$ as above, we may have $a^e \equiv 1 \bmod n$ for some $0 < e < \varphi (n)$. For example, for any $n > 1$, clearly, $1^1 \equiv 1 \bmod n$.
\begin{definition}\label{def:order_mod_n}
  Let $a, n \in \Z$ with $n \geqslant 1$ and $\gcd (a, n) = 1$. The smallest positive integer $e$ such that $a^e \equiv 1 \bmod n$ is called the \textbf{order of $a$ modulo $n$}, written as $\text{ord}_n (a)$.
\end{definition}
\begin{remark}\label{rmk:ord_cong}
  For $a$ and $n$ as in Definition~\ref{def:order_mod_n},
  \begin{enumerate}[label=(\roman*)]
  \item $\text{ord}_n (a)$ exists by Euler's theorem.
  \item If $b \equiv a \bmod n$, then $\text{ord}_n (b) = \text{ord}_n (a)$.\label{itm:itm1_day26}
  \end{enumerate}
\end{remark}

\begin{example}\label{ex:ex1_day26}
  Let us find the orders of all integers modulo $9$.
\end{example}
\begin{proof}[Solution]
  There are $\varphi (9) = 6$ integers that are coprime with $9$. They are $1, 2, 4, 5, 7, 8$. It suffices to find the orders of these six integers modulo $9$ by~\ref{itm:itm1_day26} in Remark~\ref{rmk:ord_cong}.
  As noted above $\ord_9 (1) = 1$.

  As for $2$, 
\begin{align}
2^1 = 2 &\equiv 2 \bmod 9 \\
2^2 = 4 &\equiv 4 \bmod 9 \\
2^3 = 8 &\equiv 8 \bmod 9 \\
2^4 = 16 &\equiv 7 \bmod 9 \\
2^5 = 32 &\equiv 5 \bmod 9 \\
2^6 = 2^{\varphi (9)} &\equiv 1 \bmod 9.
\end{align}
Hence $\ord_9 (2) = 6$.

  As for $4$, 
\begin{align}
4^1 = 4 &\equiv 4 \bmod 9 \\
4^2 = 16 &\equiv 7 \bmod 9 \\
4^3 = 2^6 &\equiv 1 \bmod 9. 
\end{align}
Hence $\ord_9 (4) = 3$.

  As for $5$, 
\begin{align}
5^1 = 5 &\equiv 5 \bmod 9 \\
5^2 = 25 &\equiv 7 \bmod 9 \\
5^3 \equiv 7 \times 5 &\equiv 8 \bmod 9 \\
5^4 \equiv 8 \times 5 &\equiv 4 \bmod 9 \\
5^5 \equiv 4 \times 5 &\equiv 2 \bmod 9 \\
5^6 \equiv 2 \times 5 &\equiv 1 \bmod 9.
\end{align}
Hence $\ord_9 (5) = 6$.

  As for $7$, 
\begin{align}
7^1 = 7 &\equiv 7 \bmod 9 \\
7^2 = 49 &\equiv 4 \bmod 9 \\
7^3 \equiv 4 \times 7 &\equiv 1 \bmod 9.
\end{align}
Hence $\ord_9 (7) = 3$.

  As for $8$, $8^1  \equiv 8 \bmod 9$, $8^2 = 64 \equiv 1 \bmod 9$. Hence $\ord_9 (8) = 2$.
\end{proof}

\begin{proposition}\label{prop:ord_div_k}
  Let $a, n \in \Z$ with $n \geqslant 1$ and $\gcd (a, n) = 1$. A $k \in \Z$, $k \geqslant 1$ satisfies $a^k \equiv 1 \bmod n$ if and only if $\ord_n (a) \mid k$.
\end{proposition}
\begin{proof}
  ($\Leftarrow$) Let $e = \ord_n (a)$. First assume $e \mid k$. Then $k = \ell e$ for some $\ell \in \Z$, $\ell \geqslant 1$. We have $a^k = (a^e)^{\ell} \equiv 1 \bmod n$ by definition of order of $a$ modulo $n$.

($\Rightarrow$) Assume $k \in \Z$, $k \geqslant 1$ satisfies $a^k \equiv 1 \bmod n$. By division algorithm,\footnote{Trick: the ``smallest integer'' part in the definition of order implies the avenue of the proof might be some inequality involving divisibility. Consider division algorithm.} there are unique $q, r \in \Z$ such that 
\[
k = qe + r \text{ and } 0 \leqslant r < e.
\]
Note that $r < e$ and $k > 0$. We have $q \geqslant 0$. Therefore 
\[
a^k = (a^e)^q a^r \equiv a^r \bmod n.
\]

On the other hand, $a^k \equiv 1 \bmod n$ by our assumption, so $a^r \equiv 1 \bmod n$. Since $0 \leqslant r < e$ and $e$ is the smallest positive integer with $a^e \equiv 1 \bmod n$, we must have $r = 0$, i.e., $e \mid k$.
\end{proof}

\begin{corollary}\label{cor:ord_div_phi}
$\ord_n (a) \mid \varphi (n)$ for $a, n \in \Z$ with $n \geqslant 1$ and $\gcd (a, n) = 1$.
\end{corollary}
\begin{proof}
  Take $k = \varphi (n)$ in Proposition~\ref{prop:ord_div_k} then the corollary follows.
\end{proof}

\begin{definition}\label{def:prim_rt}
  Let $a, n \in \Z$ with $n \geqslant 1$ and $\gcd (a, n) = 1$. We say $a$ is a \textbf{primitive root modulo $n$} if $\ord_n (a) = \varphi (n)$.
\end{definition}

\begin{example}
  We saw in Example~\ref{ex:ex1_day26} that $2$ and $5$ are primitive roots modulo $9$.
\end{example}

Q\underline{uestion}: Do primitive roots always exist? (No.)

Consider modulo $8$. We saw in Remark~\ref{rmk:qr_comp_mod} that $1^2 \equiv 3^2 \equiv 5^2 \equiv 7^2 \equiv 1 \bmod 8$. Therefore, any $a \in \Z$ with $\gcd (a, 8) = 1$ has 
\[
\ord_8 (a) = \left\{ \begin{array}{rl} 
                       1 & \text{if $a \equiv 1 \bmod 8$,} \\[1mm]
                       2 & \text{if $a \not\equiv 1 \bmod 8$.}
                     \end{array} \right.
\]
However $\varphi (8) = 4$, meaning $\ord_8(a) < 4$ all the time.\\[-1.5mm]

Q\underline{uestion}: Which positive integers have a primitive root?

To answer this question, we need a slow build-up.

\begin{restatable}{proposition}{powcongexpcong}\label{prop:pow_cong_exp_cong}
  Let $a, n \in \Z$ with $n \geqslant 1$ and $\gcd (a, n) = 1$. Let $k, \ell \in \Z$ such that $k, \ell \geqslant 1$. Then $a^k \equiv a^{\ell} \bmod n$ if and only if $k \equiv \ell \bmod \ord_n (a)$.
\end{restatable}
\begin{proof}
  Let $e = \ord_n (a)$. Without loss of generality, we can assume $k \geqslant \ell$.

($\Leftarrow$) First assume $k \equiv \ell \bmod \ord_n (a)$. Then $k = \ell + qe$ for some $q \in \Z$. Since $k \geqslant \ell$, we have $q \geqslant 0$. Therefore, 
\[
a^k = (a^e)^q a^{\ell} \equiv a^{\ell} \bmod n.
\]

($\Rightarrow$) Now assume $a^k \equiv a^{\ell} \bmod n$. Since $\gcd (a, n) = 1$, there is a unique $b$ such that 
\[
\hspace{5.5mm}ab \equiv 1 \bmod n.
\]
Since $k \geqslant \ell$, multiply both sides of $a^k \equiv a^{\ell} \bmod n$ by $b^{\ell}$,
\begin{align}
  a^k b^{\ell} &\equiv a^{\ell} b^{\ell} \hspace{1.3mm}\bmod n \\
  a^{k-\ell} (ab)^{\ell} &\equiv (ab)^{\ell} \bmod n \\
  a^{k-\ell} &\equiv 1 \hspace{5.7mm}\bmod n.
\end{align}
So we have $\ord_n (a) \mid k - \ell$, i.e., $k \equiv \ell \bmod \ord_n (a)$ by Proposition~\ref{prop:ord_div_k}.
\end{proof}
\begin{remark}
  Proposition~\ref{prop:pow_cong_exp_cong} is the generalization of Proposition~\ref{prop:ord_div_k}. Indeed, if we substitute $\ell = \ord_n (a)$ in the former then we get $k \equiv 0 \bmod \ord_n (a)$ as in the latter. But note that the proof of Proposition~\ref{prop:pow_cong_exp_cong} actually uses Proposition~\ref{prop:ord_div_k}, then Proposition~\ref{prop:ord_div_k} cannot be treated as a corollary of Proposition~\ref{prop:pow_cong_exp_cong}.
\end{remark}

\chapter[Lecture Twenty-Seven]{Day Twenty-Seven \hfill {\footnotesize \rm --- 29.03.2017}}

Recall from last time,
\powcongexpcong*
We will use this proposition to prove the following corollary.
\begin{corollary}\label{cor:coprime_cong_prim_rt_power}
  Let $a, n \in \Z$ with $n \geqslant 1$ and $\gcd (a, n) = 1$. If $a$ is a primitive root modulo $n$, then any $b \in \Z$ with $\gcd (b, n) = 1$ is congruent modulo $n$ to \textbf{precisely} one of
  \[
    1,\, a, \, a^2, \,\ldots,\, a^{\varphi(n) - 1}.
  \]
\end{corollary}
\begin{proof}
  Since $a$ is a primitive root modulo $n$, by definition $\ord_n (a) = \varphi (n)$. By Proposition~\ref{prop:pow_cong_exp_cong},
  \[
    a, \, a^2, \,\ldots,\, a^{\varphi(n) - 1}, \, a^{\varphi(n)} (\,\equiv 1 \bmod n)
  \]
  are all distinct modulo $n$. Indeed, if $1 \leqslant k, \ell \leqslant \varphi (n)$ with $a^k \equiv a^{\ell} \bmod n$, then $k \equiv \ell \bmod \varphi(n)$. But $1 \leqslant k, \ell \leqslant \varphi (n) \implies k = \ell$. We know there are just $\varphi (n)$ distinct congruence classes modulo $n$ consisting of integers coprime with $n$. We just showed that
  \[
    1,\, a, \, a^2, \,\ldots,\, a^{\varphi(n) - 1}
  \]
  define $\varphi (n)$ distinct congruence classes consisting of integers coprime with $n$. So that is all of them.
\end{proof}

\begin{example}
  In Example~\ref{ex:ex1_day26} we saw $2$ is a primitive root modulo $9$. So every integer coprime with $9$ is congruent modulo $9$ to one of
\[
  1, 2, 2^2, 2^3, 2^4, 2^5.
\]
\end{example}

We computed last time
\begin{center}
\begin{tabular}{c|c} \hline
  $a$ & $\ord_9 (a)$ \\ \hline
  1 & 1 \\
  $2^1$ & 6 \\
  $2^2$ & 3 \\
  $2^3$ & 2 \\
  $2^4$ & 3 \\
  $2^5$ & 6 \\ \hline
\end{tabular}
\end{center}
and we note that
\begin{align}
  \ord_9 (2^2) = 3 = \frac{6}2 &= \frac{\ord_9 (2)}2 \\
  \ord_9 (2^3) = 2 = \frac{6}3 &= \frac{\ord_9 (2)}3 \\
  \ord_9 (2^4) = 3 = \frac{6}2 &= \frac{\ord_9 (2)}2 \\
  \ord_9 (2^5) = 6 = \frac{6}1 &= \frac{\ord_9 (2)}1.
\end{align}
We saw $\displaystyle \ord_9 (2^k) = \frac{6}{\gcd (6, k)} = \frac{\ord_9 (2)}{\gcd (\ord_9 (2), k)}$ for $2 \leqslant k \leqslant 5$, $k \in \Z$.

\begin{proposition}\label{prop:ord_pow}
  Let $a, n \in \Z$ with $n \geqslant 1$ and $\gcd (a, n) = 1$. For any $k \in \Z$, $k \geqslant 0$, we have 
\[
\ord_n (a^k) = \frac{\ord_n (a)}{\gcd (\ord_n (a), k)}.
\]
\end{proposition}
\begin{proof}
  Let $e = \ord_n (a)$, $f = \ord_n (a^k)$ and $d = \gcd (e, k)$. We want to show $f = \frac{e}d$.

  Since $d \mid e$ and $d \mid k$, we know
  \[
    (a^k)^\frac{e}d = (a^{\frac{ke}d}) = (a^e)^{\frac{k}d} \equiv 1 \bmod n.
  \]
  For this number $a^k$, $(a^k)^{\frac{e}d} \equiv 1 \bmod n$ implies $\ord_n (a^k) \mid \frac{e}d$ by Proposition~\ref{prop:ord_div_k}.

  On the other hand, $a^{kf} = (a^k)^f \equiv 1 \bmod n$ since $f = \ord_n (a^k)$. Again by Proposition~\ref{prop:ord_div_k}, $e \mid kf$ since $e = \ord_n (a)$. Take $d = \gcd (e, k)$, then we have $\frac{e}d \mid \left(\frac{k}d \right)f$. However $\frac{e}d$ and $\frac{k}d$ are coprime by Example~\ref{ex:takeout_gcd_coprime}. So we must have $\frac{e}d \mid f$.

  Since $\frac{e}d \mid f$ and $f \mid \frac{e}d$, both $f$ and $\frac{e}d$ are positive integers (by orders are positive, $0$ is ruled out), then we have $f = \frac{e}d$ as desired.
\end{proof}
\begin{corollary}
  Let $n \in \Z$, $n \geqslant 1$. If there exists a primitive root modulo $n$, then there are \textbf{exactly} $\varphi (\varphi (n))$ incongruent primitive roots modulo $n$.
\end{corollary}
\begin{proof}
  Let $a$ be a primitive root modulo $n$. We saw in Corollary~\ref{cor:coprime_cong_prim_rt_power} the $\varphi (n)$ distinct incongruent classes are described by powers of $a$,
  \[
    a, \, a^2, \,\ldots,\, a^{\varphi(n) - 1}, \, a^{\varphi(n)} (\,\equiv 1 \bmod n).
  \]
  Any potential primitive root is one of these integers that are coprime with $n$. By Proposition~\ref{prop:pow_cong_exp_cong}, for $1 \leqslant k \leqslant \varphi (n)$,
  \[
    \ord_n (a^k) = \varphi (n) \iff \gcd (\varphi(n), k) = 1.
  \]
  There are $\varphi (\varphi (n))$ such $k$'s that are coprime with $\varphi (n)$ for $1 \leqslant k \leqslant \varphi (n)$.
\end{proof}  
\chapter[Lecture Twenty-Eight]{Day Twenty-Eight \hfill {\footnotesize \rm --- 03.04.2017}}

Our next goal is to prove the following theorem,
\begin{theorem}
  For any prime $p$, there exists a primitive root modulo $p$.
\end{theorem}
\begin{proof}
  The theorem is implied in Corollary~\ref{cor:prim_rt_exists_mod_p} after we prove Theorem~\ref{thm:ord_div_p-1} in Lecture~\ref{chp:prim_rt_exists}.
  \end{proof}

We showed in Proposition~\ref{prop:quad_res_no_soln} for an odd prime $p$ and $a \in \Z$, the congruence $x^2 \equiv a \bmod p$ has \underline{at most} two\footnote{zero or two if $\gcd (a, p) = 1$; one if $p \mid a$.} incongruent solutions modulo $p$. But we also saw in Remark~\ref{rmk:qr_comp_mod} this fails for composite moduli; $x^2 \equiv 1 \bmod 8$ has four incongruent solutions modulo $8$.

\begin{theorem}[Lagrange]\label{thm:lagrange}
  Let $p$ be a prime and let $f(x)$ be a polynomial with integer coefficients,
  \[
    f(x) = a_d x^d + a_{d-1} x^{d-1} + \cdots + a_1 x + a_0,
  \]
  where $a_i \in \Z$ for each $0 \leqslant i \leqslant d$, $d \in \Z$, $d \geqslant 0$. Assume $p \nmid a_d$, the congruence
  \[
    f(x) \equiv 0 \bmod p
  \]
  has \textbf{at most} $d$ incongruent solutions modulo $p$.
\end{theorem}
\begin{proof}
  If $d = 0$, $a_0 \equiv 0 \bmod p$ with $p \nmid a_0$ has no solutions.
  
  We induct on $d = \deg (f(x))$, $d \geqslant 1$. If $d = 1$, then $f(x) = a_1 x + a_0$.
  \[
    f(x) \equiv 0 \bmod p \iff a_1 x \equiv -a_0 \bmod p,
  \]
  which has a unique solution modulo $p$ by Corollary~\ref{cor:lin_cong_soln} since $p \nmid a_1 \implies \gcd (a_1, p) = 1$. This shows the base case is true.

  Assume the result is true for polynomials of degree $d - 1 \geqslant 1$. Let 
  \[
    f(x) = a_d x^d + a_{d-1} x^{d-1} + \cdots + a_1 x + a_0,
  \]
  with $a_i \in \Z$ for each $0 \leqslant i \leqslant d$, $d \in \Z$, $d \geqslant 1$ and $p \nmid a_d$.

  If there is no $b \in \Z$ with $f(b) \equiv 0 \bmod p$, then we are done since $f(x) \equiv 0 \bmod p$ has no solutions in that case.

  So we can assume there is a $b \in \Z$  with $f(b) \equiv 0 \bmod p$. Note for any $k \in \Z$, $k \geqslant 1$,
  \[
    x^k - b^k = (x - b)(x^{k-1} + x^{k-2}b + \cdots + xb^{k-2} + b^{k-1}).
  \]
  Then notice we can consider instead $f(x) - f(b) \equiv 0 \bmod p$ since $f(b) \equiv 0 \bmod p$ and
  \begin{align}
    f(x) - f(b) &= (a_d x^d + a_{d-1} x^{d-1} + \cdots + a_1 x + a_0) - \\
                & \hspace{10.8mm} (a_d b^d + a_{d-1} b^{d-1} + \cdots + a_1 b + a_0) \\
                &= a_d(x^d - b^d) + a_{d-1}(x^{d-1} - b^{d-1}) + \cdots + a_1(x - b) + 0 \\
                &= (x - b)\underbrace{\big( a_d(x^{d-1} + \cdots  + b^{d-1}) + a_{d-1} (x^{d-2} + \cdots  + b^{d-2}) + \cdots + a_1 \big)}_{\triangleq \,g(x)} \\
                &= (x - b)g(x),
  \end{align}
  where $g(x)$ is a degree $d - 1$ polynomial with integer coefficients and leading coefficient $a_d$ with $\gcd (a_d, p) = 1$.
  By our induction hypothesis, $g(x) \equiv 0 \bmod p$ has at most $d -1$ incongruent solutions modulo $p$. Now
  \begin{align}
\hspace{35mm}    \exists~ c \in \Z \text{ such} & \text{ that } f(c) \equiv 0 \hspace{2mm}\bmod p \\
    \iff & f(c) - f(b) \equiv 0 \hspace{0.3mm}\bmod p  & \btfact{\text{(by $f(b)\equiv 0 \bmod p$)}}\\
    \iff & (c-b) g(c) \equiv 0 \hspace{1.2mm}\bmod p \\
    \iff & c \equiv b \bmod p \text{ or } g(c) \equiv 0 \bmod p.
  \end{align}
  The last equivalence is due to $p$ is prime. (See Proposition~\ref{prop:p_div}.) This implies $f(x) \equiv 0 \bmod p$ has at most $d$ incongruent solutions since $g(x) \equiv 0 \bmod p$ has at most $d - 1$ incongruent solutions modulo $p$. The result therefore follows by induction.
\end{proof}
\begin{remark}
  Lagrange's theorem parallels what we have been familiar with since grade school, i.e., solutions to polynomials in $\R$ and $\Q$, since $\Z/p\Z$ behaves like a field in a sense as fields $\R, \Q$ do. ($(\Z_p, +, \times)$ is a finite field $\F_p$.)
\end{remark}
\begin{corollary}\label{cor:cor_lagrange}
  Let $p$ be a prime and let $d \mid p - 1$, $d > 0$. Then the congruence $x^d - 1 \equiv 0 \bmod p$ has \textbf{exactly} $d$ incongruent solutions modulo $p$.
\end{corollary}
\begin{proof}
  Write $p - 1 = de$ with $e \in \Z$, $e > 0$. Then
  \[
    x^{p-1} - 1 = (x^d - 1)(x^{d(e-1)} + x^{d(e-2)}) + \cdots + x^d + 1).
  \]
  Lagrange's theorem (Theorem~\ref{thm:lagrange}) implies that
  \begin{align}\label{eqn:eqn1_day28}
    x^d - 1 \equiv 0 \bmod p 
  \end{align}
  has \underline{at most} $d$ incongruent solutions while
  \begin{align}\label{eqn:eqn2_day28}
    x^{d(e-1)} + x^{d(e-2)}) + \cdots + x^d + 1 \equiv 0 \bmod p 
  \end{align}
  has \underline{at most} $d(e-1)$ incongruent solutions.
  By Fermat's little theorem, any $a \in \Z$, $1 \leqslant a \leqslant p$ satisfies
  \[
    a^{p-1} - 1 \equiv 0 \bmod p.
  \]
  Since $p$ is prime, it implies that for any $1 \leqslant a \leqslant p - 1$, either $a^d - 1 \equiv 0 \bmod p$ or 
  \[
    a^{d(e-1)} + a^{d(e-2)}) + \cdots + a^d + 1 \equiv 0 \bmod p.
  \]
  Since $p - 1 = d + d(e-1)$,\footnote{$p-1$ is the number of incongruent solutions of congruence $x^{p-1} - 1 \equiv 0 \bmod p$, or the number of integers between $1$ and $p$ coprime with $p$.} we have to max out the number of incongruent solutions of congruences in Equation~\eqref{eqn:eqn1_day28} and Equation~\eqref{eqn:eqn2_day28} respectively. The first one means nothing but $x^d -1 \equiv 0 \bmod p$ has exactly $d$ incongruent solutions modulo $p$.
\end{proof}

\chapter[Lecture Twenty-Nine]{Day Twenty-Nine \hfill {\footnotesize \rm --- 05.04.2017}}\label{chp:prim_rt_exists}

\begin{theorem}\label{thm:ord_div_p-1}
  Let $p$ be a prime and let $d \mid p - 1$, $d > 0$. There are exactly $\varphi (d)$ incongruent elements with order $d$ modulo $p$. In particular, for any $d \mid p-1$ with $d > 0$, there is an integer coprime with $p$ that has order $d$ modulo $p$. 
\end{theorem}
By Definition~\ref{def:prim_rt}, a primitive root modulo $p$ is an integer $a$ coprime with $p$ such that
\[
  \ord_p (a) = \varphi (p) = p - 1.
\]
Hence the following corollary,
\begin{corollary}\label{cor:prim_rt_exists_mod_p}
  Let $p$ be a prime. There are precisely $\varphi (p-1)$ incongruent primitive roots modulo $p$. In particular, primitive root modulo $p$ exists.
\end{corollary}
\begin{example}
 Consider the case $p = 19$. Since $19 - 1 = 18$, we have $\varphi (d) = \varphi (1) = 1$ element of order $d = 1$.
\end{example}
\begin{proof}[Proof {\rm (of Theorem~\ref{thm:ord_div_p-1})}]
  For $d \mid p-1$, $d > 0$. Let $N_d = | \left\{ 1 \leqslant a \leqslant p-1 : a \in \Z, \, \ord_p (a) = d \right\} |$. We want to show $N_d = \varphi (d)$ for every $d \mid p - 1$, $d > 0$.

  For any $1 \leqslant a \leqslant p - 1$, $\ord_p (a) \mid p-1$ by Fermat's little theorem and Proposition~\ref{prop:ord_div_k}. Thus 
\[
\sum_{\substack{d \, \mid \, p-1 \\ d \, > \, 0}} N_d = p - 1.
\]
By Gauss's identity (Theorem~\ref{thm:gauss_id})
\[
\sum_{\substack{d \, \mid \, p-1 \\ d \, > \, 0}} \varphi (d) = p - 1.
\]
Hence 
\begin{align}\label{eqn:eqn1_day29}
\sum_{\substack{d \, \mid \, p-1 \\ d \, > \, 0}} N_d = \sum_{\substack{d \, \mid \, p-1 \\ d \, > \, 0}} \varphi (d).
\end{align}
For any $d \mid p - 1$, $d > 0$, if we can prove $N_d \leqslant \varphi (d)$, Equation~\eqref{eqn:eqn1_day29} will force the summands on both sides to be equal, i.e., $N_d = \varphi (d)$.

Fix some $d \mid p - 1$, $d > 0$.
\begin{enumerate}[label=(\roman*)]
\item If $N_d = 0$, then $N_d < \varphi (d)$.\footnote{We do not know whether such an $a$ with order $d$ exists in the first place. Hence we have to consider $N_d = 0$ case.}
\item If $N_d > 0$, i.e., there is some $1 \leqslant a \leqslant p-1$ such that $\ord_p (a) = d$, by Proposition~\ref{prop:pow_cong_exp_cong} we have the same for powers of $a$ with order $d$ as for primitive roots in Corollary~\ref{cor:coprime_cong_prim_rt_power}, that 
\[
a, \, a^2, \,\ldots,\, a^{d - 1}, \, a^d
\]
are incongruent modulo $p$. Also for any $k \in \Z$, $k \geqslant 0$,
\[
(a^k)^d \equiv (a^d)^k \equiv 1 \bmod p.
\]
Therefore $a, a^2, \ldots, a^{d - 1}, a^d$ are incongruent solutions to $x^d - 1 \equiv 0 \bmod p$. By the corollary of Lagrange's theorem, Corollary~\ref{cor:cor_lagrange} implies $x^d - 1 \equiv 0 \bmod p$ has exactly $d$ solutions. Hence any solution to $x^d - 1 \equiv 0 \bmod p$ is congruent modulo $p$ to one of
\[
a, \, a^2, \,\ldots,\, a^{d - 1}, \, a^d.
\]
In particular, any integer $b$ coprime with $p$ with order $d$ modulo $p$ ($b$ satisfies $x^d - 1 \equiv 0 \bmod p$) is congruent to $a^k$ modulo $p$ for some $1 \leqslant k \leqslant d$. By Proposition~\ref{prop:ord_pow},
\begin{align}
d &= \ord_p (b) \\
  &= \ord_p (a^k) = \frac{\ord_p (a)}{\gcd (\ord_p (a), k)} \\
  &= \frac{d}{\gcd (d, k)} \iff \gcd (d, k) = 1.
\end{align}
There are $\varphi (d)$ choices for $1 \leqslant k \leqslant d$ with $\gcd (d, k) = 1$. Thus $N_d = \varphi (d)$. 
\end{enumerate}

This completes the proof.
\end{proof}
\begin{remark}
  In the language of abstract algebra, we proved the group of units $(\Z/p\Z)^\times$ in the field $\Z/p\Z$ is a cyclic group. More generally, any finite subgroup of the group of units $\F^\times$ in a field $\F$ is cyclic.
\end{remark}
\begin{remark}
  Note that  the proof gives no indication of how to find primitive roots modulo $p$. But as for computing all the primitive roots, at least we can trial and error to find the first one, then bootstrap\footnote{If $a$ is a primitive root modulo $p$, other primitive roots are congruent to powers of $a$ up to the $\varphi (p)$-th power. Choose $b > 1$ that is coprime with $\varphi (p)$. Compute $a^b$ modulo $p$, then $c \equiv a^b \bmod p$, $1 \leqslant c < p$ is another primitive root.} from the one we found.
\end{remark}

It is worth noting in the search of primitive roots, if we brute-force sweep all integers starting from 1, say we have the usual set-up, $a, n \in \Z$, $n > 0$ with $\gcd (a, n) = 1$. If $a$ is \underline{not} a primitive root, then $1 \leqslant a^k < n$ in the target range is not a primitive root either since
  \begin{align}
\hspace{50mm}    \ord_n (a^k) &= \frac{ \ord_n (a) }{ \gcd (\ord_n (a), k)} \\
                 & \leqslant \ord_n (a)  \\
                 & < \varphi (n). & \btfact{\text{(by $a$ is not a primitive root)}}
  \end{align}
  Hence the order of $a^k$ can never be as high as $\varphi (n)$.

Converse to the question we asked, given a prime $p$ whether a primitive root modulo $p$ always exists, E.~Artin asked given a primitive root $a$ modulo $p$, how many such $p$'s are there modulo which $a$ is a primitive root.
\begin{conjecture}[Artin, 1927]\label{conj:artin1927}
  For any nonsquare integer $a \neq -1$, there are infinitely many primes modulo which $a$ is a primitive root.
\end{conjecture}

For this open problem, Heath--Brown proved the following theorem,
\begin{theorem}[Heath--Brown, 1986]\label{thm:hb1986}
  Among all the nonsquare integers $a \neq -1$, there are at most two that do not satisfy Artin's conjecture.
\end{theorem}
But Heath--Brown theorem is ineffective to answer Artin's question since Theorem~\ref{thm:hb1986} does not tell us which two primitive roots $a$'s would fail Conjecture~\ref{conj:artin1927}.

\chapter[Lecture Thirty]{Day Thirty \hfill {\footnotesize \rm --- 07.04.2017}}

Our next goal is to prove the primitive root theorem,
% TODO: not sure at the moment if primitive root theorem <==> primitive element theorem, prof used the name primitive element theorem for the primitive root theorem in Strayer2001
\begin{restatable}[Primitive root theorem]{theorem}{primitiverootthm}\label{thm:prim_rt_thm}
  Let $n \in \Z$, $n \geqslant 2$. A primitive root exists modulo $n$ if and only if $n = 2, 4, p^k \text { or } 2p^k$ with $p$ an odd prime and $k \in \Z$ , $k \geqslant 1$. 
\end{restatable}

Recall from last time that if $n \in \Z$, $n \geqslant 2$ and $a$ is a primitive root modulo $n$, then any integer coprime with $n$ is congruent to a power $a^k$ modulo $n$ for some $1 \leqslant k \leqslant \varphi (n)$.
\begin{example}\label{ex:ex1_day30}
  Let $n \in \Z$, $n \geqslant 2$ and let $a \in \Z$ with $\gcd (a, n) = 1$. Assume that any $b \in \Z$ with $\gcd (b, n) = 1$ is congruent to $a^k$ modulo $n$ for some $k \in \Z$, $k \geqslant 1$. Show $a$ is a primitive root modulo $n$.
\end{example}
\begin{proof}
  By Proposition~\ref{prop:pow_cong_exp_cong}, we can convert the congruences of powers of $a$ to congruences of their exponents modulo $n$. Hence the set of integers $\left\{ a^k : k \in \Z, k \geqslant 1 \right\}$ defines no more than $\ord_n (a)$ many congruence classes modulo $n$.

  By our assumption, if every $b$ with $\gcd (b, n) = 1$ is congruent to $a^k$ for some $k \in \Z$, $k \geqslant 1$, then $\left\{ a^k : k \in \Z, k \geqslant 1 \right\}$ defines no less than $\varphi (n)$ many congruence classes modulo $n$. Thus $\ord_n (a) = \varphi (n)$, i.e., $a$ is a primitive root modulo $n$.
\end{proof}
% (X) TODO: the following shows up as corollary (but of what?) and the 3rd exercise problem in the exercise session in the last lecture
\begin{corollary}\label{cor:prim_rt_n_prim_rt_d}
\footnote{Proof was not given in the lecture. The reason why it was labeled as a corollary was probably due to the proposition: if $a$ is a primitive root modulo $p^k$ then $a$ is also a primitive root modulo $p$, whose proof looks like Example~\ref{ex:ex1_day30}: for any $b$  coprime with $p^k$, there is an integer $c > 0$ such that $a^c \equiv b \bmod p^k$ hence $a^c \equiv b \bmod p$ by reducing the modulus, i.e., $a$ is also a primitive root of $p$.
    
% TODO: not sure if the 2nd part of the proof is correct.
Also apply another proposition: if $a$ is a primitive root modulo $n_1n_2$, then $a$ is also a primitive root modulo $n_1$ and $n_2$. Proof. Assume the contrary, $a$ is not primitive modulo $n_1$ (or we can consider modulo $n_2$ similarly), then there exists an integer $0 < c < \varphi (n_1)$ such that $a^c \equiv 1 \bmod n_1$. Let $d = \gcd (n_1, n_2)$. We have $\varphi (n_1n_2) = \varphi (n_1) \varphi (n_2) \frac{d}{\varphi (d)} \geqslant \varphi (n_1) \varphi (n_2) > c \, \varphi (n_2)$, since $\varphi (d) = d \prod_{p \mid d} (1 - \frac{1}p)$ or the number of integers $1$ through $d$ is definitely greater than or equal to the number of integers up to $d$ that are coprime with $d$. Then we can raise $a$ to the $c\,\varphi(n_2)$-th power and see if $a^{c\,\varphi(n_2)} \equiv 1 \bmod n_1n_2$ holds. Indeed, by $a^c \equiv 1 \bmod n_1$, if we write $a^c = \ell n_1 + 1$ for some $\ell \in \Z$, then apply binomial theorem to $a^{c\,\varphi(n_2)} = (a^c)^{\varphi(n_2)} = (\ell n_1 + 1)^{\varphi(n_2)}$. We derived a contradiction that $a$ is not a primitive root modulo $n_1n_2$, since $a^{c\,\varphi(n_2)} \equiv 1 \bmod n_1n_2$ and $c\, \varphi(n_2) < \varphi (n_1n_2)$.}Let $a, n \in \Z$, $n \geqslant 2$ and $\gcd (a, n) = 1$. If $a$ is a primitive root modulo $n$, then $a$ is a primitive root modulo $d$ for any $d \mid n$, $d > 0$.
\end{corollary}

\begin{corollary}\label{cor:2k_no_prim_rt}
  Let $k \in \Z$ with $k \geqslant 3$. No primitive root exists modulo $2^k$.
\end{corollary}
\begin{proof}
  Let us assume there is a primitive root $a$ modulo $2^k$. Since $k \geqslant 3$, $8 \mid 2^k$. For any $b \in \Z$,
  \[
    \gcd (b, 8) = 1 \iff 2 \nmid b  \iff \gcd (b, 2^k) = 1.
  \]
  Our assumption and Example~\ref{ex:ex1_day30} imply, for any odd $b \in \Z$, $b \equiv a^m \bmod 2^k$ for some $m \in \Z$, $m \geqslant 1$. Then since $8 \mid 2^k$, by Corollary~\ref{cor:prim_rt_n_prim_rt_d}, $a$ is a primitive root modulo $8$. This contradicts the fact that $8$ has no primitive root.
\end{proof}

\chapter[Lecture Thirty-One]{Day Thirty-One \hfill {\footnotesize \rm --- 10.04.2017}}

\begin{proposition}\label{prop:mn_geq3_no_prim_rt}
  Let $m, n \in \Z$ with $m, n \geqslant 3$ and $\gcd (m, n) = 1$. Then no primitive root exists modulo $mn$.
\end{proposition}
\begin{proof}
  Note $\gcd (m, n) = 1$, $m, n > 0$, we have $\varphi (mn) = \varphi (m) \varphi (n)$. Let $d = \gcd \big(\varphi(m), \varphi(n) \big)$ and let $k = \frac{\varphi(mn)}d = \frac{\varphi(m)\varphi(n)}d$.

  Since $m, n \geqslant 3$, we have $2 \mid \varphi(m)$ and $2 \mid \varphi(n)$. So $2 \mid d$. In particular,
  \begin{align}
    k &= \frac{\varphi(mn)}d \\
      &< \varphi (mn).
  \end{align}
  Then for any $a \in \Z$ with $\gcd (a, m) = 1$,
  \begin{align}
    a^k &= a^{\frac{\varphi(m)\varphi(n)}d} \\
        &= \left( a^{\varphi(m)} \right)^{\frac{\varphi(n)}d} \\
        &\equiv 1 \bmod m,
  \end{align}
  by Euler's theorem. Similarly, $a^k \equiv 1 \bmod n$.

  By Chinese remainder theorem, $a^k \equiv 1 \bmod (mn)$ since $\gcd (m, n) = 1$. $k < \varphi(mn)$ implies $a$ is not a primitive root modulo $mn$. Since $a$ is arbitrary, no primitive root exists modulo $mn$.
\end{proof}
\begin{corollary}\label{cor:prim_rt_thm_easy_dir}
  Let $n \in \Z$, $n \geqslant 2$. If there is a primitive root modulo $n$, then $n = 2, 4, p^k \text { or } 2p^k$, where $p$ is an odd prime and $k \in \Z$, $k \geqslant 1$.
\end{corollary}
\begin{proof}
  We saw in Corollary~\ref{cor:2k_no_prim_rt} no primitive root exists modulo $2^k$ if $k \in \Z$, $k \geqslant 3$. If $n = 2^k$ and there is a primitive root modulo $n$ then $k = 1 \text{ or } 2$.

  Now assume $n$ has an odd prime divisor $p$, i.e., $p \mid n$ with $p$ odd prime. Write $n = p^km$ with $k \geqslant 1$, $p \nmid m$. Since $p$ is odd and $k \geqslant 1$, $p^k \geqslant 3$ and $\gcd (p^k, m) = 1$. By Proposition~\ref{prop:mn_geq3_no_prim_rt}, if there is a primitive root exists modulo $n = p^km$, then $m < 3$, i.e., $m = 1 \text { or } 2$.
\end{proof}

We now prove the converse of Corollary~\ref{cor:prim_rt_thm_easy_dir} is also true, hence we want to complete the proof of Theorem~\ref{thm:prim_rt_thm}. This direction is not trivial and requires several steps.

\begin{lemma}\label{lem:lem1_day31}
  Let $p$ be a prime and let $a, k \in \Z$ with $k \geqslant 1$ and $p \nmid a$. If $a$ is a primitive root modulo $p$, then
  \[
    \ord_{p^k} (a) = (p - 1)p^m
  \]
  for some $0 \leqslant m \leqslant k - 1$.
\end{lemma}
\begin{proof}
  Since $\varphi (p^k) = (p - 1)p^{k-1}$. By Corollary~\ref{cor:ord_div_phi}, $\ord_{p^k} (a) \mid \varphi (p^k) = (p - 1)p^{k-1}$.

  On the other hand, by Definition~\ref{def:order_mod_n}, we have
  \begin{align}
    a^{\ord_{p^k} (a)} &\equiv 1 \bmod p^k  \implies \\
    a^{\ord_{p^k} (a)} &\equiv 1 \bmod p. \\
  \end{align}
  By Proposition~\ref{prop:ord_div_k}, $\ord_p (a) \mid \ord_{p^k} (a)$. By our assumption, $a$ is a primitive root modulo $p$, so $\ord_p (a) = \varphi (p) = p - 1$. Therefore $p - 1 \mid \ord_{p^k} (a)$ and $\ord_{p^k} (a) \mid (p - 1)p^{k-1}$. It implies $\ord_{p^k} (a) = (p - 1)p^m$ for some $0 \leqslant m \leqslant k - 1$. (Consider the prime factorization of $\ord_{p^k} (a)$.)
\end{proof}
\begin{lemma}\label{lem:lem2_day31}
  Let $p$ be an odd prime and let $a \in \Z$ with $p \nmid a$. Assume that $a$ is a primitive root modulo $p$ and that $a^{p-1} \not\equiv 1 \bmod p^2$. Then for any $k \in \Z$ with $k \geqslant 2$, we have
  \[
    a^{(p-1)p^{k-2}} \not\equiv 1 \bmod p^k
  \]
  and $a$ is also a primitive root modulo $p^k$.
\end{lemma}
\begin{proof}
  By Lemma~\ref{lem:lem1_day31}, we know that $\ord_{p^k} (a) = (p-1)p^m$ for some $0 \leqslant m \leqslant p - 1$  and
  \begin{align}
    \text{$a$ is a primitive root modulo $p^k$} &\iff m = k - 1, \varphi (p^k) = (p - 1)p^{k-1}, \text{ in other words, } \\
    \text{$a$ is a not primitive root modulo $p^k$} &\iff m \leqslant k - 2 \\
                                                &\iff \ord_{p^k} (a) = (p - 1)p^m \text{ with some $0 \leqslant m \leqslant k - 2$} \\
                                                &\iff a^{(p-1)p^{k-2}} \equiv 1 \bmod p^k,
  \end{align}
  since $(p-1)p^m \mid (p-1)p^{k-2}$ when $m \leqslant k-2$.

  Assume $a$ is a primitive root modulo $p$. To prove $a$ is also a primitive root modulo $p^k$, it suffices to show that
  \[
    a^{(p-1)p^{k-2}} \not\equiv 1 \bmod p^k.
  \]
  We prove this by inducting on $k \geqslant 2$.
  When $k = 2$, this is true by our assumption. Assuming true for $k \geqslant 2$, we prove that it is true for $k + 1$. Since $p \nmid a$, by Euler's theorem, we have $a^{(p-1)p^{k-2}} \equiv 1 \bmod p^{k-1}$ (since $\varphi (p^{k-1}) = (p-1)p^{k-2}$). Therefore $a^{(p-1)p^{k-2}} = 1 + bp^{k-1}$ for some $b \in \Z$. Note that by our inductive hypothesis,
  \[
    a^{(p-1)p^{k-2}} \not\equiv 1 \bmod p^k,
  \]
  we have $p \nmid b$. (Otherwise $a^{(p-1)p^{k-2}} = 1 + \ell p \cdot p^{k-1} \equiv 1 \bmod p^k$ for $b = \ell p$ with some $\ell \in \Z$.)
  Substituting $a^{(p-1)p^{k-2}} = 1 + bp^{k-1}$,
  \begin{align}
    a{(p-1)p^{k-1}} &= \left( a^{(p-1)p^{k-2}} \right)^p \\
                    &= (1 + b p^{k-1})^p \\
                    &= \sum_{j = 0}^p \binom{p}{j} b^j p^{j(k-1)} \\
                    &= 1 + bp^k + \binom{p}{2} b^2 p^{2(k-1)} + \cdots + b^p p^{p(k-1)}.
  \end{align}
  Note that if $j \geqslant 3$, $k \geqslant 2$, then $j(k-1) \geqslant k + 1$. Therefore
  \[
    \sum_{j = 3}^p \binom{p}{j} b^j p^{j(k-1)} \equiv 0 \bmod p^{k+1}.
  \]
  Since $p$ is odd, $\frac{p-1}2$ is an integer and $\binom{p}2 b p^{2k - 2} = \left( \frac{p-1}2 \right) b p^{2k-1}$. Since $k \geqslant 2$, $2k - 1 \geqslant k + 1$. Therefore $\binom{p}2 b p^{2k - 2} \equiv 0 \bmod p^{k+1}$. Hence
  \begin{align}
    a^{(p-1)p^{k-1}} &\equiv 1 + b p^k \bmod p^{k+1} \\
                     &\not\equiv 1 \bmod p^{k+1},
  \end{align}
  since $p \nmid b$. By induction we proved the claim.
\end{proof}
\begin{remark}
  If we remove $p$ is \underline{odd} prime in the assumption of Lemma~\ref{lem:lem2_day31} so that $p$ could potentially be $2$ then the lemma would fail and the inductive proving technique is not effective since when we go from $p^2 = 4$ to $p^3 = 8$, we have seen that $8$ does not have primitive root. So we have to rule out $p = 2$.  
\end{remark}
\chapter[Lecture Thirty-Two]{Day Thirty-Two \hfill {\footnotesize \rm --- 12.04.2017}}

We continue our effort from last time to prove the primitive root theorem (Theorem~\ref{thm:prim_rt_thm}). What Lemma~\ref{lem:lem2_day31} does is, since we have known nice properties about $p$ and $p^2$, Lemma~\ref{lem:lem2_day31} will propagate those properties to higher powers of $p$.

\begin{proposition}\label{prop:prim_rt_exists_mod_p_pow}
  Let $p$ be an odd prime and let $k \in \Z$, $k \geqslant 1$. Then there exists a primitive root modulo $p^k$.
\end{proposition}
\begin{proof}
  By Lemma~\ref{lem:lem2_day31}, it suffices to show that there is an $a \in \Z$ with $p \nmid a$ such that $a$ is a primitive root modulo $p$ and $a^{p-1} \not\equiv 1 \bmod p^2$.

  For a prime $p$, by Corollary~\ref{cor:prim_rt_exists_mod_p} there is a $b \in \Z$ with $p \nmid b$ and $b$ is a primitive root modulo $p$.
\begin{enumerate}[label=(\roman*)]
\item We can take $a = b$ if $b^{p-1} \not\equiv 1 \bmod p^2$, then we are done.
\item Now assume $b^{p-1} \equiv 1 \bmod p^2$. We can choose $a = b + p$. ($a \equiv b \bmod p$ and $b$ is a primitive root modulo $p$ imply $a$ is a primitive root modulo $p$.) Then
  \begin{align}
\hspace{3.7cm}    a^{p-1} &= (b + p)^{p-1} \\
            &= \sum_{j = 0}^{p-1} \binom{p-1}j p^j b^{p-1-j} \\
            &= b^{p-1} + (p-1) p b^{p-2} + \sum_{j=2}^{p-1} \binom{p-1}j p^j b^{p-1-j} \\
            &\equiv 1 - p b^{p-2} \bmod p^2 & \btfact{\text{(by $b^{p-1} \equiv 1 \bmod p^2$)}} \\
            &\not\equiv 1 \bmod p^2,
  \end{align}
  since $p \nmid b \implies p^2 \nmid p b^{p-2}$.
\end{enumerate}

We completed the proof.
\end{proof}
\begin{proposition}\label{prop:prim_rt_exists_mod_2p_pow}
  Let $p$ be an odd prime. Then for any $k \in \Z$, $k \geqslant 1$, a primitive root exists modulo $2p^k$.
\end{proposition}
\begin{remark}
  Given Proposition~\ref{prop:prim_rt_exists_mod_p_pow}, we can expect this proposition to be true simply because $\varphi (2) = 1$. The intuition was it might not be very different when we consider the existence of a primitive root modulo $2p^k$ compared to that modulo $p^k$.
\end{remark}
\begin{proof}[Proof {\rm (of Proposition~\ref{prop:prim_rt_exists_mod_2p_pow})}]
  Let $b \in \Z$, $p \nmid b$ be a primitive root modulo $p^k$, whose existence is guaranteed by Proposition~\ref{prop:prim_rt_exists_mod_p_pow}.
  \begin{enumerate}[label=(\roman*)]
  \item If $b$ is odd, set $a = b$;
  \item If $b$ is even, set $a = b + p^k$.
  \end{enumerate}
  Since $p^k$ is odd, in either case, we have $a$ is odd, $p \nmid a$. Hence $\gcd (a, 2p^k) = 1$ and $a \equiv b \bmod p^k$. So $a$ is a primitive root modulo $p^k$.

  By Corollary~\ref{cor:ord_div_phi}, we have
  \[
    \ord_{2p^k} (a) \mid \varphi (2p^k) = (p-1)p^{k-1}.
  \]
  On the other hand,
  \begin{align}
    a^{\ord_{2p^k} (a)} & \equiv 1 \bmod 2p^k \\
    \implies a^{\ord_{2p^k} (a)} & \equiv 1 \bmod p^k \\
    \implies \hspace{1.1mm} \ord_{p^k} (a) &\mid \ord_{2p^k} (a),
  \end{align}
  But $\ord_{p^k} (a) = \varphi (p^k) = (p - 1)p^{k-1}$ since $a$ is a primitive root modulo $p^k$.

  So we showed that $\ord_{2p^k} (a) \mid (p - 1)p^{k-1}$ and $(p - 1)p^{k-1} \mid \ord_{2p^k} (a) $. Therefore,
  \[
    \ord_{2p^k} (a) = (p - 1)p^{k-1} = \varphi(2p^k),
  \]
  i.e., $a$ is a primitive root modulo $2p^k$.
\end{proof}
This finally completes the proof of primitive root theorem,
\primitiverootthm*
\begin{remark}
  The existence of a primitive root $a$ modulo $n$ is useful because it turns questions about the structure of integers coprime with $n$ modulo $n$ into questions about the powers of $a$, the \underline{one} particular integer.
\end{remark}

As an application, we can prove the generalized Euler's criterion.
\begin{definition}
  Let $a, n \in \Z$  with $n > 0$ and $\gcd (a, n) = 1$. For $k \in \Z$, $k \geqslant 1$, we say $a$ is a \textbf{$k$-th power residue modulo $n$} if there is an $x \in \Z$ such that
  \[
    x^k \equiv a \bmod n.
  \]
\end{definition}
\begin{example}\label{ex:kth_pow}
  Quick examples.
  \begin{itemize}
  \item $5^4 \equiv 4 \bmod 9$. So $4$ is a $4$th power residue modulo 9.
  \item $-1$ is a $3$rd power residue modulo $9$ since $-1 \equiv 8 \bmod 9$ and $8 = 2^3$.
  \item $a \in \Z$ with $a \nmid 9$ is a $6$th power residue modulo $9$ if and only if $a \equiv 1 \bmod 9$. Indeed, if $x \in \Z$ satisfies $x^6 \equiv a \bmod 9$, then $\gcd (x, 9) = 1$ since $\gcd (a, 9) = 1$. By Euler's theorem, $x^6 \equiv 1 \bmod 9$ since $6 = \varphi (9)$.
  \end{itemize}
\end{example}
\begin{theorem}\label{thm:generalized_euler_criterion}
  Let $a, n \in \Z$ with $n > 0$ and $\gcd (a, n) = 1$. Let $k \in \Z$ with $k \geqslant 1$. Assume there exists a primitive root modulo $n$, then $a$ is a $k$-th power residue modulo $n$ if and only if
  \[
    a^{\frac{\varphi(n)}d} \equiv 1 \bmod n,
  \]
  where $d  = \gcd (\varphi (n), k)$. Moreover, if this is the case, there are \textbf{exactly} $d$ incongruent solutions to $x^k \equiv a \bmod n$. 
\end{theorem}
\begin{proof}
  Let $r$ be a primitive root modulo $n$. Then there is an $\ell \in \Z$, $\ell \geqslant 1$ such that
  \[
    r^\ell \equiv a \bmod n.
  \]
  For any $x \in \Z$ with $\gcd (x, n) = 1$, there is an $m$, $m \geqslant 1$ such that
  \begin{align} \label{eqn:eqn1_day32}
    r^m \equiv x \bmod n.
  \end{align}
  Thus
  \begin{align}
    x^k \equiv a \bmod n & \iff r^{km} \equiv r^\ell \bmod n \\
                         & \iff \hspace{1mm} km \equiv \ell \hspace{1.5mm}\bmod \varphi (n), \label{eqn:eqn2_day32}
  \end{align}
  since $r$ is a primitive root modulo $n$ and by Proposition~\ref{prop:pow_cong_exp_cong}, the exponents of $r$ are congruent modulo the order of $r$ modulo $n$. It is worth noting, originally, $x$ depends on $m$ chosen in Equation~\eqref{eqn:eqn1_day32}. What we have done here is turning a problem of solving for $x$ into an equation of $m$ and solving for $m$ itself. The congruence in Equation~\eqref{eqn:eqn2_day32} has a solution in $m$ if and only if $d \mid \ell$, where $d = \gcd (\varphi (n), k)$; and if this is the case, there are precisely $d$ incongruent solutions modulo $\varphi (n)$ by Theorem~\ref{thm:num_soln_lin_cong}. These $d$ incongruent solutions correspond to the $d$ incongruent solutions $x = r^m $ to $x^k \equiv a \bmod n$. (The moral is we convert difficult congruences into \underline{linear} congruences and solve linear congruences first then convert the solution of linear congruences back to the original difficult congruences.)

  Since $r^m \equiv r^{m'} \bmod n$ if and only if $m \equiv m' \bmod \varphi (n)$ by Proposition~\ref{prop:pow_cong_exp_cong}, it only remains to show that $d \mid \ell$ if and only if $a^{\frac{\varphi (m)}d} \equiv 1 \bmod n$. Indeed,
  \begin{align}
    d \mid \ell &\iff \ell = ed \text{ for some $e \in \Z$} \\
                &\iff \varphi (n) \ell \hspace{.8mm}= \varphi (n) ed \\
                &\iff \frac{\varphi (n)}d \ell = \varphi (n) e \\
                &\iff \frac{\varphi (n)}d \ell \equiv 0 \bmod \varphi (n) \\
\label{eqn:eqn3_day32}                &\iff r^{\frac{\varphi (n)}d \ell} \hspace{0.5mm} \equiv 1 \bmod n \\
                &\iff a^{\frac{\varphi (n)}d} \hspace{1.5mm}\equiv 1 \bmod n,
  \end{align}
  since $a \equiv r^\ell \bmod n$. Note the equivalent in Equation~\eqref{eqn:eqn3_day32}, ($\Rightarrow$) direction does not need $r$ to be a primitive root (Euler's theorem), however ($\Leftarrow$) direction requires $r$ to be a primitive root modulo $n$ (Proposition~\ref{prop:ord_div_k}).
\end{proof}
\begin{remark}
  The proof of Theorem~\ref{thm:generalized_euler_criterion} is just repeatedly applying Proposition~\ref{prop:pow_cong_exp_cong}.
\end{remark}
\begin{example}[Example~\ref{ex:kth_pow}, continued]
  Note that $\gcd (\varphi (9), 4) = \gcd (6, 4) = 2$, and $4^{\frac{\varphi (9)}2} = 4^{\frac{6}2} = 4^3 \equiv 1 \bmod 9$. So $4$ is a $4$th power residue modulo $9$. There are $2$ incongruent solutions to $x^4 \equiv 4 \bmod 9$; they are $4$ and $5$.
\end{example}
Recall Euler's criterion,
\eulercriterion*
We can consider Theorem~\ref{thm:euler_criterion} as a corollary of Theorem~\ref{thm:generalized_euler_criterion}.
\begin{corollary}
  Let $p$ be an odd prime and let $a \in \Z$, $p \nmid a$. Then $a$ is a quadratic residue modulo $p$ if and only if
  \[
    a^{\frac{p-1}2} \equiv 1 \bmod p.
  \]
  Moreover, if this is the case, there are \textbf{exactly} two incongruent solutions to $x^2 \equiv a \bmod p$.
\end{corollary}
\begin{proof}
  Since $p$ is a prime, primitive roots exist modulo $p$. So we can apply the generalized Euler's criterion (Theorem~\ref{thm:generalized_euler_criterion}) with $k = 2$, in which case, $\gcd (\varphi (n), k) = \gcd (\varphi (p), 2) = 2$.
\end{proof}
\subsection*{Exercise}
\begin{enumerate}
% ko80, no. 92
\item Let $p$ be an odd prime. Solve $x^{p-1} \equiv 1 \bmod p^s$ with $s \in \Z$, $s \geqslant 1$ for all $x \in \Z$.
% ko80, no. 96
\item Let $a, n \in \Z$ with $n = 2^a + 1$, $a > 1$. Show that $n$ is prime if and only if $3^{\frac{n-1}2} \equiv 1 \bmod n$.
\end{enumerate}
\chapter[Lecture Thirty-Three]{Day Thirty-Three \hfill {\footnotesize \rm --- 14.04.2017}}

\underline{A}pp\underline{lication of }p\underline{rimitive roots in cr}yp\underline{to}g\underline{ra}p\underline{h}y \\

Recall RSA in Lecture~\ref{chp:RSA}. It is asymmetric cryptosystem, i.e., person sending messages and person receiving messages do not share the same information. It is also public in the sense that any one can send messages.

Sometimes it is useful or enough to use a \emph{symmetric} cryptosystem, i.e., where both parties have the same secret information for encrypting and decrypting.

Q\underline{uestion}: How to generate or share this secret information?

The Diffie-Hellman key exchange is a secure way to share information over an insecure channel.
\begin{remark}
The Advanced Encryption Standard (AES) is a standard cipher that allows one to encrypt and decrypt messages using a secret key, which is an integer both parties have, i.e., different keys give encryptions. Using this, the question becomes how to share a secret integer over an insecure channel.
\end{remark}
\begin{remark}
This is often used in conjunction with RSA.
\end{remark}
\underline{Diffie--Hellman ke}y\underline{ exchan}g\underline{e set-u}p (Diffie--Hellman, 1976; Williamson, 1974\footnote{Not declassified until 1997.}): 
\begin{itemize} 
\item Alice and Bob publicly agree on a large prime $p$ and a primitive root $r$ modulo $p$;
\item Alice chooses a secret positive integer $k$ and sends Bob $1 \leqslant A \leqslant p - 1$ with 
\[
A \equiv r^k \bmod p;
\] 
\item Bob chooses a secret positive integer $\ell$ and sends Alice $1 \leqslant B \leqslant p - 1$ with
  \[
    B \equiv r^\ell \bmod p;
  \]
\item Alice computes $B^k$ modulo $p$ to get
  \[
    r^{k \ell} \equiv B^k \bmod p;
  \]
\item Bob computes $A^\ell$ modulo $p$ to get
  \[
    r^{\ell k} \equiv A^k \bmod p;
  \]
\item The unique $1 \leqslant s \leqslant p - 1$ with $s \equiv r^{k \ell} \bmod p$ is the shared secret key.
\end{itemize}
\begin{remark}
  Why is it secure? Given $p, r$ and $A, B$, to compute $s$, we need to find either $k$ or $\ell$ such that
  \[
    r^k \equiv A \bmod p \text{ or } r^\ell \equiv B \bmod p.
  \]
  But this is a hard problem, called \textbf{discrete logarithm problem} in computer science.
\end{remark}
A naive algorithm: Compute $r^k$ for each $1 \leqslant k \leqslant p - 1$ (or $0 \leqslant k \leqslant p - 2$). The run time is about $p - 1$ steps. This run time grows exponentially as the \#digits of $p$ increases.

A better algorithm is \emph{Baby-Step-Giant-Step algorithm}. It runs in $\mathcal{O}(\sqrt{p-1})$ steps, which is much better than the naive approach. The downside is it needs a lot of storage. The algorithm is based on the observation,
\begin{proof}[Observation]\renewcommand*{\qedsymbol}{}
  Let $r$ be a primitive root modulo prime $p$. Let $a \in \Z$ with $p \nmid a$. Then $r^k \equiv a \bmod p$ for some $0 \leqslant k \leqslant p - 2$, $k \in \Z$. Let $m = \lceil \, \sqrt{p-1} \, \rceil$, i.e., the smallest integer $\geqslant \sqrt{p-1}$, we can write $k = im + j$ with $0 \leqslant i, j \leqslant m$. Then we have
  \begin{align}
    r^k \equiv a \bmod p &\iff r^{im+j} \equiv a \bmod p \\
                         &\iff r^{i(p-1-m)} r^{im+j} \equiv r^{i(p-1-m)} a \bmod p \\
                         &\iff r^{i(p-1)} r^j \equiv r^{i(p-1-m)} a \bmod p \\
                         &\iff r^j \equiv r^{i(p-1-m)} a \bmod p. \label{eqn:eqn1_day33}
  \end{align}
  The equivalence in Equation~\eqref{eqn:eqn1_day33} is due to Fermat's little theorem. The idea of the algorithm is to precompute $r^j \bmod p$ for $0 \leqslant j \leqslant m$ and check if the congruence in Equation~\eqref{eqn:eqn1_day33} holds.
\end{proof}

\underline{Al}g\underline{orithm} (Baby-step-giant-step algorithm for discrete logarithm problem): 
\begin{enumerate}[label=\arabic*.]
\item Set $m = \sqrt{p-1}$;
\item Compute $r^j \bmod p$ for each $0 \leqslant j \leqslant m - 1$. Store $\big(j, (r^j \bmod p) \big)$;
\item Compute $r^{p-1-m} \bmod p$;
\item Initialize $i = 0$ and $b = a$;
\item Check if $b \equiv r^j \bmod p$ for $0 \leqslant j \leqslant m - 1$. If so, return $im + j$; if not,
\item Replace $b$ with $b r^{p-1-m} \bmod p$, and $i$ with $i + 1$ then go back to the previous step.
\end{enumerate}
\begin{remark}
  As for the run time, if we assume step 5 is negligible (which we can), we do $m$ computations in step 2, one computation in step 3, and at most $m$ computations in step 6. Therefore, in total the \#steps, $2m + 1 = \mathcal{O}(\sqrt{p-1})$. It is not running in polynomial time but much better than the naive algorithm.
\end{remark}
\begin{remark}
  There is no known polynomial time (in the \#digits of $p$) algorithm for the discrete logarithm problem. The best known so far is the \emph{number field sieve}\footnote{Number field, for example, the set of Gauss integers $\{ a + i b : a, b \in \Z \}$ with addition and multiplication, where $i^2 = -1$.}, which is also the best known algorithm for factoring integers. It is more complicated and requires more theory to describe (Algebraic Number Theory e.g., at least at the level of Math 530).

  For example, in 2016, Kleinjung \emph{et al.}~\cite{Kleinjung2017} did discrete logarithm computation for a $768$ bit prime using number field sieve. The computation took about a year. 
\end{remark}

\chapter[Lecture Thirty-Four]{Day Thirty-Four \hfill {\footnotesize \rm --- 17.04.2017}}

\underline{A}pp\underline{lication of }p\underline{rimitive roots}: Miller--Rabin test (again) \\

Recall Miller--Rabin test is an efficient way to determine whether or not an integer is a probable prime. See Lecture~\ref{chp:miller_rabin} for the first introduction.

\underline{Miller--Rabin test}: Given $n$, an odd positive integer, write $n - 1 = 2^km$ with $k \geqslant 1$ and $m$ odd, $k, m \in \Z$. Choose a random $1 \leqslant a \leqslant n - 1$, $a \in \Z$. If $a^m \not\equiv 1 \bmod n$ \underline{and} $a^{2^r m} \not\equiv -1 \bmod n$ for each $0 \leqslant r \leqslant k - 1$, $r \in \Z$, then $n$ is composite. And we say $n$ fails the Miller--Rabin test with this choice of $a$ if $a^m \equiv 1 \bmod n$ \underline{or} $a^{2^r m} \equiv -1 \bmod n$ for some $0 \leqslant r \leqslant k - 1$. But the chances of an odd composite $n$ passing a single Miller--Rabin test are less than a quarter, hence the test is effective because of the following theorem, 
\begin{theorem}\label{thm:miller_rabin}
  Let $n > 9$ be an odd composite integer. Write $n - 1 = 2^k m$ with $k \geqslant 1$ and $m$ odd, $k, m \in \Z$. Let
  \[
  \sB = \left\{ 1 \leqslant a \leqslant n - 1: a^m \equiv 1 \bmod n \text{ or } a^{2^r m } \equiv -1 \bmod n, a \in \Z \text{ for some $0 \leqslant r \leqslant k - 1$, $r \in \Z$} \right\}.
  \]
  Then $\frac{\vert \sB \vert}{n - 1} < \frac{1}4$.
\end{theorem}
The immediate consequence of Theorem~\ref{thm:miller_rabin} is that the chance of an odd composite number $n > 9$ passing 10 (independent) Miller--Rabin tests is less than $\frac{1}{4^{10}} = \frac{1}{1048576} < 0.0001\%$. From computational point of view, primality test and factoring integer are quite different. For example, in \verb|Mathematica|, given $n = 101^{100} + 1$, \verb|PrimeQ[n]| returns \verb|false| instantly. However \verb|FactorInteger[n]| hangs.

\begin{remark}
  Miller--Rabin test is a probabilistic primality test. However, assuming some unsolved problems in number theory (generalized Riemann hypothesis, GRH) one can turn the Miller--Rabin test into a deterministic primality test running in polynomial time.
\end{remark}
\begin{remark}
  In 2002, Agrawal--Kayal--Saxena~\cite{Agrawal2004} developed a deterministic polynomial time primality test without assuming any unsolved problems (such as GRH).
\end{remark}

The proof runs for several lectures; it is a long proof.

\begin{proof}[Proof {\rm (of Theorem~\ref{thm:miller_rabin})}]\renewcommand*{\qedsymbol}{}
  Note any $a \in \sB$ must satisfy $\gcd (a, n) = 1$. Indeed, if $e \mid n$ and $e \mid a$, $e \in \Z$, then $e \mid a^m$ and $e \mid a^{2^r m}$ for each $0 \leqslant r \leqslant k - 1$. Therefore $e \mid 1$ or $e \mid -1$ since $a \in \sB$.

  Also since $n$ is composite, $\varphi (n) < n - 1$. It suffices to show $\frac{\vert \sB \vert}{\varphi (n)} \leqslant \frac{1}4$. (To be continued.)
\end{proof}

\chapter[Lecture Thirty-Five]{Day Thirty-Five \hfill {\footnotesize \rm --- 19.04.2017}}
\begin{proof}[Proof {\rm (of Theorem~\ref{thm:miller_rabin}, continued)}]\renewcommand*{\qedsymbol}{}
Since $n$ is odd, any prime $p \mid n$ is also odd. Let $\ell$ be the largest positive integer such that $2^\ell \mid p - 1$ for each $p \mid n$. Let 
\[
\sC = \left\{ 1 \leqslant a \leqslant n - 1 : \gcd (a, n) = 1 \text{ and } a^{2^{\ell - 1}m} \equiv \pm 1 \bmod n \right\}.
\]
\begin{claim}\label{clm:clm1_day35}
  $\sB \subseteq \sC$.
\end{claim}
\begin{remark}[of Claim~\ref{clm:clm1_day35}]
  $r$ in the set $\sB$ is a variable, making counting $a \in \sB$ difficult. We converted $\sB$ to $\sC$ and $c \in \sC$ is easier to count.
\end{remark}
\begin{proof}[Proof {\rm (of Claim~\ref{clm:clm1_day35})}]\renewcommand*{\qedsymbol}{\ensuremath{\square}}
  We can consider two cases of $a \in \sB$. % Take $a \in \sB$.
  \begin{enumerate}[label=(\roman*)]  
  \item If $a^m \equiv 1 \bmod n$, then 
    \begin{align}
      a^{2^{\ell - 1}m} &= (a^m)^{2^{\ell - 1}} \\
                        &\equiv (1)^{2^{\ell - 1}} \bmod n\\
                        &\equiv 1 \hspace{8.7mm}\bmod n,
    \end{align}
    i.e., $a \in \sC$.
  \item Now assume $a^{2^r m} \equiv -1 \bmod n$ for some $0 \leqslant r \leqslant k - 1$. Let $p \mid n$ be an odd prime ($n$ is odd) and write\footnote{$\gcd (a, n) = 1$ implies $\gcd (a, p) = 1$. $p$ is prime so $\ord_p (a) \mid p - 1$ by Corollary~\ref{cor:ord_div_phi}. Note $p - 1$ is even if $p > 2$.} $\ord_p (a) = 2^s d$ with $s \geqslant 1$ and $2 \nmid d$. Reducing the modulus,
    \begin{align}
      a^{2^r m} &\equiv -1 \bmod n \implies \\
      a^{2^r m} &\equiv -1 \bmod p,
    \end{align}
    since $p \mid n$. Therefore $\ord_p (a^{2^r m}) = \ord_p (-1)$. But $\ord_p (-1) = 2$ if $p$ is odd.

    On the other hand,
    \begin{align}
      2 &= \ord_p (a^{2^r m}) \\
        &= \frac{\ord_p (a)}{\gcd (\ord_p (a), 2^r m)} \\
        &= \frac{2^s d}{\gcd (2^s d, 2^r m)} \iff  2 \gcd (2^s d, 2^r m) = 2^s d. \label{eqn:eqn1_day35} 
    \end{align}
    Using the formula for $\gcd$ by Proposition~\ref{prop:gcd_lcm}, Equation~\eqref{eqn:eqn1_day35} implies\footnote{We focus on the power of $2$.},
    \[
      2^{1 + \min \{ s, r \}} = 2^s.
    \]
    Therefore $s = r+1$. $\ord_p (a) = 2^{r+1} d$ with $2 \nmid d$.
    By Corollary~\ref{cor:ord_div_phi}, $\ord_p (a) \mid p - 1$, it implies $2^{r+1} \mid p - 1$. By construction of $\ell$, we have $\ell \geqslant r + 1$. Then
    \begin{align}
      a^{2^{\ell - 1}m} &= (a^{2^r m})^{2^{\ell - 1 - r}} \\
                        &\equiv (-1)^{2^{\ell - 1 - r}} \bmod n \\
                        &\equiv \pm 1 \hspace{11.7mm}\bmod n.
    \end{align}
    Note in the last congruence $(-1)^{2^{\ell - 1 - r}} \equiv -1 \bmod n $ only when $\ell = r+1$.
    In this case, we have $a \in \sC$ as well.
  \end{enumerate}
  This finishes the verification of the claim, i.e., $\sB \subseteq \sC$.
\end{proof}
By Claim~\ref{clm:clm1_day35}, $\vert \sB \vert \leqslant \vert \sC \vert$. Hence, it suffices to show $\frac{\vert \sC \vert}{\varphi (n)} \leqslant \frac{1}4$.

Now we want to compute $\vert \sC \vert$. Consider the prime factorization of $n$,
\[
  n = p_1^{e_1}p_2^{e_2} \cdots p_j^{e_j},
\]
with $p_i$ odd prime and $e_i \geqslant 1$ for each $1 \leqslant i \leqslant j$, $j \in \Z$, $j > 0$.

Consider the congruence
\begin{align}
  x^{2^{\ell - 1}m} \equiv -1 \bmod p_i^{e_i}. \label{eqn:eqn2_day35}
\end{align}
For each $1 \leqslant i \leqslant j$, there exists a primitive root modulo $p_i^{e_i}$ since $p_i$ is an odd prime by the primitive root theorem (Theorem~\ref{thm:prim_rt_thm}). So we can apply the generalized Euler's criterion (Theorem~\ref{thm:generalized_euler_criterion}) to congruence Equation~\eqref{eqn:eqn2_day35}, i.e., a solution to congruence Equation~\eqref{eqn:eqn2_day35} exists if and only if 
\begin{align}
(-1)^{\displaystyle \frac{\varphi (p_i^{e_i})}{\gcd \big(\varphi (p_i^{e_i}), 2^{\ell - 1}m\big)}} \equiv 1 \bmod p_i^{e_i}, \label{eqn:eqn3_day35}
\end{align}
and if this is the case, it has exactly $\gcd \big(\varphi (p_i^{e_i}), 2^{\ell - 1}m \big)$ many incongruent solutions. 

Note if $\frac{\varphi (p_i^{e_i})}{\gcd \big(\varphi (p_i^{e_i}), 2^{\ell - 1}m\big)}$ is a multiple of $2$, then congruence in Equation~\eqref{eqn:eqn3_day35} always holds. Since $2^\ell \mid p_i - 1$, we have $2^\ell \mid (p_i - 1)p_i^{e_i - 1}  = \varphi (p_i^{e_i})$. Then 
\begin{align}
  \gcd \big(\varphi (p_i^{e_i}), 2^{\ell - 1}m \big) &= \gcd ((p_i - 1)p_i^{e_i - 1}, 2^{\ell - 1}m) \\
                                            &= \gcd (p_i - 1, 2^{\ell - 1}m),
\end{align}
since $m \mid n - 1$ and $p_i \mid n$ imply $p_i \nmid m$; (Otherwise $p_i \mid m \implies p_i \mid n-1$, and with $p_i \mid n$, it implies $p_i \mid n - (n - 1) = 1$, a contradiction.) Also $p_i$ is odd, $p_i \nmid 2$, hence $p_i \nmid 2^{\ell - 1}m$.

Further
\[
  \gcd (p_i - 1, 2^{\ell - 1}m) = 2^{\ell - 1} \gcd (p_i - 1, m),
\]
since $2^\ell \mid p_i - 1 \implies 2^{\ell - 1} \mid p_i - 1$.
Therefore the exponent of $(-1)$ in Equation~\eqref{eqn:eqn3_day35},
\[
\frac{\varphi (p_i^{e_i})}{\gcd \big(\varphi (p_i^{e_i}), 2^{\ell - 1}m\big)} = \frac{(p_i - 1)p_i^{e_i - 1}}{2^{\ell - 1} \gcd (p_i - 1, m)} \text{ is divisible by $2$,}
\]
since $p_i - 1$ in the numerator has $\ell$ copies of $2$ and $\gcd (p_i - 1, m)$ in the denominator has no copies of $2$. Thus the congruence in Equation~\eqref{eqn:eqn3_day35} always holds, i.e., congruence Equation~\eqref{eqn:eqn2_day35} has exactly $2^{\ell - 1} \gcd (p_i - 1, m)$ incongruent solutions by generalized Euler's criterion. 

By Chinese remainder theorem, each choice of solution modulo $p_i^{e_i}$ for each $1 \leqslant i \leqslant j$ corresponds to a unique solution modulo $n = \prod_{i = 1}^j p_i^{e_i}$ to congruence
\begin{align}
  x^{2^{\ell - 1}m} \equiv -1 \bmod n. \label{eqn:eqn4_day35}
\end{align}
The number of incongruent solutions to congruence Equation~\eqref{eqn:eqn4_day35} is 
\[
\prod_{i = 1}^j 2^{\ell - 1} \gcd (p_i - 1, m).
\]
A similar argument can be said about congruence 
\begin{align}
  x^{2^{\ell - 1}m} \equiv +1 \bmod n. \label{eqn:eqn5_day35}
\end{align}
There are also
\[
\prod_{i = 1}^j 2^{\ell - 1} \gcd (p_i - 1, m)
\]
many incongruent solutions to congruence Equation~\eqref{eqn:eqn5_day35}. So the size of $\sC$ is 
\[
\vert \sC \vert  = 2 \prod_{i = 1}^j 2^{\ell - 1} \gcd (p_i - 1, m).
\]
Since 
\begin{align}
  \frac{\vert \sC \vert}{\varphi (n)} &\leqslant \frac{1}4 \\
\iff \frac{ 2 \prod_{i = 1}^j 2^{\ell - 1} \gcd (p_i - 1, m) }{ \prod_{i = 1}^j (p_i - 1) p_i^{e_i -1} } &\leqslant \frac{1}4 \\
\iff \hspace{4.3mm} 8 \prod_{i = 1}^j 2^{\ell - 1} \gcd (p_i - 1, m) &\leqslant \prod_{i = 1}^j (p_i - 1) p_i^{e_i -1},
\end{align}
next we need to show 
\[
8 \prod_{i = 1}^j 2^{\ell - 1} \gcd (p_i - 1, m) \leqslant \prod_{i = 1}^j (p_i - 1) p_i^{e_i -1}
\]
for all $n > 9$.

\end{proof}
\chapter[Lecture Thirty-Six]{Day Thirty-Six \hfill {\footnotesize \rm --- 21.04.2017}}
\begin{proof}[Proof {\rm (of Theorem~\ref{thm:miller_rabin}, continued)}]
Recall from last time, we want to show 
\[%begin{align}
8 \prod_{i = 1}^j 2^{\ell - 1} \gcd (p_i - 1, m) \leqslant \prod_{i = 1}^j (p_i - 1) p_i^{e_i -1}
\]%end{align}
for all $n > 9$. Assume the contrary, i.e., 
\begin{align}
\label{eqn:eqn1_day36} 8 \prod_{i = 1}^j 2^{\ell - 1} \gcd (p_i - 1, m) > \prod_{i = 1}^j (p_i - 1) p_i^{e_i -1}.
\end{align}
Since $2^\ell \mid p_i - 1$ for each $1 \leqslant i \leqslant j$, $\gcd (p_i - 1, m) \mid p_i - 1$ for each $1 \leqslant i \leqslant j$, and $\gcd (p_i - 1, m)$ is odd (as $m$ is odd), we have 
\[
2^\ell \gcd (p_i - 1, m) \mid p_i - 1.
\]
In particular, $p_i - 1 \geqslant 2^\ell \gcd (p_i - 1, m)$. Then the inequality in Equation~\eqref{eqn:eqn1_day36} implies
\begin{align}
8 \prod_{i = 1}^j 2^{\ell - 1} \gcd (p_i - 1, m) &> \prod_{i = 1}^j (p_i - 1) p_i^{e_i -1} \\
                                                 &\geqslant \prod_{i = 1}^j 2^\ell \gcd (p_i - 1, m) p_i^{e_i -1} \\
  \implies 8 &> \prod_{i = 1}^j 2 p_i^{e_i - 1}, \label{eqn:eqn2_day36}
\end{align}
i.e., we must have $j \leqslant 2$.
\begin{enumerate}[label=(\roman*)]
\item First assume $j = 2$. We can write $n = p_1^{e_1}p_2^{e_2}$ with $p_1, p_2$ distinct odd primes, $e_1, e_2 \geqslant 1$, $e_1, e_2 \in \Z$. Plug $n$ into the inequality in Equation~\eqref{eqn:eqn2_day36}, 
\[
  8 > 4 p_1^{e_1 - 1} p_2^{e_2 - 1}.
\]
It implies $e_1 = e_2 = 1$. Then the inequality in Equation~\eqref{eqn:eqn1_day36} becomes
\[
2 > \frac{p_1 - 1}{2^\ell \gcd (p_1 - 1, m)} \cdot \frac{p_2 - 1}{2^\ell \gcd (p_2 - 1, m)}.
\]
This means $p_i - 1 = 2^\ell \gcd (p_i - 1, m)$ for $i = 1, 2$. Set $d_i = \gcd (p_i - 1, m)$. Note 
$d_i \mid m$, so 
\begin{align}
  p_1 p_2 &= n \\
          &= 1 + 2^k m \\
          &\equiv 1 \bmod d_i \text{ for $i = 1, 2$}.
\end{align}
And also $p_i - 1 = 2^\ell d_i$ for $i = 1, 2$ as well, i.e., $p_i \equiv 1 \bmod d_i$. But 
\begin{align}
\left\{ \begin{array}{r}
p_1 \equiv 1 \bmod d_1 \\[1mm]
p_1 p_2 \equiv 1 \bmod d_1 
        \end{array} \right. \implies p_2 \equiv 1 \bmod d_1.
\end{align}
Hence $d_1 \mid p_2 - 1$. Also note $d_1$ is odd, $d_1 \mid p_2 - 1 = 2^\ell d_2 \implies d_1 \mid d_2$. 

A symmetric argument says $d_2 \mid d_1$. Therefore, $d_1 = d_2$. Consequently, $p_1 = 1 + 2^\ell d_1 = 1 + 2^\ell d_2 = p_2$, a contradiction.
\item Now assume $j = 1$. We write $n = p^e$ with $p$ odd prime and $e \geqslant 2$ (since $n$ is composite). Then the inequality in Equation~\eqref{eqn:eqn2_day36} becomes
  \[
    4 > p^{e - 1}.
  \]
  With $p$ odd prime and $e \geqslant 2$, this is only possible if $p = 3$ and $e = 2$. Then $n = 3^2 = 9$, a contradiction since $n > 9$ by our assumption.
\end{enumerate}
Finally we completed the proof of Theorem~\ref{thm:miller_rabin}.
\end{proof}

We will move to our next new topic.\\

\underline{Dio}p\underline{hantine e}q\underline{uations}\\

A \textbf{Diophantine equation} is a polynomial equation in two or more variables that we want to solve for integer solutions.
\begin{example}
  Let $a, b, c \in \Z$. We saw in Lecture~\ref{chp:diophantine_two_var}, Proposition~\ref{prop:diophantine_two_var} that the linear Diophantine equation in two variables
  \[
    a x + b y = c
  \]
  has solutions in $x, y \in \Z$ if and only if $\gcd (a, b) \mid c$, and if this is the case, we can find all solutions. 
\end{example}

Q\underline{uestion}: What about nonlinear Diophantine equations?

The answer is it depends on the specific settings. However, a useful trick is based on the following observation.
\begin{proof}[Observation]\renewcommand*{\qedsymbol}{}
  If a Diophantine equation has solutions in $\Z$, then it has solutions modulo $n$ for any $n \in \Z$, $n \geqslant 1$.
\end{proof}

\begin{example}
  Consider the equation $3x^2 - y^2 = 1$. If $x, y \in \Z$ exist such that they solve the Diophantine equation, then
  \begin{align}
    3x^2 - y^2 &\equiv 1 \bmod 3 \\
    -y^2  &\equiv 1 \bmod 3 \\
    y^2 &\equiv 2 \bmod 3.
  \end{align}
  However, $2$ is not a quadratic residue modulo $3$. Hence no integer solution in $y$ exists.
\end{example}

\begin{remark}
  This trick however does not always work for showing nonexistence of solutions!
\end{remark}

\begin{example}
  Consider the equation $(x^2 - 2)(x^2 - 17)(x^2 - 34) = 0$. It has no integer solutions but it has solutions modulo $n$ for any $n \geqslant 1$.
\end{example}
\begin{proof}
  For $n = 1$, it is trivial since any integer is congruent to $0$ modulo $1$.\footnote{In computer programming, this fact can be used to set up an integral test, i.e., to test if a variable assumes an integer value, simply check if that value modulo $1$ is zero.} We can assume $n \geqslant 2$.

  Let $n = \prod_{p \mid n} p^e$ be the prime factorization of $n$. By Chinese remainder theorem, the congruence equation has a solution modulo $n$ if and only if it has a solution modulo $p^e$ for each $p \mid n$,
  \begin{align}
    (x^2 - 2)(x^2 - 17)(x^2 - 34) &\equiv 0 \bmod n \\
    \iff    (x^2 - 2)(x^2 - 17)(x^2 - 34) &\equiv 0 \bmod p^e.
  \end{align}
   It further suffices to show \underline{at least} one of
  \begin{align}
\label{eqn:eqn3_day36}    x^2 &\equiv 2 \hspace{1.7mm}\bmod p^e \\
\label{eqn:eqn4_day36}    x^2 &\equiv 17 \bmod p^e \\
\label{eqn:eqn5_day36}    x^2 &\equiv 34 \bmod p^e 
  \end{align}
  has a solution. For a prime power $p^e$ with $e \geqslant 1$, $e \in \Z$ and $a \in \Z$ with $p \nmid a$, $x^2 \equiv a \bmod p^e$ is solvable if and only if
  \begin{align}\label{eqn:cong_mod_prime_pow}
    \begin{split}
    \left\{ \begin{array}{ll}
              \hspace{1.7mm} a \equiv 1 \bmod 8 & \text{if $p = 2$ and $e \geqslant 3$,} \\[1mm]
              x^2 \equiv a \bmod p \text{ is solvable} & \text{if $p > 2$.} \end{array} \right. 
        \end{split} 
 \end{align}
  For details, see~\cite{Strayer2001}*{\S 4.1, Problem 11}.

  We can look at three cases.
  \begin{enumerate}[label=(\roman*)]
  \item If $p = 2$, then congruence Equation~\eqref{eqn:eqn4_day36} has a solution for $e \geqslant 3$ since $17 \equiv 1 \bmod 8$. When $e = 1$ or $2$, quick inspection reveals both $x^2 \equiv 17 \bmod 2$ and $x^2 \equiv 17 \bmod 4$ are solvable.
  \item If $p = 17$, then $x^2 \equiv 2 \bmod 17$ is solvable by the second supplement to quadratic reciprocity (Theorem~\ref{thm:qr_2nd_supp}) since $17 \equiv 1 \bmod 8$. Therefore by the equivalence in Equation~\eqref{eqn:cong_mod_prime_pow}, the congruence Equation~\eqref{eqn:eqn3_day36} is always solvable when $p = 17$.
  \item If $p \neq 2$ or $17$, then one of $2, 17, 34$ is a quadratic residue modulo $p$ since
    \[
      \left(\frac{34}p\right) = \left(\frac{2}p\right) \left(\frac{17}p\right) .
    \]
  \end{enumerate}
Therefore at least one of Equations~\eqref{eqn:eqn3_day36}, \eqref{eqn:eqn5_day36}, \eqref{eqn:eqn5_day36} is solvable. 
\end{proof}

\subsection*{Exercise}
\begin{enumerate}
\item Let $k, m \in \Z$ such that $m^2 + 3 = 2k$. Is it possible to write $k$ as the sum of three squares? (\emph{Hint}: $m$ is odd.) 
\item Find all positive integers $m$ such that $n^4 + m$ is not a prime for any positive integer $n$. (\emph{Hint}: $m = 4k^4$, $k \in \Z$ is one candidate.)
% ko80, no. 13  
\item Let $a, b \in \Z$, $a, b > 0$. Show $x^{2a+1} = 2^b \pm 1$ has no integer solution when $x > 1$.
\end{enumerate}

\chapter[Lecture Thirty-Seven]{Day Thirty-Seven \hfill {\footnotesize \rm --- 24.04.2017}}

Consider the Diophantine equation $x^2 + y^2 = z^2$.

It has solutions in $\Z$, for example,
\begin{enumerate}[label=(\arabic*)]
\item $3^2 + 4^2 = 5^2$.
\item $1^2 + 0^2 = 1^2$. Or for any $a \in \Z$, we have $a^2 + 0^2 = a^2$.
\item $(-3)^2 + 4^2 = (-5)^2$.
\item $6^2 + 8^2 = {10}^2$.
\end{enumerate}

\begin{definition}
  Positive integers $x, y, z$ satisfying $x^2 + y^2 = z^2$ are called \textbf{Pythagorean triple}. We say a Pythagorean triple is \textbf{primitive} if $\gcd (x, y, z) = 1$. 
\end{definition}
\begin{remark}
  The name Pythagorean triple comes from the Pythagorean theorem, named after Greek mathematician Pythagoras. But the Pythagorean theorem was known to the Mesopotamian, Indian and Chinese all independently. Even with some indication it was known in ancient Babylon. 
\end{remark}

Let us understand primitive Pythagorean triples. Understanding primitive Pythagorean triples is enough since if $x, y, z$ is any Pythagorean triple and $d = \gcd (x, y, z)$, then 
\[
\left( \frac{x}d \right)^2 + \left( \frac{y}d \right)^2 = \left( \frac{z}d \right)^2.
\]
And $\frac{x}d, \frac{y}d, \frac{z}d$ is a primitive triple.

\begin{claim}\label{clm:1odd_1even_pythagorean_triple}
  one of $x,y$ is even and the other is odd in the primitive Pythagorean triple $x, y, z$.
\end{claim}
\begin{proof}
  We can consider the cases where both $x$ and $y$ are even or odd.
  \begin{enumerate}[label=(\roman*)]
  \item If $2 \mid x$ and $2 \mid y$, then $2 \mid x^2 + y^2$, i.e., $2 \mid z^2 = x^2 + y^2 \implies 2 \mid z$. $x, y, z$ is not a primitive triple.
  \item If $x$ and $y$ are both odd, then $x^2 \equiv \pm 1 \bmod 4$ and $y^2 \equiv \pm 1 \bmod 4$. So $x^2 + y^2 \equiv 2 \bmod 4$. But for any $z \in \Z$, 
    \begin{align}
      z^2 \equiv \left\{ \begin{array}{ll}
          0 \bmod 4 & \text{if $z$ is even,} \\[1mm]
          1 \bmod 4 & \text{if $z$ is odd.} 
        \end{array} \right.
    \end{align}
    We get a contradiction.
  \end{enumerate}
  
  Hence the claim.
\end{proof}

From this point on, without loss of generality, we can assume $x$ is odd, $y$ is even and we mainly consider primitive Pythagorean triples.
\begin{theorem}\label{thm:pythagorean_prim_triple}
  $x, y, z$ is a primitive Pythagorean triple with $x$ odd and $y$ even if and only if there are $m, n \in \Z$ with $m > n > 0$, $\gcd (m, n) = 1$, and exactly one of $m, n$ is even, the other is odd such that 
\[
x = m^2 - n^2, ~y = 2mn, \text{ and } z = m^2 + n^2.
\]
In particular, there are \textbf{infinitely many} primitive Pythagorean triples. 
\end{theorem}
\begin{example}
  Quick examples.
  \begin{itemize}
  \item If $m = 2$, $n = 1$, then $x = 3$, $y = 4$, $z = 5$ is a primitive Pythagorean triple.
  \item If $m = 3$, $n = 2$, then $x = 5$, $y = 12$, $z = 13$ is a primitive Pythagorean triple.
  \item If $m = 4$, $n = 1$, then $x = 15$, $y = 8$, $z = 17$ is a primitive Pythagorean triple.
  \end{itemize}
\end{example}
\begin{remark}
  If we change the above quadratic Diophantine equation $x^2 + y^2 = z^2$ into $x^2 + y^2 = 3 z^2$, then we can show it has no nontrivial solutions. Further by Fermat's last theorem, there are no solutions to $x^n + y^n = z^n$ with nonzero $x, y, z \in \Z$ for any $n \geqslant 3$.

  This illustrates the subtlety of Diophantine equations---a small variation can completely change the answer. 
\end{remark}
\begin{proof}[Proof {\rm (of Theorem~\ref{thm:pythagorean_prim_triple})}]
  ($\Leftarrow$) First let $m, n \in \Z$ with $m > n > 0$, $\gcd (m, n) = 1$, and exactly one of $m, n$ is even. Setting
  \[
    x = m^2 - n^2, ~y = 2mn, \text{ and } z = m^2 + n^2,
  \]
  we check if $x^2 + y^2 = z^2$. Indeed,
  \[
    x^2 + y^2 = m^4 - 2 m^2 n^2 + n^4 + 4 m^2 n^2 = m^4 + 2 m^2 n^2 + n^4 = z^2.
  \]
  Note that $x, y, z > 0$ by $m > n > 0$, $x, y, z$ is a Pythagorean triple. Note also that $y$ is even, $x$ is odd and exactly one of $m, n$ is even by our assumption. We need to check if $\gcd (x, y, z) = 1$.

  Let $d = \gcd (x, y, z)$, since $x$ is odd and $d \mid x$, $d$ is also odd. Since $d \mid x$, $d \mid z$, we have $d \mid x + z = 2m^2$ and $d \mid z - x = 2n^2$. $d \mid 2m^2$ and $d \mid 2n^2$ while at the same time, $\gcd (m, n) = 1 \implies \gcd (m^2, n^2) = 1$. Then we know $d$ comes from $2$. But $d$ is odd so $d = 1$.

  ($\Rightarrow$) Now assume $x, y, z$ is a primitive Pythagorean triple with $x$ odd, $y$ even. First notice $\gcd (x, y) = 1$. (Otherwise they share some odd common divisor such that an odd prime $p \mid x$ and $p \mid y$. It further implies $p \mid x^2 + y^2 = z^2$ therefore $x, y, z$ is not a primitive triple.)

  Since $x, y, z$ is primitive implies $\gcd (x, y) = 1$, similarly we get $\gcd (y, z) = \gcd (z, x) = 1$. Since $x$ is odd, $y$ is even, $z$ is odd and each of $y$, $z - x$ and $z + x$ is even, we get
  \begin{align}\label{eqn:square_prod_coprimes}
    \left( \frac{y} 2\right)^2 = \left( \frac{z - x}2 \right) \left( \frac{z +x } 2 \right),
  \end{align}
  where $\frac{z \pm x}2 \in \Z$.

  Let $e = \gcd \left( \frac{z - x}2, \frac{z + x}2 \right)$. Then
  \begin{align}
% use relation operator \mathrel{\bigg|} to emulate \mid 
    e \mathrel{\bigg|} \frac{z - x}2 + \frac{z + x}2 &\implies e \mid z, \\
    e \mathrel{\bigg|} \frac{z - x}2 - \frac{z + x}2 &\implies e \mid x.
  \end{align}
  But $\gcd (z, x) = 1$. Therefore $e = 1$, i.e., $\left( \frac{z - x}2 \right)$ and $\left( \frac{z + x}2 \right)$ are coprime. By fundamental theorem of arithmetic, Equation~\eqref{eqn:square_prod_coprimes} implies that there are $m, n \in \Z$, $m, n > 0$ such that
  \[
    \frac{z + x}2 = m^2, ~\frac{z - x}2 = n^2, \text{ and } \frac{y}2 = mn.
  \]
  Then $x = m^2 - n^2$, $z = m^2 + n^2$ and $y = 2mn$. Since $x > 0$, $x = m^2 - n^2 > 0 \implies m > n > 0$. Further,
  \[
  \gcd \left( \frac{z - x}2, \frac{z + x}2 \right) = \gcd (m^2, n^2) = 1, 
  \]
  which implies $\gcd (m, n) = 1$.
  Finally, since $x = m^2 - n^2$ is odd, exactly one of $m, n$ is even.
\end{proof}

\subsection*{Exercise}
\begin{enumerate}
% ko80, no. 14
\item Solve 
  \begin{align}
    \left\{ \begin{array}{r}
              x^3 + y^3 + z^3 = 3 \\
              x + y + z = 3
            \end{array}
    \right.
  \end{align}
  for all $x, y, z \in \Z$. (\emph{Hint}: There are four solutions: $(x, y, z) = (1,1,1)$, $(-5,4,4)$, $(4,-5,4)$, and $(4,4,-5)$.)
% ko80, no. 24  
\item Solve $3 \cdot 2^x + 1 = y^2$ for all $x, y \in \Z$. (\emph{Hint}: There are two solutions: $(x, y) = (3, 5)$, $(4, 7)$.)
% ko80, no. 34
\item Solve $y^2 = 1 + x + x^2 + x^3 + x^4$ for all $x, y \in \Z$. (\emph{Hint}: There are six solutions: $(x, y) = (-1, \pm 1)$, $(0, \pm 1)$, and $(3, \pm 11)$.)
% ko80, no. 57
\item Solve $x^2 + y^2 + z^2 = x^2 y^2$ for all $x, y, z \in \Z$. (\emph{Hint}: The only integer solutions are $x = y = z = 0$.)
\end{enumerate}


\chapter[Lecture Thirty-Eight]{Day Thirty-Eight \hfill {\footnotesize \rm --- 26.04.2017}}

Recall Theorem~\ref{thm:pythagorean_prim_triple} characterizes solutions to the Diophantine equation 
\[
x^2 + y^2 = z^2.
\]
% A natural follow-up question would be,
Q\underline{uestion}: What are the integer  solutions to $x^n + y^n = z^n$ for $n \geqslant 3$?
\begin{theorem}[Fermat's last theorem, Wiles and Taylor--Wiles, 1995]\label{thm:fermat_last_thm}
  For any $n \in \Z$, $n \geqslant 3$, 
\[
x^n + y^n = z^n
\]
has \textbf{no} solutions with $x, y, z$ nonzero integers.   
\end{theorem}
\begin{remark}
For $n \in \Z$, $n > 0$, if $x, y, z \in \Z$ are nonzero integers such that $x^n + y^n = z^n$, then for any $d \mid n$, $d > 0$, $x^{\frac{n}d}, y^{\frac{n}d}, z^{\frac{n}d}$ are nonzero integers satisfying 
\[
\left( x^{\frac{n}d} \right)^d + \left( y^{\frac{n}d} \right)^d = \left( z^{\frac{n}d} \right)^d.
\]

So to prove there are no solutions when $n \geqslant 3$, it suffices to show Fermat's last theorem holds for $n = 4$ and $n$ odd primes.\footnote{The reduction follows from the fact that for any divisor of $n$, if $x^d + y^d = z^d$ has no solutions then $x^n + y^n = z^n$ cannot have solutions. For powers of 2, $n = 2^k$, $k \geqslant 2$, it reduces to the case $n = 4$. For other $n$'s, it suffices to ensure $x^p + y^p = z^p$ has no solutions, where $p \mid n$, $p$ an odd prime.}  
\end{remark}

The general timeline of the progress towards the proof:
\begin{itemize}
\item Germain (1820s) proved that for all odd primes $p < 197$, $x^p + y^p = z^p$ has no solutions with $x, y, z$ coprime with $p$.
\item Lam\'e in 1847 gave a false proof based on the following ideas: 

Let $\zeta = e^{\frac{2 \pi i}p} \in \C$ with $p$ a fixed odd prime. In the previous lecture on Pythagorean triples, we looked at $y^2 = (z - x) (z + x)$. If $x^p + y^p = z^p$, $x, y, z \in \Z$ then
\begin{align}
  y^p &= z^p - x^p \\
      &= \prod_{i = 1}^p (z - \zeta^i x) \text{ in $\C$. }
\end{align}
Lam\'e showed that Fermat's last theorem is true if the number ring 
\[
\Z[\zeta] = \left\{ a_0 + a_1 \zeta + \cdots + a_{p-1} \zeta^{p-1} : a_i \in \Z \right\} \subseteq \C
\]
holds unique factorization of primes, i.e., fundamental theorem of arithmetic holds. But this is false in general (pointed out by Kummer).\footnote{Kummer is one of the ``$K3$'' after whose names $K3$-surface is named. The other two are K\"ahler and Kodaira.}
\item Special case of Faltings (1983) is that $x^p + y^p = z^p$ has finitely many nonzero solutions.
\item The proof was finally given by Wiles and Taylor--Wiles in 1995 built upon the work of Frey, Serre and Ribet in the 1980s.
\end{itemize}

\emph{Annals of Mathematics} dedicated a special issue to the Wiles proof and it ran over 100 pages. We only prove $n = 4$ case in this course. 

Note if $x, y, z$ are nonzero integers such that 
\[
x^4 + y^4 = z^4 = (z^2)^2,
\]
then it suffices to prove 
\begin{theorem}
There are no nontrivial integers solutions to $x^4 + y^4 = z^2$.
\end{theorem}
\begin{proof}
We prove by Fermat's infinite descent. Assume otherwise, then replacing $x, y, z$ with their negation if necessary, we can assume $x, y, z > 0$. We can also assume $\gcd (x, y) = 1$.\footnote{Let $d = \gcd (x, y)$. Since $d \mid x$ and $d \mid y$, it implies $d^4 \mid x^4 + y^4 = z^2$. It further implies $d^2 \mid z$. Write $x = x' d$, $y = y' d$, and $z = z' d^2$, then $x^4 + y^4 = z^2$ becomes $x'^4 + y'^4 = z'^2$.}

Now assume there exists a triple $x_1, y_1, z_1 \in \Z$ such that $x_1^4 + y_1^4 = z_1^2$, $x_1, y_1, z_1 > 0$, $\gcd (x_1, y_1) = 1$, and $z_1$ is the smallest possible. We want to construct another solution $x_2, y_2, z_2$ with $x_2, y_2, z_2 > 0$, $\gcd (x_2, y_2) = 1$, and $z_2 < z_1$. This will yield a contradiction.

Note that $x_1^2, y_1^2, z_1$ is a Pythagorean triple. It is also primitive since $\gcd (x_1, y_1) = 1$ implies $\gcd (x_1^2, y_1^2) = 1$.\footnote{$\gcd (a, b, c) = 1 \notimply \gcd (a, b) = 1$ necessarily but $\gcd (a, b) = 1 \implies \gcd (a, b, c) = 1$.} By Claim~\ref{clm:1odd_1even_pythagorean_triple} we saw last time exactly one of $x_1^2, y_1^2$ is even, and swap $x_1, y_1$ if necessary, we can assume $y_1^2$ is even, $x_1^2$ is odd, i.e., $x_1$ is odd, $y_1$ is even.

By Theorem~\ref{thm:pythagorean_prim_triple}, there are $m, n \in \Z$ with $m > n > 0$, $\gcd (m, n) = 1$, and exactly one of $m, n$ is even such that 
\[
x_1^2 = m^2 - n^2, ~y_1^2 = 2mn, \text{ and } z_1 = m^2 + n^2.
\]
Then $x_1^2 + n^2 = m^2 \implies x_1, n, m$ is another Pythagorean triple. It is primitive since $\gcd (m, n) = 1$. Since $x_1$ is odd, by Theorem~\ref{thm:pythagorean_prim_triple} again, $n$ is even, $m$ is odd. Repeat, there are $a, b \in \Z$ with $a > b > 0$, $\gcd (a, b) = 1$ and exactly one of $a, b$ even such that 
\[
x_1 = a^2 - b^2, ~n = 2ab, \text{ and } m = a^2 + b^2.
\]
Our target is to show that $m, a, b$ are squares. Indeed, we have $y_1^2 = (2n)m$, $\gcd (m, 2n) = 1$ since $\gcd (m, n) = 1$, $m$ is odd. By fundamental theorem of arithmetic, $m$ and $2n$ are squares. Since $2n$ is even and a square, there is a $c \in \Z$ such that $(2c)^2 = 2n$, which implies $n = 2c^2$. $n = 2 ab = 2c^2 \implies ab = c^2$. But $\gcd (a, b) = 1$, again by fundamental theorem of arithmetic, $a, b$ are both squares.

We showed that there are $x_2, y_2, z_2 \in \Z$, $x_2^2 = a$, $y_2^2 = b$, and $z_2^2 = m$. Replacing $x_2, y_2, z_2$ with their negations if necessary, we can assume $x_2, y_2, z_2 > 0$ since $a, b, m > 0$. We have $\gcd (x_2, y_2) = 1$ since $\gcd (a, b) = 1$.

Finally, $z_2^2 = m = a^2 + b^2 = x_2^4 + y_2^4$. And $z_2 \leqslant z_2^2 = m \leqslant m^2 = m^2 + n^2 = z_1$, a contradiction.
\end{proof}

\subsection*{Exercise}
\begin{enumerate}
% ko80, no. 25  
\item Show $x^n + 1 = y^{n+1}$ has no integer solutions in $x, y, n$ with $\gcd (x, n+1) = 1$ and $n > 1$.
\end{enumerate}
  
  
  
-- End of this course.\footnote{Administrative announcement: Final exam covers material up to this point.}

\chapter[Lecture Thirty-Nine]{Day Thirty-Nine \hfill {\footnotesize \rm --- 28.04.2017}}

\underline{Non-elementar}y\underline{ number theor}y\\

For the ambitious, where does elementary number theory go beyond this point?

\underline{Anal}y\underline{tic theor}y (Math 531)
We have seen the prime number theorem in Lecture~\ref{chp:prime_num_theorem_gcd}, which states for $\pi (x) = \left| \, \left\{ p : p \leqslant x, p \text{ primes} \right\}\, \right|$, we have 
\[
\pi (x) \sim \frac{x}{\log x},
\]
i.e., $\displaystyle \lim_{x \to \infty} \frac{\pi (x)}{\frac{x}{\log x}} = 1$.

Q\underline{uestion}: How is this proved?

We can rephrase the prime number theorem as $\pi (x) = \frac{x}{\log x} + o\left( \frac{x}{\log x} \right)$, where $o\left( \frac{x}{\log x} \right)$ is the error term.\footnote{The little-$o$ notation $f(x) = o(g(x))$ involving functions $f(x), g(x)$ means $\displaystyle \lim_{x \to \infty} \frac{f(x)}{g(x)} = 1$.}

Then another question is could we get a better error term?

The classic proof is an application of results from complex analysis, though an elementary proof does exist. (See~\cite{Selberg1949} for the elementary proof.) 

In calculus, we studied convergence of infinite series, e.g., $\sum_{n = 1}^\infty a_n$.
\begin{example}
$\sum_{n = 1}^\infty \frac{1}n$ diverges. 
\end{example}
\begin{proof}[Sketch of Proof]\renewcommand*{\qedsymbol}{\ensuremath{\blacksquare}}
Consider integral $\int_1^\infty \frac{1}x \, dx = \ln x \,|_1^\infty$ and use comparison test.
\end{proof}
For $s \in \R$, if $s > 1$, the series $\sum_{n = 1}^\infty \frac{1}{n^s}$ however, converges. We define a function $\zeta (s)$, called \textbf{Riemann $\zeta$-function} on $\left\{ s \in \R : s > 1 \right\}$ by 
\[
\zeta (s) = \sum_{n = 1}^\infty \frac{1}{n^s}.
\]
\begin{example}
$\zeta (2) = \frac{ \pi^2 }6$.
\end{example}
\begin{proof}[Sketch of Proof]\renewcommand*{\qedsymbol}{\ensuremath{\blacksquare}}
  This is called \textbf{Basel problem}. See its wikipedia entry and stack exchange discussion.\footnote{\url{https://math.stackexchange.com/questions/8337/different-methods-to-compute-sum-limits-k-1-infty-frac1k2}}
\end{proof}
A fact is that the infinite series defined in Riemann $\zeta$-function also makes sense and converges as a complex number for any $s = x + i y \in \C$ with $x > 1$, $x, y \in \R$. As illustrated in the Figure~\ref{fig:zeta_converge}, $\zeta(s)$ is defined on the shaded region. This function is very nice---it is continuous and holomorphic on the defined domain. (Holomorphic means complex differentiable, cf. Math 448.)

\underline{A less eas}y\underline{ fact}: There is a unique way to extend $\zeta(s)$ to a holomorphic function on $\left\{ s \in \C : s \neq 1 \right\}$.
% illustrations and figures; 
% pstrick fig
\begin{figure}[phtb]
  \begin{center}
    \begin{pspicture}(-1.5,-1.5)(3.2,5.3) % frame size : (x1, y1)(x2, y2)
      %\psline*[linecolor=lightgray!35](1,-1.4)(1,5)(2.8,5)(2.8,-1.4)(1,-1.4)
      \psset{gradbegin=lightgray!35,gradend=white} % TODO: gradient not working
      \psTextFrame*[fillstyle=gradient,ref=c,opacity=0.1](1,-1.4)(2.8,5){$\Re (s) > 1$}
%      \psTextFrame*[linecolor=lightgray!35,ref=c,opacity=0.7](1,-1.4)(2.8,5){$\Re (s) > 1$}
      \psTextFrame*[linecolor=white,ref=bl](-1,4.85)(0,4){\Large $\C$}
      \psline[linestyle=dashed,linewidth=0.5pt](1,-0.1)(1,4.9)
      \psline[linestyle=dashed,linewidth=0.5pt](1,-1.3)(1,-0.5)
      \psaxes[Dx=1,Dy=1,tickstyle=top,ticksize=2.5pt,labels=x]{->}(0,0)(-1.4,-1.4)(3.1,5)[$\Re$,-90][$\Im$,0] % origin, x,y axes, arrow ->
    \end{pspicture}
  \end{center}
  \caption{Region of convergence of $\zeta(s)$ on the complex plane.}
  \label{fig:zeta_converge} % placed at the end
\end{figure}
\begin{theorem}
  The prime number theorem is equivalent to the statement
  \[
    \zeta(1 + i y) \neq 0 \text{ for any $y \neq 0$.}
  \]
\end{theorem}
This is how prime number theorem is usually proved.
\begin{figure}[phtb]
  \begin{center}
    \begin{pspicture}(-1.5,-1.5)(3.2,5.5) % frame size : (x1, y1)(x2, y2)
      \psset{gradbegin=lightgray!35,gradend=white} % TODO: gradient not working
      \psTextFrame*[fillstyle=gradient,ref=c,opacity=0.1](1.07,-1.4)(2.8,5){$\Re (s) \neq 1$}
      \psframe*[fillstyle=gradient,opacity=0.1](0.93,-1.4)(-1.25,5)
%      \psTextFrame*[linecolor=lightgray!35,ref=c,opacity=0.7](1,-1.4)(2.8,5){$\Re (s) > 1$}
      \psTextFrame*[linecolor=white,ref=bl](-1,5.15)(0,5){\Large $\C$}
      \psline[linestyle=dashed,linewidth=0.5pt](1,-0.1)(1,4.9)
      \psline[linestyle=dashed,linewidth=0.5pt](1,-1.3)(1,-0.5)
      \psaxes[Dx=1,Dy=1,tickstyle=top,ticksize=2.5pt,labels=x]{->}(0,0)(-1.4,-1.4)(3.1,5.3)[$\Re$,-90][$\Im$,0] % origin, x,y axes, arrow ->
    \end{pspicture}
  \end{center}
  \caption{Extending $\zeta(s)$ on the entire complex plane except $s = 1$.}
  \label{fig:zeta_converge} % placed at the end
\end{figure}

Q\underline{uestion}: How is $\zeta(s)$ connected to primes?
\begin{theorem}[Euler]
  For $s = x + iy$, $x > 1$, we have
  \[
    \zeta(s) = \sum_{n = 1}^\infty \frac{1}{n^s} = \prod_{\text{\rm $p$ prime}} \left( 1 - \frac{1}{p^s} \right)^{-1}.
  \]  
\end{theorem}
\begin{conjecture}[Riemann hypothesis]
  If $s = x + iy$ with $x \geqslant \frac{1}2$ satisfies $\zeta(s) = 0$, then $x = \frac{1}2$.
\end{conjecture}
The Riemann hypothesis is equivalent to a best possible error term for the prime number theorem. It says nontrivial zeros of Riemann $\zeta$-function are concentrated on $x = \frac{1}2$ as seen in Figure~\ref{fig:riemann_hypothesis}.
\begin{figure}[phtb]
  \begin{center}
    \begin{pspicture}(-1.5,-1.5)(3.2,5.3) % frame size : (x1, y1)(x2, y2)
      \psTextFrame*[linecolor=white,ref=bl](-1,4.75)(0,5){\Large $\C$}
      \psline[linestyle=dashed,linewidth=0.5pt](2,-0.1)(2,4.8)
      \psline[linestyle=dashed,linewidth=0.5pt](2,-1.3)(2,-0.5)
      \psline[linestyle=dotted,linewidth=0.9pt](1,-0.1)(1,4.8)
      \psline[linestyle=dotted,linewidth=0.9pt](1,-1)(1,-0.6)
      \psTextFrame*[linecolor=white,ref=c](0.5,-1)(1.5,-1.5){$x = \frac{1}2$} \psaxes[Dx=0.5,Dy=1,dx=1,tickstyle=top,ticksize=2.5pt,labels=x]{->}(0,0)(-1.4,-1.4)(3.1,4.9)[$\Re$,-90][$\Im$,0] % origin, x,y axes, arrow ->
    \end{pspicture}
  \end{center}
  \caption{Nontrivial zeros of Riemann $\zeta$-function are concentrated on $x = \frac{1}2$.}
  \label{fig:riemann_hypothesis} % placed at the end
\end{figure}

Let
\[
  \text{Li}(x) = \int_2^x \frac{1}{\log t}\, dt,
\]
we can show $\text{Li}(x) \sim \frac{x}{\log x}$.
\begin{theorem}[Koch, 1901]
  The Riemann hypothesis is equivalent to
  \[
    \pi (x) = \text{Li}(x) + \mathcal{O}(\sqrt x \log x).
  \]
\end{theorem}

\underline{Al}g\underline{ebraic theor}y (Math 530)

A number $\alpha$ is called \textbf{algebraic} if it is a root of polynomial
\[
  x^n + a_{n-1} x^{n-1} +  \cdots + a_1 x + a_0
\]
with $a_i \in \Q$.
\begin{example}
  Quick examples.
  \begin{itemize}
  \item $\sqrt 2$ is algebraic since it is a root of $x^2 - 2$.
  \item $\pi$ is \underline{not} algebraic. It is not\footnote{Lindemann in 1882 gave the first proof. He first argued $e^a$ is transcendental if $a$ is algebraic, nonzero. By Euler's identity $e^{i \pi} = -1$ is algebraic. Therefore $i \pi$ must be transcendental. See~\cite{Lindemann1882} for details.} obvious why it is not.
  \end{itemize}
\end{example}
A number $\alpha$ is an \textbf{algebraic integer} if it is a root of a polynomial 
\[
  x^n + a_{n-1} x^{n-1} +  \cdots + a_1 x + a_0
\]
with $a_i \in \Z$.
\begin{example}
  Quick examples.
  \begin{itemize}
  \item $\sqrt 2$ is an algebraic integer.
  \item $e^{\frac{2 \pi i}n}$ with $n \geqslant 1$ is an algebraic integer since it is a $n$-th root of polynomial $x^n - 1$.
  \end{itemize}
\end{example}

For $\alpha$ an algebraic integer, we study the number ring
\[
  \Z[\alpha] = \left\{ a_0 + a_1 \alpha + \cdots + a_n \alpha^n : n \geqslant 1, a_i \in \Z \right\}.
\]
We can add and multiply elements of $\Z [\alpha]$.
\begin{example}\label{ex:prime_Za}
  Primality tests in polynomial time use arithmetic in $\Z \left[ e^{\frac{2 \pi i}n} \right]$. We can define a prime element in $\Z [\alpha]$ by saying $x \in \Z [\alpha]$ is prime whenever $x \mid yz \implies x \mid y$ or $x \mid z$.
\end{example}

Q\underline{uestion}: With the notion of prime in Example~\ref{ex:prime_Za}, when does fundamental theorem of arithmetic hold in $\Z [\alpha]$? When is a prime $p$ still prime in $\Z [\alpha]$?
\begin{theorem}
  A prime $p$ is prime in $\Z [i]$ if and only if $p \equiv 3 \bmod 4$.
\end{theorem}
\begin{remark}
  This result is very closely related to the first supplement to quadratic reciprocity, i.e., $-1$ is a square modulo an odd prime $p$ if and only if $p \not\equiv 3 \bmod 4$. Generalizations of quadratic reciprocity go via this topic, i.e., asking how primes decompose or factor in rings like $\Z [\alpha]$.
\end{remark}
\begin{theorem}[Heegner--Stark]
  There are only finitely many square free integers $d < 0$ such that fundamental theorem of arithmetic holds in $\Z [\sqrt d]$.
\end{theorem}
\begin{conjecture}
  There are infinitely many square free integers $d > 0$ such that fundamental theorem of arithmetic holds in $\Z [\sqrt d]$.
\end{conjecture}

\chapter[Lecture Forty]{Day Forty \hfill {\footnotesize \rm --- 01.05.2017}}

\underline{Exercise session}
\begin{proof}[Exercise 1.]\renewcommand*{\qedsymbol}{}
  Let $a_1, a_2, \ldots, a_n, b \in \Z$. Consider the linear Diophantine equation
  \begin{align}
    a_1 x_1 + a_2 x_2 + \cdots + a_n x_n = b. \label{eqn:lin_diophantine}
  \end{align}

  Show that
  \begin{enumerate}[label=(\arabic*)]
  \item if Equation~\eqref{eqn:lin_diophantine} has a solution $x_i \in \Z$, then $\gcd (a_1, a_2, \ldots, a_n) \mid b$;
  \item if $\gcd (a_1, a_2, \ldots, a_n) \mid b$, then Equation~\eqref{eqn:lin_diophantine} has infinitely many solutions. \label{itm:itm2_ex1_day40}
  \end{enumerate}

  (\emph{Hint}: For Part~\ref{itm:itm2_ex1_day40}, we can induct on $n$.)
\end{proof}

\begin{proof}[Exercise 2.]\renewcommand*{\qedsymbol}{}
  Let $p$ be a prime and $d \mid p - 1$. Let
  \[
    f(d) = \left\vert \, \left\{ 1 \leqslant a \leqslant p-1 : \ord_p (a) = d \right\} \, \right\vert.
  \]

  To prove primitive roots modulo $p$ exist, we showed $f(d) = \varphi (d)$. This exercise gives another proof using M\"obius inversion.
  \begin{enumerate}[label=(\arabic*)]
  \item Show $\sum_{\substack{c\, \mid \, d \\c\, > \,0}} f(c) = d$.
  \item Show $f(d) = \sum_{\substack{c\, \mid \, d \\c\, > \,0}} \mu (c) \frac{d}c$.
  \item Show $f(d) = \varphi (d)$.
  \end{enumerate}

  (\emph{Hint}: This problem is the same as \cite{Strayer2001}*{\S 5.2, Problem 20}.)
\end{proof}

\begin{proof}[Exercise 3.]\renewcommand*{\qedsymbol}{}
  Let $n, d$ be positive integers with $d \mid n$.
  Show that
  \begin{enumerate}[label=(\arabic*)]
  \item for any $a \in \Z$ with $\gcd (a, d) = 1$, there is a $b \in \Z$ with $\gcd (b, n) = 1$ such that $b \equiv a \bmod d$;
  \item if $r$ is a primitive root modulo $n$, then $r$ is also a primitive root modulo $d$. \label{itm:itm2_ex3_day40}
  \end{enumerate}

  (\emph{Hint}: Part~\ref{itm:itm2_ex3_day40} showed up before as Corollary~\ref{cor:prim_rt_n_prim_rt_d}. See also a stack exchange discussion.\footnote{\url{https://math.stackexchange.com/questions/2307411/if-g-is-a-primitive-root-modulo-n-then-g-is-a-primitive-root-modulo-d-wher}} )
\end{proof}

  
%%%%%%%%%%%%%%%%%%%%%%%%%%%%%%%%%%%%%%%%%%%%%%%%%%%%%%%%%%%
%  bibliography 
%%%%%%%%%%%%%%%%%%%%%%%%%%%%%%%%%%%%%%%%%%%%%%%%%%%%%%%%%%%

% \bibliographystyle{amsalpha} 
% \bibliography{ENT40Days}

% use the direct way - add bib division right within the main file
\begin{bibdiv}
\begin{biblist}

\bib{Agrawal2004}{article}{
  ISSN = {0003486X},
  URL = {https://www.jstor.org/stable/3597229},
  DOI = {10.4007/annals.2004.160.781},
  MR = {2123939},
  author = {Manindra Agrawal and Neeraj Kayal and Nitin Saxena},
  journal = {Annals of Mathematics},
  number = {2},
  pages = {781-793},
  publisher = {Annals of Mathematics},
  title = {PRIMES Is in P},
  volume = {160},
  year = {2004}
}
  
\bib{Green2004}{article}{
  author = {Green, Ben},
  author = {Tao, Terence},
  journal = {Annals of Mathematics},
  number = {2},
  % issue = {2},
  pages = {481-547},
  publisher = {Annals of Mathematics},
  title = {The primes contain arbitrarily long arithmetic progressions},
  volume = {167},
  year = {2008}
}

\bib{Kleinjung2017}{misc}{
  author = {Thorsten Kleinjung},
  author = {Claus Diem and Arjen K. Lenstra and Christine Priplata and Colin Stahlke},
  title = {Computation of a 768-bit prime field discrete logarithm},
  howpublished = {Cryptology ePrint Archive, Report 2017/067},
  year = {2017},
  note = {\url{http://eprint.iacr.org/2017/067}}
}

\bib{Knuth1997}{book}{
 author = {Knuth, Donald E.},
 title = {The Art of Computer Programming, Volume 1 (3rd Ed.): Fundamental Algorithms},
 year = {1997},
 isbn = {0-201-89683-4},
 publisher = {Addison Wesley Longman Publishing Co., Inc.},
 address = {Redwood City, CA, USA}
} 

\bib{Ko1980}{book}{
  author = {Ko, Chao},
  title = {100 Problems in Elementary Number Theory},
  year = {1980},
  publisher = {Education Press of Shanghai}
}

\bib{Ko2001}{book}{
  author = {Ko, Chao},
  title = {Lectures on Elementary Number Theory},
  year = {2001},
  volume = {1},
  publisher = {Higher Education Press}
}

\bib{Lindemann1882}{article}{
  author={Lindemann, Ferdinand von},
  title={Ueber die Zahl $\pi$},
  journal={Mathematische Annalen},
  year={1882},
  volume={20},
  number={2},
  pages={213--225},
  issn={1432-1807},
  doi={10.1007/BF01446522},
  url={http://dx.doi.org/10.1007/BF01446522}
}

\bib{Moschovakis2004}{article}{
author = {Moschovakis, Yiannis N.},
author = {van den Dries, Lou}, % different author keys for multiple authors
doi = {10.2178/bsl/1102022663},
fjournal = {Bulletin of Symbolic Logic},
journal = {Bull. Symbolic Logic},
month = {09},
number = {3},
pages = {390--418},
publisher = {Association for Symbolic Logic},
title = {Is the Euclidean algorithm optimal among its peers?},
url = {http://dx.doi.org/10.2178/bsl/1102022663},
volume = {10},
year = {2004}
}

\bib{Rousseau1991}{article}{
  title = {On the quadratic reciprocity law}, 
  volume = {51}, 
  DOI = {10.1017/S1446788700034583}, 
  number = {3}, 
  journal={Journal of the Australian Mathematical Society. Series A. Pure Mathematics and Statistics}, 
  publisher = {Cambridge University Press}, 
  author = {Rousseau, G.}, 
  year = {1991}, 
  pages = {423–425}
}

\bib{Selberg1949}{article}{
  ISSN = {0003486X},
  URL = {http://www.jstor.org/stable/1969455},
  author = {Atle Selberg},
  journal = {Annals of Mathematics},
  number = {2},
  pages = {305-313},
  publisher = {Annals of Mathematics},
  title = {An Elementary Proof of the Prime-Number Theorem},
  volume = {50},
  year = {1949}
}

\bib{Strayer2001}{book}{
  title={Elementary Number Theory},
  author={Strayer, James K.},
  isbn={9781478610403},
  url={https://books.google.com/books?id=dYAfAAAAQBAJ},
  year={2001},
  publisher={Waveland Press}
}

\bib{Weber1995}{article}{
 author = {Weber, Kenneth},
 title = {The Accelerated Integer GCD Algorithm},
 journal = {ACM Trans. Math. Softw.},
 volume = {21},
 number = {1},
 year = {1995},
 issn = {0098-3500},
 pages = {111--122},
 numpages = {12},
 url = {http://doi.acm.org/10.1145/200979.201042},
 doi = {10.1145/200979.201042},
 acmid = {201042},
 publisher = {ACM},
 address = {New York, NY, USA},
 keywords = {GCD, integer greatest common divisor, number-theoretic computations},
} 

\end{biblist}
\end{bibdiv}


\end{document}


%%% Local Variables:
%%% mode: latex
%%% TeX-master: t
%%% TeX-command-extra-options: "-shell-escape"
%%% End:
